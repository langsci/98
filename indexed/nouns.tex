\section{Nouns}\label{sec:4}



\subsection{Overview}\label{sec:4.1}


Single Icétôd \textsc{nouns} in a speaker’s mental lexicon consist of a \textsc{root.} Roots are words that cannot be analyzed into smaller parts from the perspective of modern Icétôd. (Historical research may reveal how roots were put together over time, but that is the domain of etymology.) When plucked from the lexicon and put into actual Icétôd speech, every noun root must receive at least one suffix, which must be a \textsc{case} suffix. Every noun root ends in a vowel, and case suffixes either delete or attach to this final vowel. In addition to case suffixes, an Icétôd noun may take on a \textsc{number} suffix or may be joined with one or two other nouns to form a \textsc{compound}. Case suffixes are fully explained later in \sectref{sec:7}, while number suffixes and compounds are covered in the rest of this chapter.

Icétôd number suffixes include \textsc{pluratives} and \textsc{singulatives}. Many noun roots can be pluralized if they are inherently singular in number. A few others can be singularized because they are inherently plural. In addition to these standard number-markers, Icétôd also has special \textsc{possessive} number suffixes that combine the notions of number and \isi{possession} into one suffix. And yet other nouns are \textsc{mass} \textsc{nouns}, naming entities in the world perceived as inherently plural unities (like dust or water). These take no suffixes but are treated grammatically as plurals. Finally, some nouns are \textsc{transnumeral}, construed as singular or plural and given the appropriate singular or plural modifiers, if needed. 

Compounding is the primary way Icétôd acquires or makes new nouns – besides borrowing them from other languages. Icétôd compounds are made by putting two or three nouns together into a new composite word with special emergent characteristics. The first noun describes or specifies the second noun to make an aggregate meaning that is often different than that of the two separate nouns. Compounding and types of compounds are discussed below in \sectref{sec:4.3}.

Icétôd nominal suffixes differ individually in how they affix to noun roots. With the exception of five case suffixes, all nominal suffixes first delete the final vowel of the noun to which they attach. This is known as \textsc{subtractive} morphology. The case suffixes that preserve the final vowel are the accusative, \isi{dative}, genitive, ablative, and oblique. For more on how case suffixes attach to nouns, see \sectref{sec:7}.




\subsection{Number}\label{sec:4.2}
\subsubsection{Pluratives (\textsc{plur})}\label{sec:4.2.1}

Icétôd has four ways to show that a noun is plural: three \textsc{plurative} suffixes and \textsc{suppletive} plurals. The three \isi{plurative} suffixes are: 1) \{-íkó-\}, 2) \{-ítíní-\}, and 3) \{-ìkà-\}. The first \isi{plurative} suffix, \{-íkó-\}, is dominant in terms of \isi{vowel harmony}, meaning it changes the vowels of a [-\isi{ATR}] noun to [+\isi{ATR}] unless /a/ intervenes and blocks it. For example, in some instances, the vowel /a/ spontaneously appears between the singular root and the suffix \{-íkó-\}. (This /a/ is a relic of an ancient \isi{singulative} suffix *\textit{{}-at-} that is no longer in use in current Icétôd.)

The use of \{-íkó-\} is limited to a small number of nouns (roughly 100); it is not applied to newly borrowed nouns. \tabref{tab:nouns:pl:iko} presents several examples of nouns pluralized with this suffix. Note how the suffix harmonizes the vowels of the singular root except where the vowel /a/ blocks the leftward spread of harmony. Notice also that in some cases the suffix alters the tone of the singular root.


\begin{table}[p]
\caption{The \isi{plurative} suffix \{-íkó-\}}
\label{tab:nouns:pl:iko}


\begin{tabularx}{\textwidth}{XXXl}
\lsptoprule

Singular &  & Plural & \\
\midrule
abérí- & → & áberaikó- & ‘active termite colonies’\\
baratsó- & → & barátsíkó- & ‘mornings’\\
cúrúkù- & → & cúrúkaikó- & ‘bulls’\\
kɔr\'{ɔ}b\`{ɛ}- & → & kɔr\'{ɔ}baikó- & ‘calves’\\
ƙwɛs\'{ɛ}\`{ɛ}- & → & ƙwéséikó- & ‘broken gourds’\\
mɔƙɔr\'{ɔ}- & → & moƙóríkó- & ‘rock wells’\\
taɓá- & → & taɓíkó- & ‘boulders’\\
\lspbottomrule
\end{tabularx}
\end{table}
The second \isi{plurative}, \{-ítíní-\}, is used to pluralize nouns that have only two syllables in their root. \tabref{tab:nouns:pl:itini} gives a sample of disyllabic nouns pluralized with \{-ítíní-\}. Notice that if the singular noun has [-\isi{ATR}] vowels, then the \isi{plurative} suffix harmonizes to \{-{\Í}t{\Í}n{\Í}-\}. Unlike the suffix \{-íkó-\}, \{-ítíní-\} never alters the tone of the root, though its own tone may conform to the tone of the root.


\begin{table}[p]
\caption{The \isi{plurative} suffix \{-ítíní-\}}
\label{tab:nouns:pl:itini}


\begin{tabularx}{\textwidth}{XXXX}
\lsptoprule

Singular &  & Plural & \\
\midrule
aká- & → & akɨt{\Í}n{\Í}- & ‘mouths’\\
bòsì- & → & bositíní- & ‘ears’\\
\'{ɔ}ʝá- & → & \'{ɔ}ʝ{\Í}t{\Í}n{\Í}- & ‘sores’\\
ɗòlì & → & ɗólítíní- & ‘carcasses’\\
ekú- & → & ekwitíní- & ‘eyes’\\
ídò- & → & íditíní- & ‘breasts’\\
tsʼ\'{ʉ}bà- & → & tsʼ\'{ʉ}bɨt{\Í}n{\Í}- & ‘stoppers’\\
\lspbottomrule
\end{tabularx}
\end{table}
The third \isi{plurative}, \{-ìkà-\}, is used primarily to pluralize nouns with three or more syllables in their lexical root. \tabref{tab:nouns:pl:ika} provides a sample of polysyllabic nouns pluralized with \{-ìkà-\}. Notice that if the singular noun has [-\isi{ATR}] vowels, then the \isi{plurative} suffix harmonizes to \{-{\Ì}kà-\}. Like \{-íkó-\}, \{-ìkà-\} sometimes alters the tone of the singular noun as well as having its own tone altered.


\begin{table}[p]
\caption{The \isi{plurative} suffix \{-ìkà-\} with polysyllabic nouns}
\label{tab:nouns:pl:ika}


\begin{tabularx}{\textwidth}{XXXl}
\lsptoprule

Singular &  & Plural & \\
\midrule
àg{\Ì}tà- & → & ág{\Ì}t{\Ì}kà- & ‘metal ringlets’\\
arírá- & → & aríríkà- & ‘flames’\\
bàbàà- & → & bábàìkà- & ‘armpits’\\
ɔfɔrɔƙ\'{ɔ}- & → & ɔf\'{ɔ}r\'{ɔ}ƙ{\Ì}kà- & ‘dry honeycombs’\\
kútúŋù- & → & kútúŋìkà- & ‘knees’\\
ɲánɨn\'{ɔ}\`{ɔ}- & → & ɲánɨn\'{ɔ}{\Ì}kà- & ‘leather whips’\\
ɲéƙúrumotí- & → & ɲéƙúrùmòtìkà- & ‘gullies’\\
\lspbottomrule
\end{tabularx}
\end{table}
Secondarily, the \isi{plurative} \{-ìkà-\} is used to pluralize a few nouns that have only two syllables in their lexical root. Why these few nouns do not take \{-ítíní-\} as a \isi{plurative} instead is not known. A bit of speculation on this point might invoke the notion of \textsc{mora} or the unit of \isi{syllable} weight. Among the seven examples shown in \tabref{tab:nouns:pl:ika2}, three of them contain the \isi{semi-vowel} /w/ which may be thought to contain its own \isi{mora}, as a vowel would. Likewise, two of the examples (\textit{hòò-} and \textit{sédà-}) contain depressor consonants which may also count for one \isi{mora}. Perhaps in the remaining two (\textit{kíʝá-} and \textit{ríʝá-}), the voiced stop /ʝ/ used to be a \isi{depressor consonant}. Regardless of the historical explanation, \tabref{tab:nouns:pl:ika2} presents a few examples of \{-ìkà-\} being used to pluralize disyllabic nouns.


\begin{table}
\caption{The \isi{plurative} suffix \{-ìkà-\} with disyllabic nouns}
\label{tab:nouns:pl:ika2}


\begin{tabularx}{\textwidth}{XXXl}
\lsptoprule

Singular &  & Plural & \\
\midrule
awá- & → & àwìkà- & ‘homes’\\
gwasá- & → & gwàsìkà- & ‘stones’\\
hòò- & → & hòìkà- & ‘huts’\\
kíʝá- & → & kíʝíkà- & ‘lands’\\
kwɛtá- & → & kw\`{ɛ}t{\Ì}kà- & ‘arms’\\
ríʝá- & → & ríʝíkà- & ‘forests’\\
sédà- & → & sédìkà- & ‘gardens’\\
\lspbottomrule
\end{tabularx}
\end{table}

\subsubsection{Suppletive plurals}\label{sec:4.2.2}

Icétôd also has a handful of singular nouns that cannot be pluralized in a productive way with any of the three suffixes discussed above. Three of these nouns on record are truly \textsc{suppletive} in that their singular and plural forms bear absolutely no resemblance to each other. These are the first three in \tabref{tab:nouns:pl:sup}. The last three examples in \tabref{tab:nouns:pl:sup} represent nouns that are semi-suppletive; even though one can discern a similarity between the singular and plural forms, the way the two forms are derived from each other is not productive in the language:  


\begin{table}
\caption{Icétôd suppletive plurals}
\label{tab:nouns:pl:sup}


\begin{tabularx}{\textwidth}{XXXX}
\lsptoprule

Singular &  & Plural & \\
\midrule
ámá- & $\leftrightarrow $ & ròɓà- & ‘people’\\
eakwá- & $\leftrightarrow $ & ɲɔt\'{ɔ}- & ‘men’\\
imá- & $\leftrightarrow $ & wicé- & ‘children’\\
cekí- & $\leftrightarrow $ & cɨkámá- & ‘women’\\
ɗɨ{}- & $\leftrightarrow $ & ɗi- & ‘ones’\\
k\'{ɔ}r\'{ɔ}ɓádì- & $\leftrightarrow $ & kúrúɓádì- & ‘things’\\
\lspbottomrule
\end{tabularx}
\end{table}

\subsubsection{Singulatives (\textsc{sing})}\label{sec:4.2.3}

In contrast to pluratives, \textsc{singulatives} convert an inherently plural noun root to a derived singular. Icétôd has one such suffix that may be considered a true \isi{singulative} in the contemporary grammar of the modern language, and that is \{-àmà-\} or \{-\`{ɔ}mà-\}. Since this \isi{singulative} is only used with personal entities, it seems likely that it is related etymologically to the word \textit{ámá-} ‘person’. \tabref{tab:nouns:sing} gives the only four unambiguous examples of when this \isi{singulative} is used. Note that its tone pattern may be altered by the tone of the plural root:


\begin{table}
\caption{The Icétôd \isi{singulative} \{-àmà-\}}
\label{tab:nouns:sing}


\begin{tabularx}{\textwidth}{XXXX}
\lsptoprule

Plural &  & Singular & \\
\midrule
ʝáká- & → & ʝákámà- & ‘elder’\\
kéà- & → & kéàmà- & ‘soldier’\\
lɔŋ\'{ɔ}tá- & → & lɔŋ\'{ɔ}t\'{ɔ}mà- & ‘enemy’\\
ŋ{\Í}m\'{ɔ}kɔkaá- & → & ŋ{\Í}m\'{ɔ}kɔká-ámà- & ‘young man’\\
\lspbottomrule
\end{tabularx}
\end{table}

\subsubsection{Possessive number suffixes (\textsc{poss})}\label{sec:4.2.4}

In addition to standard pluratives and a \isi{singulative}, Icétôd has what may be called \textsc{possessive} number suffixes. These possessive suffixes – \{-èdè-\} in the singular and \{-ìnì-\} in the plural – each fuse the notions of number and \isi{possession} into one morpheme. When they are affixed to a noun, they specify a) the grammatical number of the noun and b) its association with another entity (hence the ‘\isi{possession}’). They do not specify the number of the \isi{possessor}(s). For example, the word \textit{aked\ᵃ}, a stem consisting of \textit{aká-} ‘den’ and \{-èdè-\} (\isi{nominative case}) can mean both ‘its den’ or ‘their den’. And the word \textit{akɨn}, consisting of \textit{aká-} ‘den’ and \{-ìnì-\} (\isi{nominative case}), can mean either ‘its dens’ or ‘their dens’. 

Within the broad notion of ‘\isi{possession}’, the possessive number suffixes \{-èdè-\} and \{-ìnì-\} can signify more specific semantic relationships like part-whole, kinship, and association. \tabref{tab:nouns:posssg} gives some examples of \{-èdè-\} expressing a part-whole relationship with the unnamed entity. Note how the meanings of the noun roots are extended metaphorically to denote structural parts of things. Note also that the tone of the root may be altered in the presence of \{-èdè-\}.


\begin{table}
\caption{The Icétôd singular possessive \{-èdè-\}}
\label{tab:nouns:posssg}


\begin{tabularx}{\textwidth}{lXXll}
\lsptoprule

\multicolumn{2}{X}{Root} &  & \multicolumn{2}{X}{Part-whole}\\
\midrule
bakutsí- & ‘chest’ & → & bakútsédè- & ‘its middle part’\\
bùbùì- & ‘belly’ & → & búbùèdè- & ‘its underside’\\
ekú- & ‘eye’ & → & ekwede- & ‘its essence’\\
kwayó- & ‘tooth’ & → & kweede- & ‘its edge’\\
ŋabérí- & ‘rib’ & → & ŋábèrèdè- & ‘its side’\\
\lspbottomrule
\end{tabularx}
\end{table}
The plural possessive suffix \{-ìnì-\} has two special applications with human possessors. In the first, it is used to pluralize kinship terms, where a kinship association is explicit. In the second, it refers to people associated with a certain person in general terms. \tabref{tab:nouns:posspl} illustrates both of these nuances, showing the singular root in the first column, and in the second, the root plus \{-ìnì-\}.


\begin{table}
\caption{The Icétôd plural possessive \{-ìnì-\}}
\label{tab:nouns:posspl}
\begin{tabularx}{\textwidth}{XXXl}
\lsptoprule
Kinship &  &  & \\
\midrule
abáŋ{\Ì}- & → & abáŋ{\Í}n{\Í}- & ‘my fathers (uncles)’\\
dádòò- & → & dádoíní- & ‘your grandmothers’\\
ŋ\'{ɔ}\`{ɔ}- & → & ŋɔ{\Í}n{\Í}- & ‘your mothers’\\
tátàà- & → & tátaíní- & ‘my aunts’\\
wicé- & → & wikini- & ‘his/her/their/its children’\\
\tablevspace
Association &  &  & \\
\midrule
\`{A}ɗùpàà- & → & Aɗupaíní- & ‘the people of Aɗupa’\\
Daká{\Ì}- & → & Dakáɨn{\Í}- & ‘the people of Dakai’\\
Lóʝérèè- & → & Lóʝéreíní- & ‘the people of Loʝere’\\
Ŋirikoó- & → & Ŋirikoíní- & ‘the people of Ŋiriko’\\
Tsɨláà- & → & Tsɨláɨn{\Í}- & ‘the people of Tsila’\\
\lspbottomrule
\end{tabularx}
\end{table}

\subsubsection{Mass nouns}\label{sec:4.2.5}

A small group of Icétôd noun roots are classified as non-count \textsc{mass} \textsc{nouns}. These nouns are inherently, lexically plural. As such, they require plural demonstratives and relative pronouns. This group includes words for powders, liquids, and gases various particulate substances. \tabref{tab:nouns:mass} presents seven examples of mass nouns. The roots are in the table's first column, followed in the third column by the noun in a phrase with the plural demonstrative \textit{ni} ‘those’. Note that in the English gloss, the equivalent is provided but with a singular interpretation:


\begin{table}
\caption{Icétôd non-countable mass nouns}
\label{tab:nouns:mass}
\begin{tabularx}{\textwidth}{XXXX}
\lsptoprule
búré- & ‘dust’ & búrá ni & ‘this dust’\\
cué- & ‘water’ & cua ni & ‘this water’\\
kabasá- & ‘flour’ & kabasa ni & ‘this flour’\\
sèà- & ‘blood’ & sea ni & ‘this blood’\\
tsʼúdè- & ‘smoke’ & tsʼúda ni & ‘this smoke’\\
\lspbottomrule
\end{tabularx}
\end{table}

\subsubsection{Transnumeral nouns}\label{sec:4.2.6}

Another small group of Icétôd noun roots are inherently \textsc{transnumeral}, meaning that they can be singular or plural depending on the speaker's intention. Whatever number is imputed to them must be reflected in the grammar of the rest of the sentence, for example in subject-agreement on the verb or in any demonstratives or relative pronouns used to modify them. Icétôd transnumeral nouns cannot be pluralized in any of the ways discussed up to this point. But with the bound nominal morpheme \textit{{}-icíká-} (see \sectref{sec:4.3.4}), they can be given a sense of distributiveness or variation. \tabref{tab:nouns:trans} presents three examples of Icétôd transnumeral nouns with their singular, plural, and distributive interpretations:


\begin{table}
\caption{Icétôd transnumeral nouns}
\label{tab:nouns:trans}
\begin{tabularx}{\textwidth}{XXl}
\lsptoprule
Root & ɓìɓà- & ‘egg(s)’\\
Singular & ɓiɓa na & ‘this egg’\\
Plural & ɓiɓa ni & ‘these eggs’\\
Distributive & ɓiɓaicíká- & ‘various kinds of eggs’\\
% \midrule
\tablevspace
Root & gwaá- & ‘bird(s)’\\
Singular & gwaa na & ‘this bird’\\
Plural & gwaa ni & ‘these birds’\\
Distributive & gwaicíká- & ‘various kinds of birds’\\
% \midrule
\tablevspace
Root & ínó- & ‘animal(s)’\\
Singular & ínwá na & ‘this animal’\\
Plural & ínwá ni & ‘these animals’\\
Distributive & ínóicíká- & ‘various kinds of animals’\\
\lspbottomrule
\end{tabularx}
\end{table}



\subsection{Compounds}\label{sec:4.3}


For word-building purposes, Icétôd relies heavily on \textsc{compounding}, joining two or more nouns together into a new composite word. The first noun (or pronoun) in a compound retains its lexical root form (that is hyphenated throughout this book), including its lexical tone. The last noun in a compound takes whichever case ending the syntactic context calls for. For example, in the compound \textit{riéwík\ᵃ} ‘goat kids’, the first root \textit{rié-} ‘goat’ keeps its lexical form, while the second, \textit{wicé-} ‘children’, has been modified by the \isi{nominative case} suffix \{-ᵃ\}. If compounding changes the tone of its constituent parts, it will be the first noun that affects the others. In the rare compound with three constituent nouns, the first two stay in their lexical form (not counting tone), while the third is inflected for case, for example in \textit{Icémóríɗókàkà-} ‘cowpea leaves’, a compound of \textit{Icé-} ‘Ik’, \textit{mòrìɗò-} ‘beans’, and \textit{kaká-} ‘leaves’. In \textit{Icé-móríɗó-kàkà-}, note that while the last two elements retain their lexical segments, their tone patterns have changed dramatically due to the influence of \textit{Icé-} in spreading H tone throughout the word.

Icétôd compounds create two kinds of new meaning: 1) a narrower, more specific meaning in which the first noun specifies the second, or 2) a completely novel, unpredictable meaning. An example of the first type would be \textit{bʉbʉn\'{ɔ}\'{ɔ}ʝà-} ‘ember-wound’ or ‘bullet wound’ where the first noun \textit{bʉbʉná-} ‘ember’ narrows down the possible references of \textit{\'{ɔ}ʝá-} ‘wound’ to a wound caused by a bullet. And an example of the second type of compounded meaning would be \textit{óbiʝoetsʼí-}, a compound that literally means ‘rhino urine’ but is actually the name of a species of vine (that nonetheless was apparently the favorite urination spot of rhinos). Through both types of meaning-making, Icétôd compounds add a considerable amount of expressiveness and color to the language’s vocabulary.

In addition to the two broader semantic categories of compounds discussed above, five other categories of Icétôd compounds are recognized. These include the \isi{agentive}, \isi{diminutive}, internal, variative, and relational, all discussed in the sections to follow.


\subsubsection{Agentive (\textsc{agt})}\label{sec:4.3.1}
\largerpage[-]1
Icétôd forms \textsc{agentive} compounds by using the root \textit{ámá-} ‘person’ (for singular) or \textit{icé-} (for plural) as the last element in a compound. Although the root \textit{Icé-} simply means ‘Ik people’ when standing on its own, in the \isi{agentive} construction it denotes plural agents. Here ‘agent’ is understood broadly as any person or thing that does or is whatever is characterized by the first element in the compound. The first element may be a noun, as in \textit{dɛá-ámà-} ‘messenger’, literally ‘foot-person’, or a verb as in \textit{ŋwàx\`{ɔ}n{\Ì}-àmà-} ‘lame person’, literally ‘to be lame-person’. Note, however, that even though \textit{ŋwàx\`{ɔ}n} is a verb semantically, it has been deverbalized into a noun by the \isi{infinitive} suffix \{-òn\}. Icétôd \isi{agentive} compounds can be translated into English in various ways, depending on what is appropriate. \tabref{tab:nouns:agt} presents several examples of \isi{agentive} compounds:


\begin{table}
\caption{Icétôd \isi{agentive} compounds}
\label{tab:nouns:agt}


\begin{tabularx}{\textwidth}{XXXX}
\lsptoprule

Singlar & Plural &  & \\
\midrule
aká-ámà- & aká-ícé- & mouth-person & ‘talker’\\
ɓɛƙ\'{ɛ}s{\Í}-àmà- & ɓɛƙ\'{ɛ}sí-ícé- & walking-person & ‘traveler’\\
itelesí-ámà- & itelesí-ícé- & watching-person & ‘watchman’\\
kɔŋ\'{ɛ}s{\Í}-àmà- & kɔŋ\'{ɛ}sí-ícé- & cooking-person & ‘cook’\\
ɲósomá-ámà- & ɲósomá-ícé- & studies-person & ‘student’\\
sɨsɨká-ámà- & sɨsɨká-ícé- & middle-person & ‘middle child’\\
yʉ\'{ɛ}-ámà- & yué-ícé- & lie-person & ‘liar’\\
\lspbottomrule
\end{tabularx}
\end{table}

\subsubsection{Diminutive (\textsc{dim})}\label{sec:4.3.2}
\largerpage[-1]
Icétôd forms \textsc{diminutive} compounds by using the root \textit{imá-} ‘child’ (for singular) and \textit{wicé-} ‘children’ (for plural) as the second element in a compound. In the more literal interpretation, the first element is the animate being (animal or human) of which the second element is the ‘child’ or ‘children’, as in \textit{ɗóɗò-ìmà-} ‘lamb’ or \textit{ɗóɗo-wicé-} ‘lambs’. But when the first element is inanimate, the \isi{diminutive} construction conveys a sense of ‘a small X’ or ‘small Xs’, for example \textit{ƙɔfó-ìmà-} ‘a small gourd bowl’ and \textit{ƙɔfó-wicé-} ‘small gourd bowls’. Lastly, the two interpretations can also get blurred, as when an animate being is perceived as smaller than normal but not as the child of anything. This can be seen, for instance, in the compound \textit{ídèmè-ìmà-} ‘earthworm’, literally ‘snake-child’. \tabref{tab:nouns:dim} offers several more examples of the \isi{diminutive} compound. Notice that when the whole construction is pluralized, both elements may get pluralized, as when \textit{ámá-ìmà-} ‘someone’s child’ becomes \textit{roɓa-wicé-} ‘someone’s (pl.) children’:


\begin{table}
\caption{Icétôd \isi{diminutive} compounds}
\label{tab:nouns:dim}


\begin{tabularx}{\textwidth}{XXXX}
\lsptoprule

Singular & Plural &  & \\
\midrule
ámá-ìmà- & roɓa-wicé- & person-child & ‘someone’s child’\\
bàrò-ìmà- & bár{\Í}t{\Í}ní-wicé- & herd-child & ‘small herd’\\
ɓɨsá-ímà- & ɓ{\Í}s{\Í}t{\Í}ní-wicé- & spear-child & ‘dart’\\
dómá-ìmà- & dómítíní-wicé- & pot-child & ‘small pot’\\
gwá-ímà- & gwá-wícé- & bird-child & ‘chick’\\
ŋókí-ìmà- & ŋókítíní-wicé- & dog-child & ‘puppy’\\
\'{ɔ}ʝá-ìmà- & \'{ɔ}ʝ{\Í}t{\Í}n{\Í}-wicé- & sore-child & ‘small sore’\\
\lspbottomrule
\end{tabularx}
\end{table}


\subsubsection{Internal (\textsc{int})}\label{sec:4.3.3}

So-called \textsc{internal} compounds are made with the bound nominal root \textit{aʝ{\Í}ká-} ‘among/inside’. When appended to plural noun, this nominal conveys a sense of interiority or internality to the noun. The internal compound, which occurs relatively rarely, is exemplified in \tabref{tab:nouns:intern}:


\begin{table}
\caption{Icétôd internal compounds}
\label{tab:nouns:intern}


\begin{tabularx}{\textwidth}{XXXll}
\lsptoprule

Plural &  &  & Internal & \\
\midrule
àwìkà- & ‘homes’ & → & awika-aʝ{\Í}ká- & ‘in/among homes’\\
ríʝíkà- & ‘forests’ & → & ríʝíka-aʝ{\Í}ká- & ‘in/among forests’\\
sédìkà- & ‘gardens’ & → & sédika-aʝ{\Í}ká- & ‘in/among gardens’\\
\lspbottomrule
\end{tabularx}
\end{table}

\subsubsection{Variative (\textsc{var})}\label{sec:4.3.4}

So-called \textsc{variative} compounds are made with the bound nominal root \textit{icíká-} ‘various (kinds of)’. When appended to a noun – singular or plural – this nominal morpheme communicates a sense of variety or the multiplicity of a type. As a kind of pluralizer itself, \textit{icíká-} is may be called upon to pluralize five kinds of nouns: 1) transnumeral nouns, 2) nouns not usually pluralizeable in the usual sense, 3) inherently plural nouns, 4) already pluralized nouns, and 5) verb infinitives. \tabref{tab:nouns:var} presents one example for each of these five kinds of nouns that the variative bound nominal \textit{icíká-} can be used to pluralize:


\begin{table}
\caption{Icétôd variative compounds}
\label{tab:nouns:var}


\begin{tabularx}{\textwidth}{XXXll}
\lsptoprule

\multicolumn{2}{X}{Singular/Plural} &  & \multicolumn{2}{X}{Variative}\\
\midrule
gwaá- & ‘bird(s)’ & → & gwa-icíká- & ‘kinds of birds’\\
cɛmá- & ‘fights’ & → & cɛmá-ícíká- & ‘war’\\
mɛná- & ‘issues’ & → & mɛná-ícíká- & ‘various issues’\\
dakwítíní- & ‘trees’ & → & dakwítíní-icíká- & ‘kinds of trees’\\
wetésí- & ‘to drink’ & → & wetésí-icíká- & ‘drinks’\\
\lspbottomrule
\end{tabularx}
\end{table}

\subsubsection{Relational}\label{sec:4.3.5} 

Icétôd compounding is also used to create \textsc{relational nouns} that express the spatial or structural relationship one thing has to another. In this way, Icétôd metaphorically extends body-part terms to other non-bodily relationships. \tabref{tab:nouns:rel1} presents some of the Icétôd body-part terms used metaphorically:


\begin{table}
\caption{Icétôd body-part terms with extended meanings}
\label{tab:nouns:rel1}


\begin{tabularx}{\textwidth}{XXX}
\lsptoprule

Root & Lexical meaning & Relational meaning\\
\midrule
aká- & ‘mouth’ & ‘entrance, opening’\\
aƙatí- & ‘nose’ & ‘handle, stem’\\
bakutsí- & ‘chest’ & ‘front part’\\
bùbùì- & ‘belly’ & ‘underside’\\
dɛá- & ‘foot’ & ‘base, foot’\\
ekú- & ‘eye’ & ‘center, point’\\
gúró- & ‘heart’ & ‘core, essence’\\
iká- & ‘head’ & ‘head, top’\\
kwayó- & ‘tooth’ & ‘edge’\\
ŋabérí- & ‘rib’ & ‘side’\\
\lspbottomrule
\end{tabularx}
\end{table}
So then, in an Icétôd relational compound, terms like those in \tabref{tab:nouns:rel1} form the second element in the compound, a position in which they denote the ‘part’ in a ‘whole-part’ semantic relationship. Accordingly, the first element in the relational compound represents the ‘whole’ in the structural relationship. \tabref{tab:nouns:rel2} displays a handful of such ‘whole-part’ compounds:


\begin{table}
\caption{Icétôd relational compounds}
\label{tab:nouns:rel2}


\begin{tabularx}{\textwidth}{XXX}
\lsptoprule

Roots & Lexical meaning & Relational meaning\\
\midrule
aká-kwáyó- & mouth-tooth & ‘lip’\\
dáŋá-àkà- & termite-mouth & ‘termite mound hole’\\
dòɗì-èkù- & vagina-eye & ‘cervix’\\
fátára-bakutsí- & ridge-chest & ‘front of vertical ridge’\\
fetí-ékù- & sun-eye & ‘east’\\
kaiɗeí-áƙátí- & pumpkin-nose & ‘pumpkin stem’\\
kwará-d\`{ɛ}à- & mountain-foot & ‘base of mountain’\\
kwaré-ékù- & mountain-eye & ‘saddle between peaks’\\
taɓá-d\`{ɛ}à- & boulder-foot & ‘base of boulder’\\
tsʼaɗí-ákà- & fire-mouth & ‘flame’\\
\lspbottomrule
\end{tabularx}
\end{table}

