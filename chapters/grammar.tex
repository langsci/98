
\section{Introduction}


Although the bulk of this book is dedicated to the Icétôd-English dictionary and English-Icétôd reversal index, the following section offers an overview sketch of Icétôd grammar. The sketch covers most important features of the grammatical system but only to a shallow depth. Those who wish to dig deeper are encouraged to consult the fuller treatment published as \textit{A grammar of Icé-tód: Northeast Uganda’s last thriving Kuliak language} \citep{Schrock2014}, which is freely available for download from several websites on the internet. 

Linguistic concepts are most easily defined with linguistic terminology. Thus, due to limitations of time and space, this sketch of Icétôd grammar is geared in style toward the general linguist. That said, an aim has been to clearly define some of the key terms used and to describe the grammatical structures in simple, clear language. Unfortunately, some of the discussion may remain opaque to any non-linguist readers. For this, I offer my sincere apologies. I am very willing to clarify or explain in layman’s terms any point raised in this grammar sketch. Feel free to contact me at: \href{mailto:betsoniik@gmail.com}{betsoniik@gmail.com}.

The grammar sketch begins with a description of the language’s sound system (phonology) and then proceeds to words and word-building strategies (morphology). It ends with a very shallow dip into syntax. Because of its length, technical discussion, and many sections and subsections, the sketch is probably most useful as a reference tool. However, should the reader (especially the language-learner) have time, it may prove beneficial to read the sketch from front to back. Doing so would provide a bird’s-eye view of the whole system.

Learning any language from printed sources alone is rarely ideal. Ideally, every learner would have the chance to soak up language naturally as children do. Sadly, most adult learners do not have that luxury. I recommend mixing approaches to suit one’s personality, learning style, schedule, and responsibilities. Studying grammar from a book like this one will not appeal to everyone, yet all learners will occasionally get stuck on points of grammar during the course of their learning. Just as the foregoing dictionary can help you fill in gaps where specific words need to be, this grammar sketch can help fill in holes in your understanding of how Icétôd works. If it should succeed at all in that regard, all my efforts will have been proven worthwhile. 
 
\section{Phonology: the sound system}
 
\subsection{Consonants and vowels}


Icétôd has an array of thirty consonants\textsc{} and nine vowels. These are presented in \tabref{tab:2}.1 below. In the table’s first column are shown the alphabetical letters used to represent these sounds. The second column shows the phonetic symbol for the sound used by the International Phonetic Alphabet (IPA). Then in the third column, an approximate English equivalent is given in bold typeface, or else an explanation of how the sound is made if there is no English approximation:


\begin{table}
\caption{1: Icétôd sound inventory}
\label{tab:2}

\begin{tabularx}{\textwidth}{XXX}
\lsptoprule

Alphabetic & Phonetic & English equivalent\\
Aa & [a] & as in ‘f\textbf{a}ther’\\
Bb & [b] & as in ‘\textbf{b}oy’\\
Ɓɓ & [ɓ] & as an English \textbf{b} but with air sucked in\\
Cc & [tʃ] & as in ‘\textbf{ch}ild’\\
Dd & [d̻] & as in ‘\textbf{d}aughter’\\
Ɗɗ & [ɗ] & as an English \textbf{d} but with air sucked in\\
Dzdz & [ʣ̻] & as in ‘a\textbf{dz}e’\\
Ee & [e] & as in ‘b\textbf{ai}t’ with a shorter, crisper sound\\
Ɛɛ & [ɛ] & as in ‘b\textbf{e}t’\\
Ff & [f] & as in ‘\textbf{f}ood’\\
Gg & [ɡ] & as in ‘\textbf{g}ood’\\
Hh & [h] & as in ‘\textbf{h}appy’\\
Hyɦy & [ɦʲ] & as an English \textbf{h} but with a raspy sound\\
Ii & [i] & as in ‘b\textbf{ea}t’ with a shorter, crisper sound\\
Ɨɨ & [ɪ] & as in ‘b\textbf{i}t’\\
Jj & [ʤ] & as in ‘\textbf{j}oy’\\
Jʼʝ & [ʄ] & as a \textbf{dy} sound but with air sucked in\\
Kk & [k] & as in ‘\textbf{k}arma’\\
Ƙƙ & [kʼ]

[ɠ] & 1) as an English \textbf{k} with a popping release

2) as an English \textbf{g} with air sucked in\\
Ll & [l] & as in ‘\textbf{l}ove’\\
Mm & [m] & as in ‘\textbf{m}an’\\
Nn & [n̻] & as in ‘\textbf{n}ature’\\
Ɲɲ & [ɲ] & as in ‘o\textbf{ni}on’\\
Ŋŋ & [ŋ] & as in ‘si\textbf{ng}’\\
Oo & [o] & as in ‘b\textbf{oa}t’ with a shorter, crisper sound\\
Ɔɔ & [ɔ] & as in ‘b\textbf{ough}t’\\
Pp & [p] & as in ‘\textbf{p}lay’\\
Rr & [ɾ]

[r] & 1) as a Spanish or Swahili flapped \textbf{r}

2) as a Spanish or Swahili trilled \textbf{r}\\
Ss & [s] & as in ‘\textbf{s}orrow’\\
Tsts & [ʦ] & as in ‘bli\textbf{tz}’\\
Tsʼtsʼ & [ʦʼ] & as an English \textbf{ts}/\textbf{tz} with a hissing release\\
Tt & [t̻] & as in ‘\textbf{t}error’\\
Uu & [u] & as in ‘b\textbf{oo}t’\\
Ʉʉ & [ʊ] & as in ‘p\textbf{u}t’\\
Ww & [w] & as in ‘\textbf{w}onder’\\
Xx & [ʃ] & as in ‘\textbf{sh}oulder’\\
Yy & [j] & as in ‘\textbf{y}es’\\
Zz & [z] & as in ‘\textbf{z}ebra’\\
Ʒʒ & [ʒ] & as in ‘plea\textbf{s}ure’\\
\lspbottomrule
\end{tabularx}

\end{table}

Those sounds in \tabref{tab:2}.1 that have a small square under the IPA symbol are pronounced with the tip of the tongue a bit farther forward than in English. Especially [d̻], [n̻], and [t̻] are affected; sometimes they are fronted so much they touch the back of the front teeth. It is important not to pronounce [d̻] exactly like and an English ‘d’ as this sounds more like the Icétôd sound [ɗ] which contrasts with [d̻]. The sounds [ɓ, ɗ, ɠ, ʝ] are called \textsc{imposives} because they are made by ‘imploding’ or sucking air into the mouth rather than expelling air from the lungs. The sounds [kʼ] and [tsʼ] are called \textsc{ejectives} because they are made by ejecting air from the throat cavity instead of from the lungs. Lastly, the sound [ɦʲ], unlike an [h], is made with the vocal chords vibrating, giving it a raspy, throaty sound. It only occurs at the beginning of words. The nine Icétôd vowels—[a, e, ɛ, i, ɪ, ɔ, o, ʊ, u]—operate in a vowel harmony system, which is discussed in §2.5.
 
\subsection{Consonant devoicing}


At the end of an Icétôd word, if silence immediately follows, voiced consonants are slightly devoiced. In other words, they sound more like unvoiced consonants in that environment. This is similar to German, where the word \textit{Tag} ‘day’ is pronounced as [tak]. Consonant devoicing most noticeably affects /d/ and /g/ in Icétôd, as when \textit{êd} ‘name’ sounds like [êt] or when \textit{hɛg} ‘marrow’\textit{} sounds like [hɛk]. 
 
\subsection{Vowel devoicing}


Icétôd vowels are also devoiced before silence (a pause of any significant length). This is important to keep in my since every Icétôd word in every grammatical context—without exception—ends in a vowel. If that final vowel is not immediately followed by another sound, then it is pronounced in a whispered way. After certain consonants, namely /f, m, n, ɲ, ŋ, r, s, z, ʒ/, the vowel may be totally inaudible. The latter is not a hard-and-fast rule but rather a general tendency. It has become a tradition in scholarly writing on Icétôd to write whispered vowels with the raised symbols < ͥ, ᶤ, ᵉ, ᵋ, ᵃ, ᵓ, ᶶ, ᵘ >.
 
\subsection{Morphophonology}
\subsubsection{Deaffrication}

The affricates\textsc{} /c/ and /j/ are sometimes deaffricated or ‘hardened’ into their non-affricate counterparts /k/ and /g/, respectively. This is not a general phonological tendency in the language but is, rather, limited to a small handful of words. Moreover, the principle is applied in different ways to different words. For instance, in the word \textit{muceé-} ‘path, way’, the /c/ is hardened to /k/ when the word is used in the instrumental case (see §7.7): \textit{muko} ‘on the way’. The plural inclusive pronoun \textit{ɲjíní-} ‘we all (including addressees)’ is pronounced idiosyncratically as \textit{ŋgíní-} by a minority of speakers. Thirdly, when the words \textit{Icé-} ‘Ik people’ and \textit{wicé-} ‘children’ are declined for the nominative or instrumental cases, their /c/ hardens to /k/. This can be clearly seen in a case declension, like the one in \tabref{tab:2}.2 below. Note that, as explained later in §2.4.3, cases have non-final and final forms:


\begin{table}
\caption{2: Case declension of \textit{Icé-} ‘Ik’ and \textit{wicé-} ‘children’}
\label{tab:2}


\begin{tabularx}{\textwidth}{XXXXX} & \multicolumn{2}{X}{‘Ik’} & \multicolumn{2}{X}{‘children’}\\
\lsptoprule
& Non-final & Final & Non-final & Final\\
Nominative & Ika & Ikᵃ & wika & wikᵃ\\
Accusative & Icéá & Icékᵃ & wicéá & wicékᵃ\\
Dative & Icéé & Icékᵉ & wicéé & wicékᵉ\\
Genitive & Icéé & Icé & wicéé & wicé\\
Ablative & Icóó & Icéᵒ & wicóó & wicéᵒ\\
Instrumental & Ico/Iko & Icᵒ/Ikᵒ & wico/wiko & wicᵒ/wikᵒ\\
Copulative & Icóó & Icékᵒ & wicóó & wicékᵒ\\
Oblique & Ice & Ice & wice & wice/wicᵉ\\
\lspbottomrule
\end{tabularx}
\end{table}


\subsubsection{Haplology}

In Icétôd, when a consonant in one morpheme is made at the same place of articulation as a consonant in the next morpheme, \textsc{haplology} may occur—the deletion of the first of the two similar consonants. One example of this involves the venitive\textsc{} suffix \{-ét-\} and the andative\textsc{} suffix \{-uƙot-\}, both of which end in /t/. If another suffix containing alveolar /t/, /d/, or /s/ is attached to either of these, their final /t/ may be omitted. To illustrate this, \tabref{tab:2}.3 below presents a conjugation of the verb \textit{ŋatɛtɔn} ‘to run this way’. Notice how the /t/ in \{-ét-\} disappears from the suffix in the forms for \textsc{2sg} (‘you’), \textsc{1pl.inc} (‘we all’), and \textsc{2pl} (‘you all’). The 3\textsc{pl} form (‘they’) is an exception as it does not drop its final /t/ in the same environment:


\begin{table}
\caption{3: Haplology in \textit{ŋatɛtɔn} ‘to run this way’}
\label{tab:2}


\begin{tabularx}{\textwidth}{XXXXX}
\lsptoprule

\textsc{1sg} & ŋat-ɛt-ɨ &  & ŋat-ɛt-ɨ & ‘I run this way.’\\
\textsc{2sg} & ŋat-ɛt-ɨd & → & ŋat-ɛ{}-ɨd & ‘You run this way.’\\
\textsc{3sg} & ŋat-ɛt &  & ŋat-ɛt & ‘S/he runs this way.’\\
\textsc{1pl.exc} & ŋat-ɛt-ɨm &  & ŋat-ɛt-ɨm & ‘We run this way.’\\
\textsc{1pl.inc} & ŋat-ɛt-ɨsɨn & → & ŋat-ɛ{}-ɨsɨn & ‘We all run this way.’\\
\textsc{2pl} & ŋat-ɛt-ɨt & → & ŋat-ɛ{}-ɨt & ‘You all run this way.’\\
\textsc{3pl} & ŋat-ɛt-át &  & ŋat-ɛt-át & ‘They run this way.’\\
\lspbottomrule
\end{tabularx}
\end{table}

A second example of haplology occurs when a verb root ending in /g/, /k/, or /ƙ/ is followed directly by the andative\textsc{} suffix \{-uƙot-\}. When this happens, the final (\textsc{velar}) consonant of the verb root gets omitted in anticipation of the velar /ƙ/ in \{-uƙot-\}. \tabref{tab:2}.4 illustrates this by listing a few verbs ending in /g/, /k/, or /ƙ/, consonants which disappear when the next morpheme is the andative\textsc{} suffix \{-uƙot-\}:


\begin{table}
\caption{4: Haplology in verbs ending in a velar consonant}
\label{tab:2}


\begin{tabularx}{\textwidth}{XXXX}
\lsptoprule

ɦyɔtɔg-ʉƙɔt- & → & ɦyɔtɔ-ɔƙɔt- & ‘go near’\\
iɓók-uƙot- & → & iɓó-óƙot- & ‘shake off’\\
ɨpák-ʉƙɔt- & → & ɨpá-áƙot- & ‘swipe off’\\
kɔk-ʉƙɔt- & → & kɔ-ɔƙɔt- & ‘close up’\\
ŋƙáƙ-uƙot- & → & ŋƙá-áƙot- & ‘eat up’\\
oƙ-uƙot- & → & o-oƙot- & ‘put aside’\\
torík-uƙot- & → & torí-íƙot- & ‘lead away’\\
\lspbottomrule
\end{tabularx}
\end{table}

\subsubsection{Non-final consonant deletion}

Icétôd makes a clear distinction between \textsc{non-final} and \textsc{final} forms of all morphemes and words. Presumably this is to delineate syntactic boundaries, often with stylistic overtones. Non-final forms are those that occur within a string of speech, with at least one element immediately following them. Final forms, by contrast, are those that occur at the end of a string of speech, before a pause, with nothing immediately following. This basic distinction was already shown to affect the voicing of vowels in §2.3 above. In the case of a small number of morphemes, it also affects consonants. \tabref{tab:2}.5 presents a few of these morphemes whose final forms contain consonants that are omitted in their non-final forms. The first column of the table shows the underlying form (\textsc{uf}) of the morpheme in question. This is followed in the next two columns by the non-final (\textsc{nf}) and final (\textsc{ff}) forms that actually occur in speech. Notice how the non-final forms are missing one consonant that is fully present in the \textsc{uf} and the \textsc{ff}:


\begin{table}
\caption{5: Consonant deletion in non-final forms}
\label{tab:2}


\begin{tabularx}{\textwidth}{XXXX}
\lsptoprule

\textsc{uf} & \textsc{nf} & \textsc{ff} & Morpheme description\\
{}-ka & {}-a & {}-kᵃ & accusative case suffix\\
{}-ke & {}-e & {}-kᵉ & dative case suffix\\
{}-ko & {}-o & {}-kᵒ & copulative case suffix\\
{}-\'{} ka & {}-\'{} a & {}-\'{} kᵃ & present perfect suffix\\
{}-\'{} de & {}-\'{} e & {}-\'{} dᵉ & dummy pronoun suffix\\
nákà & náà & nákᵃ & ‘earlier today’\\
bàtsè & bèè & bàtsᵉ & ‘yesterday’\\
nòkò & nòò & nòkᵒ & ‘long ago’\\
ʝɨkɛ & ʝɨɨ & ʝɨkᵋ & ‘also, too’\\
ɲákà & ɲáà & ɲákᵃ & ‘just’\\
\lspbottomrule
\end{tabularx}
\end{table}

\subsubsection{Vowel assimilation}

In addition to consonants, Icétôd vowels also undergo phonological changes at the boundaries of morphemes. For instance, when two dissimilar vowels come in contact with each other as a result of two morphemes joining together, there is a powerful urge for them to become more like each other. This \textsc{vowel assimilation} was already seen at work in \tabref{tab:2}.4, as when putting the root \textit{torík-} ‘lead’ and affix \textit{{}-uƙot-} ‘away’ together led to \textit{tor}\textit{íí}\textit{ƙot-} instead of *\textit{tor}\textit{íú}\textit{ƙot-}. It is also seen in \tabref{tab:2}.5 where the ‘yester-’ adverb \textit{bàtsè} becomes \textit{b}\textit{èè}\textit{} in its non-final form instead of *\textit{b}\textit{àè}. Icétôd vowel assimilation only takes place between morphemes and not inside morphemes. Inside morphemes, many combinations of dissimilar vowels are allowed, for example in \textit{kaɨn} ‘year’, \textit{mɛʉr} ‘drongo’, and \textit{kɔɨn} ‘scent’. 

Icétôd vowel assimilation can be clearly seen throughout the lexicon, as when the transitive infinitive suffix \{-és\} and the intransitive infinitive suffix \{-òn\} are affixed to verb roots. If the verb root ends in /a/ or /e/, the vowel of the suffix fully assimilates it. \tabref{tab:2}.6 below offers a few examples of vowel assimilation in verbal infinitives:


\begin{table}
\caption{6: Vowel assimilation in verbal infinitives}
\label{tab:2}


\begin{tabularx}{\textwidth}{XXXX}
\lsptoprule

Transitive &  &  & \\
fá-és & → & féés & ‘to boil’\\
ɨsá-és & → & ɨsɛɛs & ‘to miss’\\
ɨtɨŋá-és & → & ɨtɨŋɛɛs & ‘to force’\\
tamá-és & → & tamɛɛs & ‘to extol’\\
wa-és & → & weés & ‘to harvest’\\
Intransitive &  &  & \\
ƙà-òn & → & ƙòòn & ‘to go’\\
ŋká-ón & → & ŋkóón & ‘to stand up’\\
tsá-ón & → & tsóón & ‘to be dry’\\
tsè-òn & → & tsòòn & ‘to dawn’\\
zè-òn & → & zòòn & ‘to be big’\\
\lspbottomrule
\end{tabularx}
\end{table}

Another environment illustrating Icétôd vowel assimilation is the case declension of nouns. Since all Icétôd nouns end in a vowel, and since seven of the eight case suffixes consist of or contain a vowel, case suffixation creates a fertile ground for vowel assimilation. For example, as \tabref{tab:2}.7 shows below, in the declension of the noun root \textit{ŋókí-} ‘dog’, the /o/ in the ablative case suffix \{-o\} and the copulative case suffix \{-ko\} partially assimilate the final /i/ of \textit{ŋókí-} to /u/:


\begin{table}
\caption{7: Vowel assimilation in the declension of \textit{ŋókí-} ‘dog’}
\label{tab:2}


\begin{tabularx}{\textwidth}{XXX}
\lsptoprule

Case & \textsc{nf} & \textsc{ff}\\
Nominative & ŋók-á & ŋók-ᵃ\\
Accusative & ŋókí-à & ŋókí-kᵃ\\
Dative & ŋókí-è & ŋókí-kᵉ\\
Genitive & ŋókí-è & ŋókí-\textsuperscript{Ø}\\
Ablative & ŋókú-ò & ŋókú-\textsuperscript{Ø}\\
Instrumental & ŋók-ó & ŋók-ᵒ\\
Copulative & ŋókú-ò & ŋókú-kᵒ\\
Oblique & ŋókí & ŋókⁱ\\
\lspbottomrule
\end{tabularx}
\end{table}

Further vowel assimilation effects are seen in the case declension of a noun like \textit{ŋʉrá-} ‘cane rat’. As shown in \tabref{tab:2}.8 below, the final /a/ of \textit{ŋʉrá-} is susceptible to being assimilated by the dative, genitive, ablative, and copulative case suffixes in their non-final forms:


\begin{table}
\caption{8: Vowel assimilation in the declension of \textit{ŋʉrá-} ‘cane rat’}
\label{tab:2}


\begin{tabularx}{\textwidth}{XXX}
\lsptoprule

Case & \textsc{nf} & \textsc{ff}\\
Nominative & ŋʉr-a & ŋʉr-\textsuperscript{Ø}\\
Accusative & ŋʉrá-á & ŋʉrá-kᵃ\\
Dative & ŋʉrɛ-ɛ & ŋʉrá-kᵋ\\
Genitive & ŋʉrɛ-ɛ & ŋʉrá-ᵋ\\
Ablative & ŋʉrɔ-ɔ & ŋʉrá-ᵓ\\
Instrumental & ŋʉr-ɔ & ŋʉr-ᵓ\\
Copulative & ŋʉrɔ-ɔ & ŋʉrá-kᵓ\\
Oblique & ŋʉra & ŋʉr\\
\lspbottomrule
\end{tabularx}
\end{table}

Icétôd vowel assimilation may be partial, as when the form \textit{ŋókí-kᵒ} ‘It is a dog’ is rendered as \textit{ŋókú-kᵒ}. There, the /i/ at the end of \textit{ŋókí-} ‘dog’ only moves back in the mouth to become /u/; it does not fully assimilate to become identical to the /o/ in the suffix. But vowel assimilation can also be total, as when \textit{ŋʉrá-ɛ} ‘of the cane rat’ becomes \textit{ŋʉrɛ-ɛ}. In that case, the /a/ at the end of \textit{ŋʉrá}{}- becomes identical to the vowel in the suffix. Icétôd vowel harmony can also be regressive as in both of these examples, where a vowel exerts pressure on a preceding noun. But it can also be progressive, as in the example of \textit{torí-úƙot-} becoming \textit{torí-íƙot}{}-, where the /i/ acts ahead on the /u/.
 
\subsubsection{Vowel desyllabification}

When the back-of-the-mouth vowels /ɔ/, /o/, /ʉ/ or /u/ wind up next to another vowel across a morpheme boundary, the back vowel may lose its status as the nucleus of a syllable and become the semi-vowel /w/ instead. When this vowel \textsc{desyllabification} occurs, the syllabic ‘weight’ of the vowel gets transferred to the following vowel in a process called \textsc{compensatory lengthening}. This is evident, for example, in the transitive infinitives of verbs ending in a back vowel. \tabref{tab:2}.9 depicts how the back vowel at the end of the verb root changes to /w/ and then lengthens the vowel in the suffix \{-és\}.


\begin{table}
\caption{9: Vowel desyllabification in verbs}
\label{tab:2}


\begin{tabularx}{\textwidth}{XXXX}
\lsptoprule

tʉtsʉ-ɛs & → & tʉtswɛɛs & ‘to wring’\\
rɔ-ɛ & → & rwɛɛs & ‘to string’\\
ho-és & → & hweés & ‘to cut’\\
ó-és & → & wéés & ‘to call’\\
ru-és & → & rweés & ‘to uproot’\\
\lspbottomrule
\end{tabularx}
\end{table}

Vowel desyllabification also takes place in the case declensions of nouns. Any noun root that ends in a back vowel can have that vowel desyllabified to /w/, with the result that the case suffix is lengthened. As \tabref{tab:2}.10 demonstrates, this happened with a noun like \textit{dakú-} ‘plant, tree’ that ends with the back vowel /u/. In five of the eight cases—accusative, dative, genitive, ablative, copulative—the final /u/ of \textit{dakú-} changes to /w/ and then lengthens the case suffix. Note that in the nominative case, the /u/ of \textit{dakú-} is desyllabified but does not lengthen the nominative suffix \{-a\}. This is a peculiarity of the nominative case only and is seen in many other noun declensions.


\begin{table}
\caption{10: Vowel desyllabification in nouns}
\label{tab:2}


\begin{tabularx}{\textwidth}{XXXX}
\lsptoprule

Case & \multicolumn{3}{X}{ Non-final}\\
Nominative & dakw-a &  & \\
Accusative & dakú-á & → & dakw-áá\\
Dative & dakú-é & → & dakw-éé\\
Genitive & dakú-é & → & dakw-éé\\
Ablative & dakú-ó & → & dakw-óó\\
Instrumental & dak-o &  & \\
Copulative & dakú-ó & → & dakw-óó\\
Oblique & daku &  & \\
\lspbottomrule
\end{tabularx}
\end{table}

\subsection{Vowel harmony}


Icétôd vowels participate in a system of \textsc{vowel harmony}. This means that the language’s sound system seeks vocalic ‘harmony’ by ensuring that all vowels in a single word belong to the same vowel class. The vowel classes involved are the following: 1) the [+ATR] or ‘heavy’ vowels /i, e, o, u/ that are made with a larger cavity in the throat, giving them a ‘heavier’, more resonant sound, and 2) the [-ATR] or ‘light’ vowels /ɨ, ɛ, ɔ, ʉ/ that are made with a smaller cavity in the throat, giving them a ‘lighter’, less resonant sound. Where the ninth vowel /a/ fits in with these two classes is a theoretical question that has not been conclusively resolved. However, it is clear is that in Icétôd, /a/ sometimes behaves as a [+ATR] vowel and other times as a [-ATR] vowel. And it certainly is found together with vowels from both classes within a single word. The Icétôd vowel classes anchored by the low vowel /a/ are depicted below in \tabref{tab:2}.11:


\begin{table}
\caption{11: Icétôd vowel classes}
\label{tab:2}

\begin{tabularx}{\textwidth}{XXXXX}
\lsptoprule

\multicolumn{2}{X}{ \textsc{[+ATR]}} &  & \multicolumn{2}{X}{ \textsc{[-ATR]}}\\
 i & u &  & ɨ & ʉ\\
 e & o &  & ɛ & ɔ\\
&  & a &  & \\
\lspbottomrule
\end{tabularx}

\end{table}

Generally speaking, because of vowel harmony, all the vowels in a single word will belong to one of the vowel classes shown in \tabref{tab:2}.11. This is clearly evident in the lexicon where verbs consisting of multiple syllables and morphemes contain either [+ATR] or [-ATR] vowels, but not both. \tabref{tab:2}.12 shows an opposing set of such verbs. Notice how all the vowels in each word belong to one vowel class:


\begin{table}
\caption{12: Vowel harmony in the lexicon}
\label{tab:2}


\begin{tabularx}{\textwidth}{XX}
\lsptoprule

[+ATR] & \\
béberés & ‘to pull’\\
béberetés & ‘to pull this way’\\
béberésúƙotᵃ & ‘to pull that way’\\{}
[-ATR] & \\
bɛɗɛs & ‘to want’\\
bɛɗɛtɛs & ‘to look for’\\
bɛɗɛsʉƙɔtᵃ & ‘to go look for’\\
\lspbottomrule
\end{tabularx}
\end{table}
In some situations though, /a/ blocks vowel harmony from spreading to all the morphemes in a word. For example, when the stative suffix \{-án-\} falls between a verb with [-ATR] vowels and the intransitive suffix \{-òn-\}, the /a/ in \{-án-\} prevents the spread of harmony to the whole word. \tabref{tab:2}.13 gives a few examples of the harmony-blocking behavior of /a/. Notice how [-ATR] vowels are found to the left of \{-án-\} (in bold), while the [+ATR] /o/ in \{-òn-\} comes after it:


\begin{table}
\caption{Vowel harmony blocking of /a/}
\label{tab:13}


\begin{tabularx}{\textwidth}{XX}
\lsptoprule

akwɛtɛkwɛt\textbf{án}ón & ‘to writhe around’\\
ɓɛlɛɓɛl\textbf{án}ón & ‘to be cracked’\\
gɔlɔgɔl\textbf{án}ón & ‘to be crooked’\\
ɨlɔɗɨŋ\textbf{án}ón & ‘to be discriminatory’\\
ŋʉzʉm\textbf{án}ón & ‘to bicker’\\
\lspbottomrule
\end{tabularx}
\end{table}
Icétôd has three suffixes which are said to be \textsc{dominant} in that they always spread their [+ATR] value as far as they can within a word. These include the pluractional suffix \{-í-\}, the middle suffix \{-ím-\}, and the plurative suffix \{-íkó-\}, all of which contain the vowel /i/. Unless an /a/ blocks the way, these three suffixes will cause all the vowels in the word they are found in to harmonize to [+ATR]. This dominant behavior is illustrated below in \tabref{tab:2}.14. Notice how the [-ATR] vowels in the first column all become [+ATR] in the third column as a result of the dominance of the suffixes (in bold typeface):


\begin{table}
\caption{14: Icétôd dominant suffixes}
\label{tab:2}


\begin{tabularx}{\textwidth}{XXXXX}
\lsptoprule

abʉtɛs & ‘to sip’ & → & abut\textbf{i}és & ‘to sip continuously\\
kɔnɔn & ‘to be one’ & → & kón\textbf{í}ón & ‘to be one-by-one’\\
&  & → &  & \\
ɨlɔɛs & ‘to defeat’ & → & ilo\textbf{im}étòn & ‘to be defeated’\\
kɔkɛs & ‘to close’ & → & kok\textbf{ím}étòn & ‘to close (alone)’\\
&  & → &  & \\
ɔrɔr & ‘stream’ & → & orór\textbf{íkw}ᵃ & ‘streams’\\
wɛl & ‘opening’ & → & wél\textbf{íkw}ᵃ & ‘openings’\\
\lspbottomrule
\end{tabularx}
\end{table}
Two other instances of vowel harmony deserve mention. First, when two nouns are joined together to form a compound word (§4.3), vowel harmony does not occur between them. For example, the noun roots \textit{rébè-} ‘millet’ and \textit{mɛsɛ-} ‘beer’ can be joined into the compound \textit{rébèmɛsɛ-} ‘millet beer’, in which, notice, the vowels belong to two different [ATR] vowel classes. An exception to this rule is when the second noun in the compound begins with the vowel /i/, in which case /i/ harmonizes the last vowel of the first noun, as when \textit{ɲɔkɔkɔrɔ-ímà-} ‘chick’ becomes \textit{ɲɔkɔkɔró-ímà-} (where the first noun’s /ɔ/ is harmonized to /o/). Second, many of Icétôd’s clitics take on the [ATR] value of their host word, for example when the anaphoric\textsc{} pronoun \textit{déé} becomes \textit{dɛɛ} in the phrase \textit{mɔƙɔrɔɛ=dɛɛ} ‘in that rock pool’. Again, the exception is when the clitic contains /i/, in which case it becomes dominant, harmonizing its host, as when \textit{bárɨtɨnʉɔ=díí} ‘from those corrals’ becomes \textit{bárɨtɨnúo=díí} (where the vowels /ʉɔ/ become /úo/.
 
\subsection{Tone}
\subsubsection{Tone inventory}

Icétôd is a tonal language. In terms of acoustics, this means that every vowel is identified not only by where it is formed in the vocal chamber but also by the \textsc{pitch} with which it is uttered. This further entails the every syllable, morpheme, word, and phrase exhibits a specific and indispensable \textsc{tone} pattern. At a phonological (or psychological) level, Icétôd has just two tones: \textsc{high} (H) and \textsc{low (L)}. All other tones that one hears can be traced back to these two. However, for more practical applications like orthography and language learning, four sub-tones must be recognized. These include: \textsc{high}, \textsc{high-falling}, \textsc{mid}, and \textsc{low}. High tone is pronounced with a level, relatively high pitch. High-falling tone falls quickly from relatively high to relatively low pitch, often in the presence of a depressor consonant (see §2.6.4 below). Mid tone is a level, relatively medium-height pitch, while low tone is either relatively low and flat or tapering off before a pause. \tabref{tab:2}.15 presents the Icétôd tones with their names in the first column, pitch profiles in the second, and the orthographic diacritics for writing them in the third (the same diacritics employed throughout the foregoing dictionary sections):


\begin{table}
\caption{15: Icétôd tones}
\label{tab:2}


\begin{tabularx}{\textwidth}{XXX}
\lsptoprule

Tone & Pitch & Symbol\\
\textsc{high} & [4] & Á á\\
\textsc{high-falling} & [j] & \^{A} â\\
\textsc{mid} & [3] & A a\\
\textsc{low} & [2]/[a] & \`{A} à\\
\lspbottomrule
\end{tabularx}
\end{table}


\subsubsection{Lexical tone}

As mentioned above, every word in the Icétôd lexicon has a tone pattern or ‘melody’. That is, Icétôd words are not identified solely on the basis of consonants and vowels (as in non-tonal languages) but also on their tone pattern, which must be learned. Since every vowel and therefore every syllable bears a tone, the combination of many syllables in words produces a large variety of tone patterns. And since the tone pattern of a word is totally unpredictable, language learners must resort to memorizing the pattern with the word. \tabref{tab:2}.16 gives a sample of the lexical tone patterns on some short words in Icétôd:


\begin{table}
\caption{16: Icétôd lexical tone patterns}
\label{tab:2}


\begin{tabularx}{\textwidth}{XXX}
\lsptoprule

Nouns &  & \\
HH & ámá- & ‘person’\\
HL & ɛbà- & ‘horn’\\
LH & cekí- & ‘woman’\\
LL & ɲèrà- & ‘girls’\\
Verbs &  & \\
H & ŋáɲ- & ‘open’\\
H(L) & éd\`{} - & ‘carry on back’\\
L & àts- & ‘come’\\
\lspbottomrule
\end{tabularx}
\end{table}

\subsubsection{Grammatical tone}

Icétôd does not have grammatical tone in the sense that tone alone can carry out a grammatical function. But tone often accompanies other grammatical signals, thereby reinforcing them. So in that regard, it could be said that Icétôd has ‘semi-grammatical’ tone. For example, when the suffix \{-íkó-\} is used to pluralize a single noun, the tone of the single noun usually changes, as when \textit{kɔl} ‘ram’ becomes \textit{kólíkwᵃ}. Similarly, when the venitive suffix \{-ét-\} is added to a verb \textsc{stem}, it often changes the overall tone pattern, as when \textit{bɛɗɛs} ‘to want’ becomes \textit{bɛɗɛtɛs} ‘to look for’, whereby the tone of the root \textit{bɛɗ-} goes from \textsc{high} to \textsc{mid}. Indeed, many of the nominal and verbal suffixes of the language are associated with significant tone changes to the stem. So even if one learns the tonal melodies of nouns and verbs on their own, these melodies may change in particular grammatical contexts. This tone changeability is one of the system’s more difficult aspects.

The Icétôd tone system is challenging for foreigners and is not yet fully understood from an analytical point of view. Still, the good news is that with lots of practice, language learners can reasonably expect to develop a certain degree of communicative competency. For the most complete description of the tone system to date, the reader is invited to consult §3.2 in \textit{A grammar of Icé-tód} \citep{Schrock2014}. That section expands on what has been presented here and includes more detailed discussions of other features of the Icétôd tone system.
 
\subsubsection{Depressor consonants}

In Icétôd, the class of voiced consonants /b, d, dz, g, j, z, ʒ/ plus /h/ act as \textsc{depressor consonants}. Depressors are so-called because they ‘depress’ or pull down the pitch of neighboring vowels. In doing so, they act almost as if they had a very low tone of their own. The effect of Icétôd depressors is so strong that, over time, it led to the creation of a whole new set of lexical tone patterns. For instance, all Icétôd verbs with a HL pattern in their roots have a depressor as the first consonant after the initial high tone: \textit{dɛgɛm-} ‘crouch’, \textit{gʉgʉr-} ‘hunched’, \textit{íbòt-} ‘jump’, \textit{kídzìm-} ‘descend’, and \textit{tsʼágwà-} ‘be raw’. This is because, in anticipation of the extra-low pitch of the depressor, the language compensated by putting a high tone before it where there used to be none. As another example, all nouns with the root tone pattern HL have a depressor as their only consonant: \textit{dóbà-} ‘mud’, \textit{ɛbà-} ‘horn’, \textit{édì-} ‘name’, \textit{nébù-} ‘body’, \textit{wídzò-} ‘evening’, etc. And when these types of nouns lose their final vowel due to vowel devoicing, that is when the \textsc{high-falling} contour tone comes into play, as in \textit{dɔbᵃ} ‘mud’, \textit{ɛbᵃ} ‘horn’, \textit{êdᵃ} ‘name’, \textit{wîdzᵃ} ‘evening’, etc.

Whenever a depressor consonant falls immediately between two high tones, the second high tone is lowered or ‘downstepped’ to a mid tone level (indicated by the symbol ꜜ followed by a high tone). From the point of view of pronunciation, this is because the speaker’s pitch cannot make it from the pitch depression all the way back up to a high pitch. This happens, for example, with the anaphoric pronouns \textit{déé} and \textit{díí}, as in \textit{ámá=}\textit{ꜜ}\textit{déé} ‘that person’ \textit{ínwá=ꜜdíí} ‘those animals’.
 
\section{Morphology—the making of words}
 
\subsection{Overview}


\textsc{morphology} is the system by which a language grammar makes words. While the preceding chapter introduced meaningful sound units (phonemes), the present chapter describes larger meaningful units called \textsc{morphemes}. Icétôd exhibits three types of morpheme: word, affix, and clitic. A \textsc{word} is defined as a free morpheme that can meaningfully stand alone. An \textsc{affix} is a bound morpheme that must attach to a word to maintain its integrity. Affixes are indicated in this grammar by a hyphen before (and sometime after) them, as in \{-án-\}, the stative adjectival suffix. A \textsc{clitic} is a hybrid: in some constructions it acts like a word standing alone, while in other constructions, it attaches to a word like an affix. Clitics are sometimes marked in this grammar by an equals sign, as in \{=kì\} ‘those’. 

Traditionally, languages are described as having \textsc{word classes}, that is, categories of morphemes that have certain characteristics. These classes include the familiar major ones like ‘nouns’ and ‘verbs’ but often several others as well. For the purposes of this grammar sketch, free-standing words and clitics are considered ‘words’, while affixes are not. In Icétôd, thirteen word classes are recognized and include the following: nouns, pronouns, demonstratives, quantifiers, numerals, prepositions, verbs, adverbs, ideophones, interjections, nursery words, complementizers, and connectives (or conjunctions). Each of these word classes is briefly introduced in the following subsections, while a full list of Icétôd affixes can be found later in Appendix A.
 
\subsection{Nouns}


\textsc{Nouns} and verbs make up the language’s only two open word classes, meaning that they may have new members continually added to them. Nouns make up roughly 47\% of the total Icétôd lexicon. Noun roots can be short, like \textit{eí-} ‘chyme’, or long like \textit{ɲákaɓɔɓwáátá-} ‘finger ring’, but they must all have at least two syllables. This is because some case suffixes delete the last vowel of the noun root when they affix to it. Noun roots are represented throughout this book with hyphenated forms, indicating that in actual Icétôd speech, any noun must have at least a case suffix. In addition to case, nouns may take singulative or plurative suffixes and may be joined with other nouns to make compound nouns. §4 is devoted to expounding on Icétôd nouns.
 
\subsection{Pronouns}


\textsc{Pronouns} form a closed word class, admitting no new members. They ‘stand in’ for nouns whose specific names need not always be mentioned or repeated. Pronouns make up less than 1\% of the Icétôd lexicon and yet have great grammatical importance. Most Icétôd pronouns are \textsc{free}, capable of standing on their own, while others are \textsc{bound} to verbs. They may be \textsc{personal}, capable of specifying grammatical person, or \textsc{impersonal}. Other categories of pronoun include: indefinite, interrogative, demonstrative, relative, and reflexive. §5 is devoted to describing the pronouns of Icétôd.
 
\subsection{Demonstratives}


\textsc{Demonstratives} form another closed word class, admitting no new members. They ‘demonstrate’ nouns by ‘pointing them out’, referring to them spatially, temporally, or discursively. They too make up less than 1\% of the lexicon. Many Icétôd demonstratives have been analyzed as clitics: They seem sometimes to act like separate words, and yet in terms of vowel harmony, they act like suffixes. As clitics, they may be written connected to words in linguistic writing (with =), whereas in non-linguistic writing, they are written separately. For example, the phrase ‘these trees’ would be written as \textit{dakwítína=ni} in linguistic publications and as \textit{dakwítína ni} elsewhere. Icétôd has four kinds of demonstrative: spatial, temporal, anaphoric, and locative adverbial—all of which are covered in more detail in §6.
 
\subsection{Quantifiers}


As their name implies, \textsc{quantifiers} ‘quantify’ the nouns that precede them. That is, they are separate words that follow nouns and convey the general quantity of the noun in terms of allness, bothness, fewness, or manyness. Specific, numeric quantity is expressed by the numerals which are the topic of the next subsection. Icétôd quantifiers sometimes act more like numerals by directly following the noun they modify without an intervening relative pronoun, as in \textit{wika ƙwaɗᵉ} ‘few children’. But other times they act more like adjectival verbs by taking a relative pronoun between them and the noun they modify, for example, \textit{wika ni ƙwaɗᵉ} ‘children that (are) few’. In the former function as numerals, they have a distinct, perhaps more ancient root, as in \textit{ƙwàɗè}, whereas in their function as adjectival verbs, they have a truncated root in a verbal infinitive, in this case \textit{ƙwàɗ-òn} ‘to be few’. The eight known Icétôd quantifiers are given below in \tabref{tab:3}.1:


\begin{table}
\caption{1: Icétôd quantifiers}
\label{tab:3}


\begin{tabularx}{\textwidth}{XXX}
\lsptoprule

Non-final & Final & \\
ɗàŋɨɗàŋɨ & ɗàŋɨɗàŋ & ‘all, entire, whole’\\
mùɲù & mùɲ & ‘all, entire, whole’\\
mùɲùmùɲù & mùɲùmùɲ & ‘all, entire, whole’\\
tsɨɗɨ & tsɨɗᶤ & ‘all, entire, whole’\\
tsɨɗɨtsɨɗɨ & tsɨɗɨtsɨɗᶤ & ‘all, entire, whole’\\
ɡáí & ɡáí & ‘both’\\
ƙwàɗè & ƙwàɗᵉ & ‘few’\\
kòmà & kòm & ‘many’\\
\lspbottomrule
\end{tabularx}
\end{table}



\subsection{Numerals}


\textsc{Numerals} convey the specific number of the noun they modify. Icétôd has a quinary or ‘base-5’ counting system, meaning that it has words for the numbers 1-5 and then builds numbers 6-9 by adding the appropriate number to 5, as in \textit{tude ńda kiɗi tsʼagús} ‘five and those four’, which is 9. The number 10 is not a numeral, but the noun \textit{toomíní-}. Icétôd numerals directly follow the noun they modify, without an intervening relative pronoun. Just as the quantifiers \textit{ƙwàɗè} ‘few’ and \textit{kòmà} ‘many’ can function as verbs, the numerals 1-5 can also function as verbs. \tabref{tab:3}.2 presents Icétôd numerals 1-9:


\begin{table}
\caption{2: Icétôd numerals}
\label{tab:3}


\begin{tabularx}{\textwidth}{XXXX}
\lsptoprule

\# & Non-final & Final & \\
1 & kɔnà & kɔn & ‘one’\\
2 & lèɓètsè & lèɓètsᵉ & ‘two’\\
3 & àɗè & àɗᵉ & ‘three’\\
4 & tsʼagúsé & tsʼagús & ‘four’\\
5 & tùdè & tùdᵉ & ‘five’\\
6 & tude ńdà kɛɗɨ kɔn & ...ńdà kɛɗɨ kɔn & ‘five and one’\\
7 & tude ńda kiɗi léɓètsè & ...ńda kiɗi léɓètsᵉ & ‘five and two’\\
8 & tude ńdà kìɗì àɗè & ...ńdà kìɗì àɗᵉ & ‘five and three’\\
9 & tude ńda kiɗi tsʼagúsé & ...ńda kiɗi tsʼagús & ‘five and four’\\
\lspbottomrule
\end{tabularx}
\end{table}
To form numbers 11-19, Icétôd builds off the noun \textit{toomíní-} ‘ten’ and then repeats the quinary system shown above in \tabref{tab:3}.2. For example, the number 17 is expressed as \textit{toomín ńda kiɗi túde ńda kiɗi léɓètsᵉ} ‘ten and those five and those two’. Then, after 19, the numbers 20, 30, 40, etc. are based on the compound \textit{toomín-ékù-} ‘ten-eye’, as in \textit{toomínékwa léɓètsᵉ} ‘ten-eye two’, which is 20. The numbers for 100 (\textit{ŋamɨáɨ-}) and 1,000 (\textit{álìfù-}) have both been borrowed from Swahili.
 
\subsection{Prepositions}


\textsc{Prepositions} are usually small particles ‘pre-posed’, that is, put in front of a noun to indicate what its relationship is to another noun or to the wider sentence in which it occurs. Many of the functions that prepositions fulfill in other languages are handled by cases in Icétôd (see §7). However, Icétôd still has a very small, closed group of prepositions that somehow have survived the hegemony of case. Still, they interact with case as each preposition selects the case that its noun head (or host) must take. \tabref{tab:3}.3 presents all the known Icétôd prepositions with their meanings and the cases they require on nouns:


\begin{table}
\caption{3: Icétôd prepositions}
\label{tab:3}


\begin{tabularx}{\textwidth}{XXX}
\lsptoprule

Preposition & Meaning & Case required\\
nàpèì & ‘from, since’ & \textsc{ablative}\\
ɗɨtá & ‘as, like’ & \textsc{genitive}\\
nɛɛ & ‘from, through’ & \textsc{genitive}\\
akánɨ & ‘until, up to’ & \textsc{oblique}\\
àkɨlɔ & ‘instead of’ & \textsc{oblique}\\
gònè & ‘until, up to’ & \textsc{oblique}\\
ikóteré & ‘because of’ & \textsc{oblique}\\
ńdà & ‘and, with’ & \textsc{oblique}\\
pákà & ‘until, up to’ & \textsc{oblique}\\
tònì & ‘even’ & \textsc{oblique}\\
\lspbottomrule
\end{tabularx}
\end{table}
The following example sentences offer an opportunity to see the prepositions from \tabref{tab:3}.3 above in natural language contexts:


 
\ea\label{ex:}
\gll {napei}\textit{ Kaaɓɔŋʉɔ}     \textit{páka}   awᵃ  \\
    \\
from   Kaabong:\textsc{abl}   up.to   home:\textsc{obl} 
\glt ‘from Kaabong up to home’ 
\z

\ea\label{ex:}
\gll {Gógese   tufúlá}       \textit{ɗɨtá}   rié. \\
    \\
peg:\textsc{pass}   field.rat:\textsc{nom}   like   goat:\textsc{gen}
\glt ‘And the field rat is pegged up like a goat.’ 
\z





\ea\label{ex:}
\gll {Atsía}     \textit{nɛɛ}   Tímuaƙwɛɛ     nɛ. \\
    \\
come:\textsc{1sg}   from   Timu:inside:\textsc{gen}   that
\glt ‘I’m coming from within Timu there.’ 
\z




\ea\label{ex:}
\gll {Hoɗuƙotᵉ,}   \textit{akɨlɔ}     cɛɛsʉƙɔtᶤ. \\
    \\
set.free:\textsc{imp}   instead.of   killing:\textsc{obl}
\glt ‘Set (him) free instead of killing (him).’ 
\z




\ea\label{ex:}
\gll {Duƙotuo}   \textit{gone}   hoo     déé. \\
    \\
take:\textsc{seq}   up.to   hut:\textsc{obl}   that
\glt ‘And she took (it) up to that hut.’ 
\z




\ea\label{ex:}
\gll {Ƙáátaa   Tábayɛɛ}   \textit{ikóteré}   ɲɛƙᵋ. \\
    \\
go:\textsc{3pl:prf}   West:\textsc{dat}   because.of   hunger:\textsc{obl}
\glt ‘They’ve gone west because of hunger.’ 
\z




\ea\label{ex:}
\gll {tɛwɛɛsa     kɔlɨlɨɛ}       \textit{ńda}   lomuƙeⁱ \\
    \\
sow:\textsc{inf:nom}   cucumber:\textsc{gen}   and   squash:\textsc{obl}
\glt ‘the sowing of cucumber and squash’ 
\z




\ea\label{ex:}
\gll {toni}   Pakóíce     ʝɨk,   góƙánɨkɛdᵋ \\
    \\
even  Turkanas:\textsc{obl}   also   seated:\textsc{ips:sim:dp}
\glt ‘even the Turkanas as well, (were) staying there’ 
\z






\subsection{Verbs}


\textsc{Verbs} comprise the second of Icétôd’s two large open word classes. Like nouns, Icétôd verbs make up approximately 48\% of the lexicon. Verb roots can be short like \textit{ó-} ‘call’, long like \textit{gwɛrɛʝɛʝ-} ‘be coarse’, or reduplicated like \textit{diridír-} ‘be sugary’ and \textit{ɨpɨrɨpɨr-} ‘drill’. Verb roots are represented throughout this book with hyphenated forms, indicating that in actual Icétôd speech, any verb must have at least one suffix. That minimal suffix may be a subject-agreement suffix or a tense-aspect-mood (TAM) suffix like an imperative or optative. Icétôd verb stems can stand alone as an independent, self-contained clause and can have many suffixes strung together, as in \textit{soƙórítiísínàkᵃ} ‘we all have clawed’ and \textit{zeikááƙotinîdᵉ} ‘and they all grew large there’. Among the many suffixes that can derive nouns from verbs or inflect verbs for different meanings, there are: deverbatives, subject-agreement markers, directionals, the dummy pronominal, modals, aspectuals, voice and valency changers, and adjectivals. All these verb-related topics (and others) are treated more fully later on in §8.




\subsection{Adverbs}


\textsc{Adverbs} make up a catch-all category of words that modify verbs or whole clauses. The roughly sixty Icétôd adverbs make up less than 1\% of the total lexicon. They include ‘manner’ adverbs like \textit{hɨɨʝɔ} ‘slowly’ and \textit{zùkù} ‘very’, epistemic adverbs like \textit{tsábò} ‘apparently’ and \textit{tsamʉ} ‘of course’, and general adverbs like \textit{ɛɗá} ‘only’ and \textit{naɓó} ‘again’. Other important categories of adverbs are the tense-marking adverbs, certainty and contingency markers, and the conditional-hypothetical adverbs. All these types of Icétôd adverb are described further in §9.




\subsection{Ideophones}


\textsc{Ideophones} form a word class that is characterized by highly expressive words that denote physical phenomena like color, motion, sound, shape, volume, etc. They are often ‘sound-symbolic’ or onomatopoeic. That means just the sound of them as they are pronounced evokes the physical perception they signify. For example, the ideophone \textit{bùlùƙᵘ} means ‘the sound something makes when dropping into water’, like ‘splashǃ’ or ‘kersplunkǃ’ in English. At present, one hundred forty Icétôd ideophones (1.6\% of total) have been recorded, but there are most certainly many more in the language. And they are probably continually created. \tabref{tab:3}.4 offers a sample of the variety of Icétôd ideophones that are recognized:


\begin{table}
\caption{4: Icétôd ideophones}
\label{tab:3}


\begin{tabularx}{\textwidth}{XX}
\lsptoprule

Animal sounds & \\
bèrrr & ‘baaaǃ’\\
buúù & ‘moooǃ’\\
ƙútú & ‘cluckǃ’\\
Other sounds & \\
ɓɛkɛ & ‘snapǃ’\\
gʉlʉʝʉ & ‘gulpǃ’\\
pùsù & ‘plopǃ’\\
Colors & \\
pàkì & ‘pure white’\\
tíkí & ‘pitch black’\\
tsònì & ‘blood red’\\
Attributes & \\
ɓa & ‘unliftably heavy’\\
dùù & ‘very deep’\\
tsɛkɛ & ‘completely full’\\
\lspbottomrule
\end{tabularx}
\end{table}



\subsection{Interjections}


Like adverbs, \textsc{interjections} form a bit of a catch-all word class. Interjections include any word expressing emotions or mental states of any kind, usually outside the grammar of the sentence. The roughly thirty Icétôd interjections that have been recorded make up less than 1\% of the total lexicon. Icétôd interjections may consist of a single word like \textit{aaii} ‘ouchǃ’ or \textit{wúlù} ‘yikesǃ’ or a short phrase like \textit{wika ni} ‘these kids (I tell you)ǃ’ or \textit{tɨɔ ʝɔɔ} ‘there, there (it’s okay)ǃ’. Several other interjections are provided below in \tabref{tab:3}.5:


\begin{table}
\caption{5: Icétôd interjections}
\label{tab:3}


\begin{tabularx}{\textwidth}{XX}
\lsptoprule

ee Ɲakuʝᵃ & ‘oh my Godǃ’\\
ee/éé & ‘yeah, yes’\\
hà & ‘whateverǃ’\\
maráŋ & ‘fine, okayǃ’\\
ɲɔto ni & ‘these guys (I tell you)ǃ’\\
ne & ‘here you goǃ’\\
ńtóo(n)dó & ‘nah, no’\\
otí & ‘whoaǃ’\\
wóí & ‘aahhǃ’\\
yóói & ‘uh-huh..sureǃ’\\
\lspbottomrule
\end{tabularx}
\end{table}



\subsection{Nursery words}


\textsc{Nursery} \textsc{words} make up a small class of one-word expressions—only ten recorded so far—that act as commands or encouragements to babies or toddlers to do something. The ten Icétôd nursery words on record are lain out below in \tabref{tab:3}.6 with English approximations:


\begin{table}
\caption{6: Icétôd nursery words}
\label{tab:3}


\begin{tabularx}{\textwidth}{XXX}
\lsptoprule

bubú & ‘nighty-night’ & for going to sleep\\
ɓá & ‘yummy’ & for eating\\
dɪ & ‘poo’ & for defecating\\
dʊʊdʊ & ‘sitty-sit’ & for sitting down\\
kó & ‘wa-wa’ & for drinking water\\
kɔkɔ & ‘no-no’ & for not touching\\
kukú & ‘up-up’ & for riding on mother’s back\\
kwàà & ‘pee’ & for urinating\\
mamá & ‘yum-yum’ & for eating\\
nʊʊnʊ & ‘yum-yum’ & for breastfeeding\\
\lspbottomrule
\end{tabularx}
\end{table}



\subsection{Complementizers}


\textsc{Complementizers} are words that introduce reported speech or thought. For example, in the English sentence ‘She said that she agrees’, the word \textit{that} is the complementizer that introduces that reported statement \textit{she agrees}. Icétôd has only two complementizers. One of them, \textit{tòìmènà-} ‘that’, is technically a noun and thus belongs in the noun word class. But because of its function, it is dealt with here. The word \textit{tòìmɛnà-}, a compound of the verb \textit{tód-} ‘speak’ and \textit{mɛná-} ‘words’, is used with a variety of speaking and thinking verbs. The second Icétôd complementizer, \textit{tàà}, is a probably a derivative of the verb \textit{kʉta} ‘s/he says’ that has been reduced over time. Even now it is usually used after the verb \textit{kʉt-} ‘say’. Example (9) below shows how \textit{tòìmɛnà-} is used in a sentence to introduce the clause \textit{mɨtɨda bɔnán} ‘you are an orphan’. And example (10) shows the complementizer \textit{tàà} introducing the clause \textit{iya ɲjíníkiʝa kɔɔkɛ} ‘our land is over there’:




\ea\label{ex:}
\gll {Hyeíá}   \textit{toimɛna}   mɨtɨda   bɔnán. \\
    \\
know:\textsc{1sg}   that:\textsc{nom}   be:\textsc{2sg}   orphan:\textsc{obl}
\glt ‘I know that you are an orphan.’ 
\z




\ea\label{ex:}
\gll {Kʉta   ɲcie}   \textit{taa}   iya     ɲjíníkiʝa    kɔɔkɛ. \\
    \\
say:\textsc{3sg}   I:\textsc{dat}   that   be:\textsc{3sg}  we:land:\textsc{nom} there
\glt ‘He says to me that our land is over there.’ 
\z






\subsection{Connectives}


\textsc{Connectives} (also known as ‘conjunctions’) are words whose function is to join together other words, phrases, or clauses. If they are \textsc{coordinating} connectives like \textit{ńdà} ‘and’, then they join grammatical units of equal status, like a word to a word, or an independent clause to another independent one. Whereas if they are \textsc{subordinating} connectives like \textit{na} ‘if’, they join grammatical units of unequal status, usually a dependent clause to an independent one. Even though their role is to link grammatical units, not all of them come between the units they link. Many come before both, often as the first word in the sentence. Icétôd has roughly eight coordinating connectives and thirty subordinating ones—making up less than 1\% of the lexicon. The coordinating connectives are presented in \tabref{tab:3}.7, while \tabref{tab:3}.8 offers a representative sampling of the subordinating connectives:


\begin{table}
\caption{7: Icétôd coordinating connectives}
\label{tab:3}


\begin{tabularx}{\textwidth}{XX}
\lsptoprule

kèɗè & ‘or’\\
kɨná & ‘and then, so then, then’\\
kòrì & ‘or’\\
kòtò & ‘and, but, so, then, therefore’\\
mɨsɨ...mɨsɨ... & ‘either...or...’\\
náàtì & ‘and then’\\
naɓó & ‘furthermore, moreover’\\
ńdà & ‘and’\\
\lspbottomrule
\end{tabularx}
\end{table}
The following natural-language examples illustrate three of the more commonly used coordinating connectives: \textit{kèɗè}, \textit{kòtò}, and \textit{ńdà}. In example (11), the connective \textit{kèɗè} ‘or’ joins two equal constituents, the nouns \textit{Tábayɔɔ} and \textit{Fetíékù}. In (12), the connective \textit{kòtò} ‘and, but, then,’ links two independent but semantically related clauses, and in (13), the connective \textit{ńdà} ‘and’ connects two equal passive clauses:




\ea\label{ex:}
\gll {Tábayɔɔ}   \textit{keɗe}   Fetíékù? \\
    \\
West:\textsc{abl}   or   East:\textsc{abl}
\glt ‘From the West or from the East?’ 
\z



\ea\label{ex:}
\gll {Ɨʉmʉƙɔtɨakôdᵉ....}  \\
    \\
marry.forcibly:\textsc{1sg:seq:dp}
\glt ‘And from there I took (her) away as my wife....’ 
\z

\ea\label{ex:}
\gll {Moo}     \textit{koto}   sáɓánɨ   ínóà? \\
    \\
not:\textsc{seq}   but   kill:\textsc{ips}   animal:\textsc{nom}
\glt ‘But was an animal not killed (as a nuptial offering)?’ 
\z



\ea\label{ex:}
\gll {Sáɓese   basaúr}   \textit{ńda}   kotsana   cue. \\
    \\
kill:\textsc{sps}  eland:\textsc{nom}   and   fetch:\textsc{ips} water:\textsc{nom}
\glt ‘Elands were killed, and water was fetched.’ 
\z

In contrast to the coordinating connectives shown in \tabref{tab:3}.7 and examples (11)-(13), \textit{sub}ordinating connectives join units of unequal status, usually a subordinate (dependent) clause to a main one. \tabref{tab:3}.8 provides a representative sample of the thirty Icétôd subordinating connectives, while examples (14)-(16) below illustrate the function of some of these connectives in a few natural-language environments.


\begin{table}
\caption{8: Icétôd subordinating connectives}
\label{tab:3}


\begin{tabularx}{\textwidth}{XX}
\lsptoprule

átà & ‘even (if)’\\
ɗɛmʉsʉ & ‘before, unless, until’\\
ikóteré & ‘because’\\
kánɨ & ‘in order that, so that’\\
mɨsɨ & ‘if, whether’\\
na= & ‘if, when’\\
náà & ‘when (earlier today)’\\
nàpèì & ‘since’\\
nɛɛ & ‘if, when’\\
nòò & ‘when (long ago)’\\
nótsò & ‘when (a while ago)’\\
pákà & ‘until’\\
sɨnà & ‘when (yester-)’\\
tònì & ‘even’\\
\lspbottomrule
\end{tabularx}
\end{table}
In (14) below, the subordinating connective \textit{ɗɛmʉsʉ} ‘before, unless, until’ introduces a dependent clause that connects semantically to the following one. The same grammatical structure is also evident in (15) and (16), where the connectives \textit{mɨsɨ} ‘if, whether’ and \textit{na} ‘if, when’ set off short dependent clauses that logically lead into main clauses:



\ea\label{ex:}
\gll {Ɗ}\textit{ɛmʉsʉ}   Pakóíce     deti   riékᵃ, \\
    \\
before   Turkanas:\textsc{obl}   bring   goats:\textsc{acc}
\glt ‘Before the Turkanas brought goats, 
\z



\ea\label{ex:}
\gll {isio     noo   ŋábìàn?} \\
    \\
what:\textsc{cop}   \textsc{pst3}   wear:\textsc{plur:ips}
\glt what was typically worn (as clothing)?’
\z  



\ea\label{ex:}
\gll {Mɨsɨ}   ɨtáána   basaúrékᵉ,   sáɓes. \\
    \\
if   reach:\textsc{ips}   eland:\textsc{dat}   kill:\textsc{sps}
\glt ‘If they reach the eland, it is killed.’ 
\z




\ea\label{ex:}
\gll {Na}   átsikᵉ,       zɛƙwɛtɔɔ   nayéé   na. \\
    \\
when   come:\textsc{3sg:sim}   sit:\textsc{3sg:seq}   here   this
\glt ‘When she came, she sat down here.’ 
\z




\section{Nouns}



\subsection{Overview}


Single Icétôd \textsc{nouns} in a speaker’s mental lexicon consist of a \textsc{root.} Roots are words that cannot be analyzed into smaller parts from the perspective of modern Icétôd. (Historical research may in many cases reveal how roots were put together over time, but that is the domain of etymology.) When plucked from the lexicon and put into actual Icétôd speech, every noun root must receive at least one suffix, which must be a \textsc{case} suffix. In addition to case suffixes, an Icétôd noun may take on a \textsc{number} suffix or may be joined with one or two other nouns to form a \textsc{compound}. Case suffixes are fully explained later in §7, while number suffixes and compounds are covered in the rest of this chapter.

Icétôd number suffixes include \textsc{pluratives} and \textsc{singulatives}. Many noun roots can be pluralized if they are inherently singular in number. A few others can be singularized because they are inherently plural. In addition to these standard number-markers, Icétôd also has special \textsc{possessive} number suffixes that combine the notions of number and possession into one suffix—singular or plural. And yet other nouns are \textsc{mass} \textsc{nouns}, naming entities in the world perceived as inherently plural unities (like dust or water). These take no suffixes but are treated grammatically as plurals. Finally, some nouns are \textsc{transnumeral}, meaning they can be construed as singular or plural and given the appropriate singular or plural modifiers, if needed. 

Compounding (discussed below in §4.3) is the primary way Icétôd acquires or makes new nouns—besides borrowing them from other languages. Icétôd compounds are made by putting two or three nouns together into a new compound word with special characteristics. The first noun describes or specifies the second noun to make an aggregate meaning that is often different than that of the two separate nouns. 

Icétôd nominal suffixes differ in how they affix to noun roots. With the exception of five case suffixes, all nominal suffixes first delete the final vowel of the noun to which they attach. This is known as \textsc{subtractive} morphology. The case suffixes that preserve the final vowel are the accusative, dative, genitive, ablative, and oblique. For more on how case suffixes attach to nouns, refer ahead to §7.




\subsection{Number}
\subsubsection{Pluratives (\textsc{plur})}

Icétôd has four ways to show that a noun is plural: three \textsc{plurative} suffixes and suppletive plurals. The three plurative suffixes are: 1) \{-íkó-\}, 2) \{-ítíní-\}, and 3) \{-ìkà-\}. The first plurative suffix, \{-íkó-\}, is dominant in terms of vowel harmony, meaning it changes the vowels of a [-ATR] noun to [+ATR] unless /a/ intervenes and blocks it. For example, in some instances, the vowel /a/ spontaneously appears between the singular root and the suffix \{-íkó-\}. (This /a/ is a relic of an ancient singulative suffix *\textit{{}-at-} that is no longer in use in Icétôd.)

The use of \{-íkó-\} is strictly limited to a relatively small number of nouns (roughly 100); it is not applied to newly borrowed nouns. \tabref{tab:4}.1 presents several examples of nouns pluralized with this suffix. Note how the suffix harmonizes the vowels of the singular root except where the vowel /a/ blocks the leftward spread of harmony. Notice also that in some cases the suffix alters the tone of the singular root:


\begin{table}
\caption{1: The plurative suffix \{-íkó-\}}
\label{tab:4}


\begin{tabularx}{\textwidth}{XXXX}
\lsptoprule

Singular &  & Plural & \\
abérí- & → & áberaikó- & ‘active termite colonies’\\
baratsó- & → & barátsíkó- & ‘mornings’\\
cúrúkù- & → & cúrúkaikó- & ‘bulls’\\
kɔrɔbɛ- & → & kɔrɔbaikó- & ‘calves’\\
ƙwɛsɛɛ- & → & ƙwéséikó- & ‘broken gourds’\\
mɔƙɔrɔ- & → & moƙóríkó- & ‘rock wells’\\
taɓá- & → & taɓíkó- & ‘boulders’\\
\lspbottomrule
\end{tabularx}
\end{table}
The second plurative, \{-ítíní-\}, is used to pluralize nouns that have only two syllables in their lexical root. \tabref{tab:4}.2 provides a sample of disyllabic nouns pluralized with \{-ítíní-\}. Notice that if the singular noun has [-ATR] vowels, then the plurative suffix harmonizes to \{-ɨtɨnɨ-\}. Unlike the suffix \{-íkó-\}, \{-ítíní-\} never alters the tone of the root, though its own tone may conform to the tone of the root:


\begin{table}
\caption{2: The plurative suffix \{-ítíní-\}}
\label{tab:4}


\begin{tabularx}{\textwidth}{XXXX}
\lsptoprule

Singular &  & Plural & \\
aká- & → & akɨtɨnɨ- & ‘mouths’\\
bòsì- & → & bositíní- & ‘ears’\\
ɔʝá- & → & ɔʝɨtɨnɨ- & ‘sores’\\
ɗòlì & → & ɗólítíní- & ‘carcasses’\\
ekú- & → & ekwitíní- & ‘eyes’\\
ídò- & → & íditíní- & ‘breasts’\\
tsʼʉbà- & → & tsʼʉbɨtɨnɨ- & ‘stoppers’\\
\lspbottomrule
\end{tabularx}
\end{table}
The third plurative, \{-ìkà-\}, is used primarily to pluralize nouns with three or more syllables in their lexical root. \tabref{tab:4}.3 provides a sample of polysyllabic nouns pluralized with \{-ìkà-\}. Notice that if the singular noun has [-ATR] vowels, then the plurative suffix harmonizes to \{-ɨkà-\}. Unlike \{-ítíní-\} but like \{-íkó-\}, \{-ìkà-\} sometimes alters the tone of the singular noun as well as having its own tone altered:


\begin{table}
\caption{3: The plurative suffix \{-ìkà-\} with polysyllabic nouns}
\label{tab:4}


\begin{tabularx}{\textwidth}{XXXX}
\lsptoprule

Singular &  & Plural & \\
àgɨtà- & → & ágɨtɨkà- & ‘metal ringlets’\\
arírá- & → & aríríkà- & ‘flames’\\
bàbàà- & → & bábàìkà- & ‘armpits’\\
ɔfɔrɔƙɔ- & → & ɔfɔrɔƙɨkà- & ‘dry honeycombs’\\
kútúŋù- & → & kútúŋìkà- & ‘knees’\\
ɲánɨnɔɔ- & → & ɲánɨnɔɨkà- & ‘leather whips’\\
ɲéƙúrumotí- & → & ɲéƙúrùmòtìkà- & ‘gullies’\\
\lspbottomrule
\end{tabularx}
\end{table}
Secondarily, the plurative \{-ìkà-\} is used to pluralize a handful of nouns that have only two syllables in their lexical root. Why these nouns do not take \{-ítíní-\} instead is not known. A bit of speculation might invoke the notion of \textsc{mora} or the unit of syllable weight. Among the seven examples shown in \tabref{tab:4}.4, three of them contain the semi-vowel /w/ which may be thought to contain its own mora, as a vowel would. Likewise, two of the examples (\textit{hòò-} and \textit{sédà-}) contain depressor consonants which may also count for one mora. Perhaps in the remaining two (\textit{kíʝá-} and \textit{ríʝá-}), the /ʝ/ used to be a depressor. Regardless of the historical explanation, \tabref{tab:4}.4 presents a few examples of \{-ìkà-\} being used to pluralize disyllabic nouns:


\begin{table}
\caption{4: The plurative suffix \{-ìkà-\} with disyllabic nouns}
\label{tab:4}


\begin{tabularx}{\textwidth}{XXXX}
\lsptoprule

Singular &  & Plural & \\
awá- & → & àwìkà- & ‘homes’\\
gwasá- & → & gwàsìkà- & ‘stones’\\
hòò- & → & hòìkà- & ‘huts’\\
kíʝá- & → & kíʝíkà- & ‘lands’\\
kwɛtá- & → & kwɛtɨkà- & ‘arms’\\
ríʝá- & → & ríʝíkà- & ‘forests’\\
sédà- & → & sédìkà- & ‘gardens’\\
\lspbottomrule
\end{tabularx}
\end{table}

\subsubsection{Suppletive plurals}

Icétôd also has a handful of singular nouns cannot be pluralized in a productive way with any of the three suffixes discussed above. Three of these nouns on record are truly \textsc{suppletive} in that their singular and plural forms bear absolutely no resemblance to each other. These are the first three in \tabref{tab:4}.5. The last three examples in \tabref{tab:4}.5 represent nouns that are semi-suppletive; even though one can discern a similarity between the singular and plural forms, the way the two forms are derived from each other is not productive in the language.  


\begin{table}
\caption{5: Icétôd suppletive plurals}
\label{tab:4}


\begin{tabularx}{\textwidth}{XXXX}
\lsptoprule

Singular &  & Plural & \\
ámá- & $\leftrightarrow $ & ròɓà- & ‘people’\\
eakwá- & $\leftrightarrow $ & ɲɔtɔ- & ‘men’\\
imá- & $\leftrightarrow $ & wicé- & ‘children’\\
cekí- & $\leftrightarrow $ & cɨkámá- & ‘women’\\
ɗɨ{}- & $\leftrightarrow $ & ɗi- & ‘ones’\\
kɔrɔɓádì- & $\leftrightarrow $ & kúrúɓádì- & ‘things’\\
\lspbottomrule
\end{tabularx}
\end{table}

\subsubsection{Singulatives (\textsc{sing})}

In contrast to pluratives, \textsc{singulatives} convert an inherently plural noun root to a derived singular. Icétôd has one such suffix that may be considered a true singulative in the contemporary grammar of the modern language, and that is \{-àmà-\} or \{-ɔmà-\}. Since this singulative is only used with personal entities, it seems likely that it is related etymologically to the word \textit{ámá-} ‘person’. \tabref{tab:4}.6 gives the only four unambiguous examples of when this singulative is used. Note that its tone pattern may be altered by the tone of the plural root:


\begin{table}
\caption{6: The Icétôd singulative \{-àmà-\}}
\label{tab:4}


\begin{tabularx}{\textwidth}{XXXX}
\lsptoprule

Plural &  & Singular & \\
ʝáká- & → & ʝákámà- & ‘elder’\\
kéà- & → & kéàmà- & ‘soldier’\\
lɔŋɔtá- & → & lɔŋɔtɔmà- & ‘enemy’\\
ŋɨmɔkɔkaá- & → & ŋɨmɔkɔká-ámà- & ‘young man’\\
\lspbottomrule
\end{tabularx}
\end{table}

\subsubsection{Possessive number suffixes (\textsc{poss})}

In addition to standard pluratives and a singulative, Icétôd also has what may be called \textsc{possessive} number suffixes. These possessive suffixes—\{-èdè-\} in the singular and \{-ìnì-\} in the plural—each fuse the notions of number and possession into one morpheme. When they are affixed to a noun stem, they specify a) the grammatical number of the noun stem and b) its association with another entity (hence the ‘possession’). They do not specify the number of the possessing entity. For example, the word \textit{akedᵃ}, a stem consisting of \textit{aká-} ‘den’ and \{-èdè-\} (in the nominative case) can mean both ‘its den’ or ‘their den’. And the word \textit{akɨn}, consisting of \textit{aká-} ‘den’ and \{-ìnì-\} (in the nominative case), can mean either ‘its dens’ or ‘their dens’. 

Within the broad notion of ‘possession’, the possessive number suffixes \{-èdè-\} and \{-ìnì-\} can signify more specific semantic relationships like part-whole, kinship, and association. \tabref{tab:4}.7 gives some examples of \{-èdè-\} expressing a part-whole relationship with the unnamed entity. Note how the meanings of the noun roots are extended metaphorically to denote structural parts of things. Note also that the tone of the root may be altered in the presence of \{-èdè-\}:


\begin{table}
\caption{7: The Icétôd singular possessive \{-èdè-\}}
\label{tab:4}


\begin{tabularx}{\textwidth}{XXXXX}
\lsptoprule

\multicolumn{2}{X}{Root meaning} &  & \multicolumn{2}{X}{Extended part-whole meaning}\\
bakutsí- & ‘chest’ & → & bakútsédè- & ‘its middle part’\\
bùbùì- & ‘belly’ & → & búbùèdè- & ‘its underside’\\
ekú- & ‘eye’ & → & ekwede- & ‘its essence’\\
kwayó- & ‘tooth’ & → & kweede- & ‘its edge’\\
ŋabérí- & ‘rib’ & → & ŋábèrèdè- & ‘its side’\\
\lspbottomrule
\end{tabularx}
\end{table}
The plural possessive suffix \{-ìnì-\} has two special applications with human possessors. In the first, it is used to pluralize kinship terms, where a kinship association is explicitly implied. In the second, it refers to people associated with a certain person in general terms. \tabref{tab:4}.8 illustrates both of these nuances, showing the singular root in the first column, and in the second, the root inflected with \{-ìnì-\}:


\begin{table}
\caption{8: The Icétôd plural possessive \{-ìnì-\}}
\label{tab:4}


\begin{tabularx}{\textwidth}{XXXX}
\lsptoprule

Kinship &  &  & \\
abáŋɨ- & → & abáŋɨnɨ- & ‘my fathers (uncles)’\\
dádòò- & → & dádoíní- & ‘your grandmothers’\\
ŋɔɔ- & → & ŋɔɨnɨ- & ‘your mothers’\\
tátàà- & → & tátaíní- & ‘my aunts’\\
wicé- & → & wikini- & ‘his/her/their/its children’\\
Association &  &  & \\
\`{A}ɗùpàà- & → & Aɗupaíní- & ‘the people of Aɗupa’\\
Dakáɨ- & → & Dakáɨnɨ- & ‘the people of Dakai’\\
Lóʝérèè- & → & Lóʝéreíní- & ‘the people of Loʝere’\\
Ŋirikoó- & → & Ŋirikoíní- & ‘the people of Ŋiriko’\\
Tsɨláà- & → & Tsɨláɨnɨ- & ‘the people of Tsila’\\
\lspbottomrule
\end{tabularx}
\end{table}

\subsubsection{Mass nouns}

A small group of Icétôd noun roots are classified as non-count \textsc{mass} \textsc{nouns}. These nouns are inherently, lexically plural. As such, they require plural demonstratives and relative pronouns. This group includes words for powders, liquids, and gases—particulate substances. \tabref{tab:4}.9 presents seven examples of mass nouns. The roots are in the first column, followed in the third column by the noun in a phrase with the plural demonstrative \textit{ni} ‘those’. Note that in the English, the equivalent is provided but with a singular interpretation.


\begin{table}
\caption{9: Icétôd non-countable mass nouns}
\label{tab:4}


\begin{tabularx}{\textwidth}{XXXX}
\lsptoprule

búré- & ‘dust’ & búrá ni & ‘this dust’\\
cué- & ‘water’ & cua ni & ‘this water’\\
kabasá- & ‘flour’ & kabasa ni & ‘this flour’\\
sèà- & ‘blood’ & sea ni & ‘this blood’\\
tsʼúdè- & ‘smoke’ & tsʼúda ni & ‘this smoke’\\
\lspbottomrule
\end{tabularx}
\end{table}

\subsubsection{Transnumeral nouns}

Another small group of Icétôd noun roots appear as inherently \textsc{transnumeral}, meaning that they can be singular or plural depending on what the speaker wants to communicate. Whatever number is imputed to them must be reflected in the grammar of the rest of the sentence, for example in subject-agreement on the verb or in any demonstratives or relative pronouns used to modify them. Icétôd transnumeral nouns cannot be pluralized in any of the ways discussed up to this point. But with the bound nominal morpheme \textit{{}-icíká-} (see §4.3.4), they can be given a sense of distributiveness or variation. \tabref{tab:4}.10 presents three examples of Icétôd transnumeral nouns with their singular, plural, and distributive interpretations:


\begin{table}
\caption{10: Icétôd transnumeral nouns}
\label{tab:4}


\begin{tabularx}{\textwidth}{XXX}
\lsptoprule

Root & ɓìɓà- & ‘egg(s)’\\
Singular & ɓiɓa na & ‘this egg’\\
Plural & ɓiɓa ni & ‘these eggs’\\
Distributive & ɓiɓaicíká- & ‘various kinds of eggs’\\
Root & gwaá- & ‘bird(s)’\\
Singular & gwaa na & ‘this bird’\\
Plural & gwaa ni & ‘these birds’\\
Distributive & gwaicíká- & ‘various kinds of birds’\\
Root & ínó- & ‘animal(s)’\\
Singular & ínwá na & ‘this animal’\\
Plural & ínwá ni & ‘these animals’\\
Distributive & ínóicíká- & ‘various kinds of animals’\\
\lspbottomrule
\end{tabularx}
\end{table}



\subsection{Compounds}


For word-building purposes, Icétôd relies heavily on \textsc{compounding}, joining two or more nouns together into a composite word. The first noun (or pronoun) in a compound retains its lexical root form (that is hyphenated throughout this book), including its lexical tone. The last noun in a compound takes whichever case ending the syntactic context calls for. For example, in the compound \textit{riéwíkᵃ} ‘goat kids’, the first root \textit{rié-} ‘goat’ keeps its lexical form, while the second, \textit{wicé-} ‘children’, has been modified by the nominative case suffix \{-ᵃ\}. If compounding changes the tone of its constituent parts, it will be the first noun that affects the others. In the rare compound with three constituent nouns, the first two stay in their lexical form (not counting tone), while the third is inflected for case, for example in \textit{Icémóríɗókàkà-} ‘cowpea leaves’, a compound of \textit{Icé-} ‘Ik’, \textit{mòrìɗò-} ‘beans’, and \textit{kaká-} ‘leaves’. In \textit{Icé-móríɗó-kàkà-}, note that while the last two elements retain their lexical segments, their tone patterns have changed dramatically due to the influence of \textit{Icé-} in spreading H tone.

Icétôd compounds create two kinds of new meaning: 1) a narrower, more specific meaning in which the first noun specifies the second, or 2) a completely novel, unpredictable meaning. An example of the first type would be \textit{bʉbʉnɔɔʝà-} ‘ember-wound’ or ‘bullet wound’ where the first noun \textit{bʉbʉná-} ‘ember’ narrows down the possible references of \textit{ɔʝá-} ‘wound’ to a wound caused by a bullet. And an example of the second type of compounded meaning might be \textit{óbiʝoetsʼí-} that literally means ‘rhino urine’ but is actually the name of a species of vine (that nonetheless was apparently the favorite urination spot of rhinos). Through both types of meaning, Icétôd compounds add a considerable amount of expressiveness and color to the language’s vocabulary.

In addition to the two broader semantic categories of compounds discussed above, five other categories of Icétôd compounds are recognized. These include the agentive, diminutive, internal, variative, and relational. Each of these is briefly touched on below.


\subsubsection{Agentive (\textsc{agt})}

Icétôd forms \textsc{agentive} compounds by using the root \textit{ámá-} ‘person’ (for singular) or \textit{icé-} (for plural) as the last element in a compound. Although the root \textit{icé-} simply means ‘Ik people’ when standing on its own, in the agentive construction it denotes plural agents. Here ‘agent’ is understood broadly as any person or thing that does or is whatever is characterized by the first element in the compound. The first element may be a noun, as in \textit{dɛá-ámà-} ‘messenger’, literally ‘foot-person’, or a verb as in \textit{ŋwàxɔnɨ-àmà-} ‘lame person’, literally ‘to be lame-person’. Note, however, that even though \textit{ŋwàxɔn} is a verb semantically, it has been deverbalized into a noun by the infinitive suffix \{-òn\}. Icétôd agentive compounds can be translated into English in various ways, depending on what is appropriate. \tabref{tab:4}.11 presents several example of singular and plural agentive compounds:


\begin{table}
\caption{11: Icétôd agentive compounds}
\label{tab:4}


\begin{tabularx}{\textwidth}{XXXX}
\lsptoprule

Singlar & Plural &  & \\
aká-ámà- & aká-ícé- & mouth-person & ‘talker’\\
ɓɛƙɛsɨ-àmà- & ɓɛƙɛsí-ícé- & walking-person & ‘traveler’\\
itelesí-ámà- & itelesí-ícé- & watching-person & ‘watchman’\\
kɔŋɛsɨ-àmà- & kɔŋɛsí-ícé- & cooking-person & ‘cook’\\
ɲósomá-ámà- & ɲósomá-ícé- & studies-person & ‘student’\\
sɨsɨká-ámà- & sɨsɨká-ícé- & middle-person & ‘middle child’\\
yʉɛ-ámà- & yué-ícé- & lie-person & ‘liar’\\
\lspbottomrule
\end{tabularx}
\end{table}

\subsubsection{Diminutive (\textsc{dim})}

Icétôd forms \textsc{diminutive} compounds by using the root \textit{imá-} ‘child’ (for singular) and \textit{wicé-} ‘children’ (for plural) as the second element in a compound. In the more literal interpretation, the first element is the animate being (animal or human) of which the second element is the ‘child’ or ‘children’, as in \textit{ɗóɗò-ìmà-} ‘lamb’ or \textit{ɗóɗo-wicé-} ‘lambs’. But when the first element is inanimate, the diminutive construction conveys a sense of ‘a small X’ or ‘small Xs’, for example \textit{ƙɔfó-ìmà-} ‘a small gourd bowl’ and \textit{ƙɔfó-wicé-} ‘small gourd bowls’. Lastly, the two interpretations can also get blurred, as when an animate being is perceived as smaller than normal but not as the child of anything. This can be seen, for instance, in the compound \textit{ídèmè-ìmà-} ‘earthworm’, literally ‘snake-child’. \tabref{tab:4}.12 offers several more examples of the diminutive compound. Notice that when the whole construction is pluralized, both elements may get pluralized, as when \textit{ámá-ìmà-} ‘someone’s child’ becomes \textit{roɓa-wicé-} ‘someone’s (pl.) children’.


\begin{table}
\caption{12: Icétôd diminutive compounds}
\label{tab:4}


\begin{tabularx}{\textwidth}{XXXX}
\lsptoprule

Singular & Plural &  & \\
ámá-ìmà- & roɓa-wicé- & person-child & ‘someone’s child’\\
bàrò-ìmà- & bárítíní-wicé- & herd-child & ‘small herd’\\
ɓɨsá-ímà- & ɓɨsɨtɨní-wicé- & spear-child & ‘dart’\\
dómá-ìmà- & dómítíní-wicé- & pot-child & ‘small pot’\\
gwá-ímà- & gwá-wícé- & bird-child & ‘chick’\\
ŋókí-ìmà- & ŋókítíní-wicé- & dog-child & ‘puppy’\\
ɔʝá-ìmà- & ɔʝɨtɨnɨ-wicé- & sore-child & ‘small sore’\\
\lspbottomrule
\end{tabularx}
\end{table}

\subsubsection{Internal (\textsc{int})}

So-called \textsc{internal} compounds are made with the bound nominal root \textit{aʝɨká-} ‘among/inside’. When appended to plural noun, this nominal conveys a sense of interiority or internality to the noun. The internal compound, which is quite rare, is exemplified in \tabref{tab:4}.13:


\begin{table}
\caption{13: Icétôd internal compounds}
\label{tab:4}


\begin{tabularx}{\textwidth}{XXXXX}
\lsptoprule

Plural &  &  & Interal plural & \\
àwìkà- & ‘homes’ & → & awika-aʝɨká- & ‘in/among homes’\\
ríʝíkà- & ‘forests’ & → & ríʝíka-aʝɨká- & ‘in/among forests’\\
sédìkà- & ‘gardens’ & → & sédika-aʝɨká- & ‘in/among gardens’\\
\lspbottomrule
\end{tabularx}
\end{table}

\subsubsection{Variative (\textsc{var})}

So-called \textsc{variative} compounds are made with the bound nominal root \textit{icíká-} ‘various (kinds of)’. When appended to a noun—singular or plural—this nominal communicates a sense of variety or the multiplicity of a type. As a kind of pluralizer itself, \textit{icíká-} is may be called upon to pluralize five kinds of nouns: 1) transnumeral nouns, 2) nouns not usually pluralizeable in the usual sense, 3) inherently plural nouns, 4) already pluralized nouns, and 5) verb infinitives. \tabref{tab:4}.14 presents one example for each of these five kinds of nouns that the variative bound nominal \textit{icíká-} can be used to pluralize:


\begin{table}
\caption{14: Icétôd variative compounds}
\label{tab:4}


\begin{tabularx}{\textwidth}{XXXXX}
\lsptoprule

\multicolumn{2}{X}{Singular/Plural} &  & \multicolumn{2}{X}{Variative plural}\\
gwaá- & ‘bird(s)’ & → & gwa-icíká- & ‘kinds of birds’\\
cɛmá- & ‘fights’ & → & cɛmá-ícíká- & ‘war’\\
mɛná- & ‘issues’ & → & mɛná-ícíká- & ‘various issues’\\
dakwítíní- & ‘trees’ & → & dakwítíní-icíká- & ‘kinds of trees’\\
wetésí- & ‘to drink’ & → & wetésí-icíká- & ‘drinks’\\
\lspbottomrule
\end{tabularx}
\end{table}

\subsubsection{Relational} 

Icétôd compounding is also used to create \textsc{relational nouns} that express the spatial or structural relationship one thing has to another. As many languages do, Icétôd metaphorically extends body-part terminology to other non-bodily structural relationships. \tabref{tab:4}.15 presents some of the Icétôd body-part terms used metaphorically:


\begin{table}
\caption{15: Icétôd body-part terms with extended meanings}
\label{tab:4}


\begin{tabularx}{\textwidth}{XXX}
\lsptoprule

Root & Lexical meaning & Relational meaning\\
aká- & ‘mouth’ & ‘entrance, opening’\\
aƙatí- & ‘nose’ & ‘handle, stem’\\
bakutsí- & ‘chest’ & ‘front part’\\
bùbùì- & ‘belly’ & ‘underside’\\
dɛá- & ‘foot’ & ‘base, foot’\\
ekú- & ‘eye’ & ‘center, point’\\
gúró- & ‘heart’ & ‘core, essence’\\
iká- & ‘head’ & ‘head, top’\\
kwayó- & ‘tooth’ & ‘edge’\\
ŋabérí- & ‘rib’ & ‘side’\\
\lspbottomrule
\end{tabularx}
\end{table}
So, in a relational compound, terms like those in \tabref{tab:4}.15 are the second element in the compound, a position in which they denote the ‘part’ in a ‘whole-part’ semantic relationship. Accordingly, the first element in the compound represents the ‘whole’ in the relationship. \tabref{tab:4}.16 displays a handful of such ‘whole-part’ compounds:


\begin{table}
\caption{16: Icétôd relational compounds}
\label{tab:4}


\begin{tabularx}{\textwidth}{XXX}
\lsptoprule

Roots & Lexical meaning & Relational meaning\\
aká-kwáyó- & mouth-tooth & ‘lip’\\
dáŋá-àkà- & termite-mouth & ‘termite mound hole’\\
dòɗì-èkù- & vagina-eye & ‘cervix’\\
fátára-bakutsí- & ridge-chest & ‘front of vertical ridge’\\
fetí-ékù- & sun-eye & ‘east’\\
kaiɗeí-áƙátí- & pumpkin-nose & ‘pumpkin stem’\\
kwará-dɛà- & mountain-foot & ‘base of mountain’\\
kwaré-ékù- & mountain-eye & ‘saddle between peaks’\\
taɓá-dɛà- & boulder-foot & ‘base of boulder’\\
tsʼaɗí-ákà- & fire-mouth & ‘flame’\\
\lspbottomrule
\end{tabularx}
\end{table}

\section{Pronouns}



\subsection{Overview}


\textsc{Pronouns} ‘stand in’ for nouns that are not explicitly mentioned. Most Icétôd pronouns are free-standing words, but the subject-agreement pronominals and the dummy pronominal are suffixes that are bound to verbs (and so are treated in §8 on verbs). In a sentence, free pronouns are handled just like nouns in that they take case. The free pronouns discussed in this section fall into the following nine categories: personal, the impersonal possessum, indefinite, interrogative, demonstrative, relative, reflexive, distributive, and cohortative.




\subsection{Personal pronouns}


Icétôd \textsc{personal pronouns} represent the various grammatical persons that can be referred to in a sentence. The name is slightly misleading in that the pronouns can also denote nonpersonal, inanimate entities expressed by ‘it’ and ‘they’ (when referring to things). The Icétôd personal pronoun system operates along three axes: person (1, 2, 3), number (\textsc{sg}, \textsc{pl}), and clusivity (\textsc{exc}, \textsc{inc}). The ‘first person’ refers to ‘I’ and ‘we’, the second to ‘you’, and the third to ‘she’, ‘he’, ‘it’, and ‘they’. ‘Number’ (singular or plural) obviously has to do with whether the entity is one or more than one. And ‘clusivity’ (\textsc{exclusive} or \textsc{inclusive}) tells whether the addressee of the speech is \textit{ex}cluded from or \textit{in}cluded in the reference of ‘we’. \tabref{tab:5}.1 presents the seven Icétôd personal pronouns in their lexical forms, while \tabref{tab:5}.2 on the next page offers a full case declension of them:


\begin{table}
\caption{1: Icétôd personal pronouns}
\label{tab:5}


\begin{tabularx}{\textwidth}{XXX}
\lsptoprule

\textsc{1sg} & ɲcì- & ‘I’\\
\textsc{2sg} & bì- & ‘you’\\
\textsc{3sg} & ntsí- & ‘s/he/it’\\
\textsc{1pl.exc} & ŋgó- & ‘we’\\
\textsc{1pl.inc} & ɲjíní- & ‘we all’\\
\textsc{2pl} & bìtì- & ‘you all’\\
\textsc{3pl} & ńtí- & ‘they’\\
\lspbottomrule
\end{tabularx}
\end{table}
\setcounter{page}{1}
\begin{table}
\caption{2: Case declension of Icétôd personal pronouns}
\label{tab:5}


\begin{tabularx}{\textwidth}{XXXXXXXXXXXXXXX} & \multicolumn{2}{X}{ ‘I’} & \multicolumn{2}{X}{ ‘you’} & \multicolumn{2}{X}{ ‘s/he/it’} & \multicolumn{2}{X}{ ‘we’} & \multicolumn{2}{X}{ ‘we all’} & \multicolumn{2}{X}{ ‘you all’} & \multicolumn{2}{X}{ ‘they’}\\
\lsptoprule
& \textsc{nf} & \textsc{ff} & \textsc{nf} & \textsc{ff} & \textsc{nf} & \textsc{ff} & \textsc{nf} & \textsc{ff} & \textsc{nf} & \textsc{ff} & \textsc{nf} & \textsc{ff} & \textsc{nf} & \textsc{ff}\\
\textsc{nom} & \'{ŋ}kà & \'{ŋ}kᵃ & bìà & bì & ntsa & ntsᵃ & ŋgwa & ŋgwᵃ & ɲjíná & ɲjín & bìtà & bìtᵃ & ńtá & ńtᵃ\\
\textsc{acc} & ɲcìà & ɲcìkᵃ & bìà & bìkᵃ & ntsíá & ntsíkᵃ & ŋgóá & ŋgókᵃ & ɲjíníà & ɲjíníkᵃ & bìtìà & bìtìkᵃ & ńtíà & ńtíkᵃ\\
\textsc{dat} & ɲcìè & ɲcìkᵉ & bìè & bìkᵉ & ntsíé & ntsíkᵉ & ŋgóé & ŋgókᵉ & ɲjíníè & ɲjíníkᵉ & bìtìè & bìtìkᵉ & ńtíè & ńtíkᵉ\\
\textsc{gen} & ɲcìè & ɲcì & bìè & bì & ntsíé & ntsí & ŋgóé & ŋgóᵉ & ɲjíníè & ɲjíní & bìtìè & bìtì & ńtíè & ńtí\\
\textsc{abl} & ɲcùò & ɲcù & bùò & bù & ntsúó & ntsú & ŋgóó & ŋgó & ɲjínúò & ɲjínu & bìtùò & bìtù & ńtúò & ńtú\\
\textsc{ins} & \'{ŋ}kò & \'{ŋ}kᵒ & bùò & bù & ntso & ntsᵒ & ŋgo & ŋgᵒ & ɲjínó & ɲjínᵒ & bìtò & bìtᵒ & ńtó & ńtᵒ\\
\textsc{cop} & ɲcùò & ɲcùkᵒ & bùò & bùkᵒ & ntsúó & ntsúkᵒ & ŋgóó & ŋgókᵒ & ɲjínúò & ɲjínúkᵒ & bìtùò & bìtùkᵒ & ńtúò & ńtúkᵒ\\
\textsc{obl} & ɲcì & ɲcⁱ & bì & bì & ntsi & ntsⁱ & ŋgo & ŋgᵒ & ɲjíní & ɲjín & bìtì & bìtⁱ & ńtí & ńtⁱ\\
\lspbottomrule
\end{tabularx}
\end{table}



\subsection{Impersonal possessum pronoun (\textsc{pssm})}


Icétôd also has a special pronoun whose only function is to represent a \textsc{possessum}, that is, an entity associated with another entity (a \textsc{possessor}) through a general relationship of possession or association. This pronoun has the form \textit{ɛnɨ-} and is bound to another noun or pronoun in a compound construction. It is \textsc{impersonal} in that it communicates nothing about the possessor or the possessum except for the relationship of possession itself. The impersonal possessum pronoun can be in a compound with personal pronouns or other nouns. \tabref{tab:5}.3 shows \textit{ɛnɨ-} with all seven personal pronouns:


\begin{table}
\caption{3: Icétôd impersonal possessum with pronouns}
\label{tab:5}


\begin{tabularx}{\textwidth}{XXX}
\lsptoprule

ɲj-ɛnɨ- & I-\textsc{possessum} & ‘mine’\\
bi-ɛnɨ- & you-\textsc{possessum} & ‘yours’\\
nts-ɛnɨ- & s/he/it-\textsc{possessum} & ‘hers/his/its’\\
ŋgó-ɛnɨ- & we-\textsc{possessum} & ‘ours’\\
ɲjíní-ɛnɨ- & we all-\textsc{possessum} & ‘all of ours’\\
biti-ɛnɨ- & you all-\textsc{possessum} & ‘all of yours’\\
ńtí-ɛnɨ- & they-\textsc{possessum} & ‘theirs’\\
\lspbottomrule
\end{tabularx}
\end{table}
The impersonal possessum pronoun \textit{ɛnɨ-} can also be used with full nouns (even deverbalized verbs) as the compound’s first element. This type of possessive construction is illustrated below in \tabref{tab:5}.4:


\begin{table}
\caption{4: Icétôd impersonal possessum with nouns}
\label{tab:5}


\begin{tabularx}{\textwidth}{XXX}
\lsptoprule

aɗoni-ɛnɨ- & to be three-\textsc{possessum} & ‘the third time’\\
cɨkámɛ-ɛnɨ- & women-\textsc{possessum} & ‘the women’s’\\
ɦyɔ-ɛnɨ & cattle-\textsc{possessum} & ‘the foreigners’’\\
Icé-ɛnɨ- & Ik-\textsc{possessum} & ‘the Ik’s’\\
ɲɔtɔ-ɛnɨ- & men-\textsc{possessum} & ‘the men’s’\\
roɓe-ɛnɨ- & people-\textsc{possessum} & ‘the people’s’\\
wicé-ɛnɨ- & children-\textsc{possessum} & ‘the children’s’\\
\lspbottomrule
\end{tabularx}
\end{table}



\subsection{Indefinite pronouns}


Pronouns that are \textsc{indefinite} stand for other entities but with a certain degree of indefiniteness or vagueness. All but one of the Icétôd indefinite pronouns are based on the root \textit{kɔnɨ-} ‘one’ or its plural counterpart \textit{kíní-} ‘more than one’. The one that is not based on these roots is \textit{saí-} ‘some more/other’, a root that may not actually belong with this set but is included on the basis of its English translation. \tabref{tab:5}.5 provides a run-down of the main Icétôd indefinite pronouns:


\begin{table}
\caption{5: Icétôd indefinite pronouns}
\label{tab:5}


\begin{tabularx}{\textwidth}{XXX}
\lsptoprule

kɔnɨ- & one & ‘another, some (sg.)’\\
kɔn-áí- & one-place & ‘somewhere (else)’\\
kɔnɨ-ɛnɨ- & one-\textsc{possessum} & ‘a(n), some (sg.)’\\
kɔnɨ-ámà- & one-person & ‘somebody, someone’\\
kɔn-ɔmà- & one-\textsc{singulative} & ‘some unknown person’\\
kíní-ámá- & many-person & ‘some unknown people’\\
kíní-ɛnɨ- & many-\textsc{possessum} & ‘some (pl.)’\\
saí- & some & ‘some more, some other’\\
\lspbottomrule
\end{tabularx}
\end{table}



\subsection{Interrogative pronouns}


The role of \textsc{interrogative} pronouns is to query the identity of the entity they represent. As a result, they are used to form questions. All but one of the Icétôd interrogative pronouns incorporate the ancient northeastern African interrogative particle \textit{*nd-/nt-}, and the one that is not has the form \textit{ìsì-} ‘what’. The small handful of six Icétôd interrogative pronouns are provided below in \tabref{tab:5}.6. 


\begin{table}
\caption{6: Icétôd interrogative pronouns}
\label{tab:5}


\begin{tabularx}{\textwidth}{XXX}
\lsptoprule

ìsì- & what & ‘what?’\\
ndaí- & ?-place & ‘where?’\\
ǹdò- & who & ‘who?’\\
ńt- & ? & ‘where?’\\
ńtɛɛnɨ- & ?-possessum & ‘which (sg.)’\\
ńtíɛnɨ- & ?-possessum & ‘which (pl.)’\\
\lspbottomrule
\end{tabularx}
\end{table}
In a question, Icétôd interrogative pronouns take the same slot as the nouns they are representing. But it is also common for the interrogative pronoun to be ‘fronted’: moved to the first place in the sentence for emphasis. When this happens, the pronoun is given the copulative case (see §7.8). Example sentences (1)-(2) are provided below to illustrate fronting. Both orders are perfectly acceptable. For more on how questions are formed in Icétôd, please refer to §10.4.3.




\ea\label{ex:}
\gll {Bɛɗɨdà}   \textit{ìs}\textit{?}      \textit{Isi}o     bɛɗɨdᵃ? \\
    \\
want:\textsc{2sg}   what:\textsc{nom}    what:\textsc{cop}   want:2\textsc{sg}
\glt ‘You want what?’      ‘What do you want? 
\z




\ea\label{ex:}
\gll {Ia     ndaíkᵉ?    Ndaíó   iâdᵉ?} \\
    \\
be:\textsc{3sg}   where:\textsc{dat}    where:\textsc{cop}   be:\textsc{3sg:dp}
\glt ‘It is where?’      ‘Where is it?’ 
\z






\subsection{Demonstrative pronouns}


Icétôd has a set of \textsc{demonstrative} pronouns that referentially ‘demonstrate’ or point to an entity. They are all based on the singular form \textit{ɗɨ{}-} ‘this (one)’ or the plural form \textit{ɗi-} ‘these (ones)’ that differ formally only in regard to their vowel (/ɨ/ versus /i/). The Icétôd demonstrative pronoun system is divided in three categories based on spatial distance from the speaker: 1) \textsc{proximal}, meaning near the speaker, 2) \textsc{medial}, a relatively medium distance from the speaker, and 3) \textsc{distal}, meaning relatively far from the speaker. The medial and distal forms, for both singular and plural, consist of the root \textit{ɗɨ{}-/ɗi-} preceded by the cliticized distal demonstratives \textit{kɨ} ‘that’ (derived from \textit{ke}) for singular and \textit{ki} ‘those’ for plural. Note further that the only difference between the medial and distal pronouns in the tone pattern whereby the medial form has a high tone on the last syllable, while the distal form does not. \tabref{tab:5}.7 presents the Icétôd demonstrative pronouns in their lexical forms, while \tabref{tab:5}.8 gives the full case declensions of all six base forms. Note that the medial and distal forms are indistinguishable except in the \textsc{nom}, \textsc{ins}, and \textsc{obl} cases:


\begin{table}
\caption{7: Icétôd demonstrative pronouns}
\label{tab:5}


\begin{tabularx}{\textwidth}{XXXXX} & Singular &  & Plural & \\
\lsptoprule
Proximal & ɗɨ{}- & ‘this’ & ɗi- & ‘these’\\
Medial & kɨɗɨ- & ‘that’ & kɨɗɨ- & ‘those\\
Distal & kɨɗɨ- & ‘that’ & kɨɗɨ- & ‘those’\\
\lspbottomrule
\end{tabularx}
\end{table}

\begin{table}
\caption{8: Case declensions of the demonstrative pronouns}
\label{tab:5}


\begin{tabularx}{\textwidth}{XXXXXXX} & \multicolumn{2}{X}{ Proximal} & \multicolumn{2}{X}{ Medial} & \multicolumn{2}{X}{ Distal}\\
\lsptoprule
& \textsc{sg} & \textsc{pl} & \textsc{sg} & \textsc{pl} & \textsc{sg} & \textsc{pl}\\
\textsc{nom} & ɗa & ɗa & kɨɗá & kiɗá & kɨɗa & kiɗa\\
\textsc{acc} & ɗɨá & ɗíá & kɨɗɨá & kiɗíá & kɨɗɨá & kiɗíá\\
\textsc{dat} & ɗɛɛ & ɗíé & kɨɗɛɛ & kiɗíé & kɨɗɛɛ & kiɗíé\\
\textsc{gen} & ɗɛɛ & ɗíé & kɨɗɛɛ & kiɗíé & kɨɗɛɛ & kiɗíé\\
\textsc{abl} & ɗɔɔ & ɗúó & kɨɗɔɔ & kiɗúó & kɨɗɔɔ & kiɗúó\\
\textsc{ins} & ɗɔ & ɗo & kɨɗɔ & kiɗó & kɨɗɔ & kiɗo\\
\textsc{cop} & ɗɔɔ & ɗúó & kɨɗɔɔ & kiɗúó & kɨɗɔɔ & kiɗúó\\
\textsc{obl} & ɗɨ & ɗi & kɨɗɨ & kiɗí & kɨɗɨ & kiɗi\\
\lspbottomrule
\end{tabularx}
\end{table}



\subsection{Relative pronouns (\textsc{rel})}


The role of \textsc{relative} pronouns is to introduce a relative clause: a clause embedded in a main clause to specify the reference of an entity in the main clause. A fascinating thing about the Icétôd relative pronoun system is that it is tensed. That is, it is able to encode the time period at which the statement contained in the relative clause holds or held true. The five time periods covered by these pronouns are 1) \textsc{non-past}, 2) \textsc{recent past} (earlier today), 3) \textsc{removed past} (yester-), 4) \textsc{remote past} (a while ago), and 5) \textsc{remotest past} (long ago).

The Icétôd relative pronouns are all enclitics based on the proto- demonstratives \textit{na} ‘this’ and \textit{ni} ‘these’ (see §6.2 below). Those forms are identical to the non-past relative pronouns \textit{na} ‘that/which’ and \textit{ni} ‘that/which (pl.)’. To create the other tensed versions of these relative pronouns, the language has employed one prefix and several suffixes that are affixed to the base form. \tabref{tab:5}.9 shows the whole paradigm:


\begin{table}
\caption{9: Icétôd relative pronouns}
\label{tab:5}


\begin{tabularx}{\textwidth}{XXXX} & Singular & Plural & \\
\lsptoprule
Non-past & =na & =ni & ‘that/which...’\\
Recent past & =náa & =níi & ‘that/which...’\\
Removed past & =sɨna & =sini & ‘that/which...’\\
Remote past & =nótso & =nútsu & ‘that/which...’\\
Remotest past & =noo & =nuu & ‘that/which...’\\
\lspbottomrule
\end{tabularx}
\end{table}
In sentence, no matter where an Icétôd relative clause (RC) appears, the relative pronoun will introduce it as the first element in the clause. The entity in the main clause that the relative clause is modifying—called the \textsc{common argument}—must be the last word before the relative clause. As a clitic, the relative pronoun attaches to the common argument. Examples (3)-(4) are given below to illustrate the syntactic position of relative pronouns and clauses. But to learn more about the syntax of relative clauses, please refer to §10.3.2.




\ea\label{ex:}
\gll {Atsáá     ceka     [náa       ƙwaatetᵃ]}\textsc{\textsubscript{rc}}. \\
    \\
come:\textsc{3sg:prf}   woman:\textsc{nom} =\textsc{rel:sg} give.birth:\textsc{3sg}
\glt ‘The woman [who gave birth today] has come.’ 
\z




\ea\label{ex:}
\gll {Tɔŋɔlano     rie     [sini     detí]}\textsc{\textsubscript{rc}}. \\
    \\
slaughter:\textsc{hort}   goats:\textsc{obl}      =\textsc{rel:pl}   bring:\textsc{1sg}
\glt ‘Let’s slaughter the goats [that I brought yesterday].’ 
\z






\subsection{Reflexive pronoun}


Icétôd has a \textsc{reflexive} pronoun that ‘reflects’ the impact of a verb back onto the subject of the verb. In other words, with the reflexive, the subject and object of an action are the same entity. The Icétôd reflexive pronoun has the form \textit{asɨ-} in the singular and \textit{ásɨkà-} in the plural which can be translated as ‘-self’ and ‘-selves’, respectively. Mostly likely, this pronoun is related to the word \textit{as} ‘body’ in Sɔɔ/Tepeth, one of Icétôd’s sister Kuliak languages. This link is further supported by the fact that another way Icétôd expresses reflexivity is by using its own word for ‘body’, \textit{nébù-}, as in \textit{Isio náa kawuƙóídee binébùkᵃ} ‘Why did you chop yourself (lit. ‘your body’)?’.

The reflexive pronouns are used extensively to make \textsc{semi-transitive} verbs: verbs falling between transitive and intransitive. For example, while the verb \textit{ídzòn} ‘to discharge, emit’ is intransitive and the verb \textit{ídzès} ‘to discharge, emit, shoot’ is transitive, the verb \textit{ídzesa asɨ} ‘to shoot across (lit. to ‘shoot -self’)’ is ‘semi-transitive’ because the subject and object of the shooting are the same entity. The full case declensions of the reflexive pronouns are given below in \tabref{tab:5}.10. But first, examples sentences (5)-(6) are provided to illustrate the reflexive and semi-transitive usages of these special pronouns:




\ea\label{ex:}
\gll {Kwatsítúƙoe     as.} \\
    \\
small:\textsc{caus:comp:imp}   self:\textsc{obl}
\glt ‘Humble yourself (lit: make yourself small).’ 
\z




\ea\label{ex:}
\gll {Ƙ}aio     dzúíka   ɨtɨɗɨɗátie     ásɨkàkᵃ. \\
    \\
go:\textsc{seq}   thieves:\textsc{nom}   sneak:\textsc{3pl:sim} selves:\textsc{acc}
\glt ‘The thieves went slinking away.’ 
\z



\begin{table}
\caption{10: Case declensions of the reflexive pronouns}
\label{tab:5}


\begin{tabularx}{\textwidth}{XXXXX} & Singular &  & Plural & \\
\lsptoprule
& \textsc{nf} & \textsc{ff} & \textsc{nf} & \textsc{ff}\\
\textsc{nom} & asa & as & ásɨkà & ásɨkᵃ\\
\textsc{acc} & asɨá & asɨkᵃ & ásɨkàà & ásɨkàkᵃ\\
\textsc{dat} & asɨɛ & asɨkᵋ & ásɨkɛɛ & ásɨkàkᵋ\\
\textsc{gen} & asɨɛ & asɨ & ásɨkɛɛ & ásɨkàᵋ\\
\textsc{abl} & asʉɔ & asʉ & ásɨkɔɔ & ásɨkàᵓ\\
\textsc{ins} & asɔ & asᵓ & ásɨkɔ & ásɨkᵓ\\
\textsc{cop} & asʉɔ & asʉkᵓ & ásɨkɔɔ & ásɨkàkᵓ\\
\textsc{obl} & asɨ & as & ásɨkà & ásɨkᵃ\\
\lspbottomrule
\end{tabularx}
\end{table}


\section{Demonstratives}



\subsection{Overview}


Icétôd’s \textsc{demonstratives} grammatically point to a referent. In the case of \textsc{nominal} demonstratives, the referent is an entity named by a noun, whereas \textsc{adverbial} demonstratives point to scene or situation of some sort. The Icétôd nominal demonstratives are all \textsc{enclitics} that come just after their host (the referent), as in \textit{ámá=nà} ‘this person’. Because the locative adverbial demonstratives function as adverbs, they tend to come at the end of the clause they are modifying. Unlike demonstrative pronouns (see §5.6), spatial and temporal demonstratives are not nouns and therefore never take case endings.




\subsection{Spatial demonstratives (\textsc{dem})}


Icétôd’s \textsc{spatial} demonstratives locate their referent in physical space in degrees of distance from the speaker. For singular referents, there are three degrees of distance: \textsc{proximal} (near), \textsc{medial} (relatively near/far), and \textsc{distal} (more distant). For plural referents, the language only distinguishes between proximal and distal. The singular demonstratives are usually translated into English as ‘this’ and ‘that’ and the plural ones as ‘these’ or ‘those’. \tabref{tab:6}.1 below presents the whole set of spatial nominal demonstratives. Notice that in their final forms (\textsc{ff}), their final vowels \textit{may} be whispered or omitted altogether:


\begin{table}
\caption{1: Icétôd spatial demonstratives}
\label{tab:6}


\begin{tabularx}{\textwidth}{XXXXX} & Singular &  & Plural & \\
\lsptoprule
& \textsc{nf} & \textsc{ff} & \textsc{nf} & \textsc{ff}\\
Proximal & =nà & =na (=n) & =nì & =ni (=n)\\
Medial & =nè & =na (=n) &  & \\
Distal & =kè & =ke (=kᵉ) & =kì & =ki (=kⁱ)\\
\lspbottomrule
\end{tabularx}
\end{table}
Spatial demonstratives usually directly follow their referent, as in:




\ea\label{ex:}
\gll {Eakwóó   ɗa} n. \\
    \\
man:\textsc{cop}  this.one:\textsc{nom}=\textsc{dem.sg.prox}
\glt ‘This one is a \textit{man}.’ 
\z




\ea\label{ex:}
\gll {Káwese   koto   ríʝá} ke. \\
    \\
cut:\textsc{sps}   then   forest=\textsc{dem.sg.dist}
\glt ‘And then that forest over there was cut down.’ 
\z






\subsection{Temporal demonstratives (\textsc{dem.pst})}


The \textsc{temporal} demonstratives locate their referent in five periods of time: \textsc{non-past} (present and future), \textsc{recent} past (earlier today), \textsc{removed} past (yester-), \textsc{remote} past (a while ago before yesterday), and \textsc{remotest} past (long ago). The Ik language has both singular and plural temporal nominal demonstratives, and these are listed below in \tabref{tab:6}.2. These temporal demonstratives are usually translated into English as ‘this’ and ‘that’ in the singular, and ‘these’ and ‘those’ in the plural, but with a sense of time rather than location. Recall that Icétôd’s relative pronouns (\tabref{tab:5}.9) are identical in form to the temporal demonstratives in \tabref{tab:6}.2 below, only that because relative pronouns never occur before a pause, they lack the final forms (\textsc{ff}).


\begin{table}
\caption{2: Icétôd temporal demonstratives}
\label{tab:6}


\begin{tabularx}{\textwidth}{XXXXX} & Singular &  & Plural & \\
\lsptoprule
& \textsc{nf} & \textsc{ff} & \textsc{nf} & \textsc{ff}\\
Non-past & =nà & =n & =nì & =n\\
Recent past & =náà & =nákᵃ & =níì & =níkⁱ\\
Removed past & =sɨnà & =sɨn & =sìnì & =sìn\\
Remote past & =nótsò & =nótsò & =nútsù & =nútsù\\
Remotest past & =nòò & =nòkᵒ & =nùù & =nùkᵘ\\
\lspbottomrule
\end{tabularx}
\end{table}
Just like spatial demonstratives, temporal demonstratives directly follow the noun they refer to, as the following examples illustrate:




\ea\label{ex:}
\gll Ráʝéte     ɗɨ nákᵃ.\\
return:\textsc{ven:imp}   one:\textsc{obl=dem.sg.rec}\\
\glt ‘Give back the earlier one.’ 
\z




\ea\label{ex:}
\gll {Gaana   kaɨna} \textit{nótso}       Lopíarɨɛ     zùkᵘ. \\
    \\
bad:\textsc{3sg}   year:\textsc{nom}=\textsc{dem.sg.rem}   Lopiar.\textsc{gen} very
\glt That year (a while back) of Lopiar was very bad.’
\z  





\subsection{Anaphoric demonstratives (\textsc{anph})}


The \textsc{anaphoric} demonstratives locate their referent not in space or time \textit{per se} but in \textit{shared communicative context}. In other words, they point back to a referent that has either been mentioned already in the same discourse or is already known by both speaker and hearer by some other means. Icétôd has a singular and a plural anaphoric demonstrative which are clitics that have the same form in both non-final and final environments (i.e., their final vowels are not omitted). These anaphoric demonstratives, translated into English as ‘that’ in the singular and ‘those’ in the plural, are shown below in \tabref{tab:6}.3:


\begin{table}
\caption{3: Icétôd anaphoric demonstratives}
\label{tab:6}


\begin{tabularx}{\textwidth}{XX}
\lsptoprule

Singular & Plural\\
=déé & =díí\\
\lspbottomrule
\end{tabularx}
\end{table}
Ik anaphoric demonstratives also directly follow their referents, as in:




\ea\label{ex:}
\gll {Itíóna     ɲatala} déé. \\
    \\
be.important:\textsc{3sg}   tradition:\textsc{nom}=\textsc{anaph.sg}
\glt ‘That tradition (already discussed) is important.’ 
\z




\ea\label{ex:}
\gll {Atsa noo     roɓa} \textit{díí}            Sópìàᵒ. \\
    \\
come:\textsc{3sg}=\textsc{pst}   people:\textsc{nom}=\textsc{anaph.pl}  Ethipia:\textsc{abl}
\glt ‘Those people (already mentioned) came from Ethiopia.’ 
\z






\subsection{Adverbial demonstratives}
\subsubsection{Overview}

Besides the three types of nominal demonstratives described above, Icétôd also has a complex system of \textsc{adverbial} demonstratives that involve both locative and anaphoric locative reference. Unlike the nominal demonstratives, the adverbial demonstratives are technically nouns themselves in that they are marked for case and can take their own nominal demonstratives. Their function, however, is adverbial.


\subsubsection{Locative adverbial demonstratives}

The first type of adverbial demonstrative, the \textsc{locative adverbial} demonstrative, locates the state or event expressed in a clause in physical space. Icétôd has three sets of such demonstratives. Sets 1 and 2 are built on degree of distance (see \tabref{tab:6}.4 below), while Set 3, in addition to degree of distance, is also split into singular and plural. These demonstratives are usually translated into English as ‘here’, ‘there’, ‘over there’, etc., depending on relative distance. 


\begin{table}
\caption{4: Icétôd locative adverbial demonstratives}
\label{tab:6}


\begin{tabularx}{\textwidth}{XXX} & \multicolumn{1}{X}{Set 1} & Set 2\\
\lsptoprule
Proximal & \multicolumn{1}{X}{} & náxánà- (=nà)\\
Medial & \multicolumn{1}{X}{nédì- (=nè)} & \\
Distal & \multicolumn{1}{X}{kédì- (kè)} & kɨxánà- (=kè)\\
\multicolumn{1}{X}{Set 3} & Singular & Plural\\
Proximal & naí- (=nà) & nií- (=nì)\\
Medial & naí- (=nè) & \\
Distal & kɔɔ (=kè) & kií- (=kì)\\
\lspbottomrule
\end{tabularx}
\end{table}
Examples (7)-(8) illustrate the locative adverbial demonstratives:




\ea\label{ex:}
\gll {Ɨ}\textit{táɨa bee}     \textit{kɨxánee} kᵉ. \\
    \\
reach:\textsc{1sg}=\textsc{pst}   there=\textsc{dem.sg.dist}
\glt ‘I reached there yesterday.’ 
\z




\ea\label{ex:}
\gll {Ƙ}\textit{aini   dzígwaa}   \textit{naíé} ne. \\
    \\
go:\textsc{seq}   trade:\textsc{acc}   there=\textsc{dem.sg.med}
\glt ‘And they went to do trade just right there.’ 
\z




\subsubsection{Anaphoric locative demonstratives}

The second type of Icétôd adverbial demonstratives is \textsc{anaphoric} \textsc{locative}. Like the locative nominal demonstratives, these point to a specific place—or metaphorically, a specific time—while also signifying anaphorically that that place or time is already known, either from earlier in the discourse or for some other reason. Icétôd has two such demonstratives with roughly the same meaning, and these are \textit{tsʼɛdɛ-} and \textit{tʉmɛdɛ-} ‘there/then’. Because these are actually nouns, \tabref{tab:6}.5 presents a case declension of them:


\begin{table}
\caption{5: Case declension of anaphoric locative demonstratives}
\label{tab:6}


\begin{tabularx}{\textwidth}{XXX} & ‘there’ & ‘there’\\
\lsptoprule
\textsc{nom} & tsʼɛda & tʉmɛda\\
\textsc{acc} & tsʼɛdɛá & tʉmɛdɛá\\
\textsc{dat} & tsʼɛdɛɛ & tʉmɛdɛɛ\\
\textsc{gen} & tsʼɛdɛɛ & tʉmɛdɛɛ\\
\textsc{abl} & tsʼɛdɔɔ & tʉmɛdɔɔ\\
\textsc{ins} & tsʼɛdɔ & tʉmɛdɔ\\
\textsc{cop} & tsʼɛdɔɔ & tʉmɛdɔɔ\\
\textsc{obl} & tsʼɛdɛ & tʉmɛdɛ\\
\lspbottomrule
\end{tabularx}
\end{table}
Examples (9)-(10) illustrate the locative adverbial demonstratives:




\ea\label{ex:}
\gll {Ƙ}\textit{aa noo   óŋora ʝɨɨ}     tsʼɛdɛɛ. \\
    \\
go:\textsc{3sg=pst}   elephant:\textsc{nom}=also   there:\textsc{dat}
\glt ‘Even the elephants went there (already mentioned).’ 
\z




\ea\label{ex:}
\gll {Pɛlɛmʉɔ   saa}     tʉmɛdɔɔ. \\
    \\
appear:\textsc{seq}   others:\textsc{nom}   there:\textsc{abl}
\glt ‘And others appeared from there (already known).’ 
\z




\section{Case}



\subsection{Overview}


Icétôd has a \textsc{case} system. This means that every noun has a special marking to show what role it has in the sentence. Icétôd marks this role by means of a set of case \textsc{suffixes} (endings). Four of the cases are marked with suffixes consisting of a single vowel, while for three others, the suffix consists of /k/ plus a vowel. Another case, the oblique, is marked by the absence of any suffix. In the following examples, notice how the word \textit{ŋókí-} ‘dog’ at the end of each sentence has a different ending depending on the case for which it is marked:




\ea\label{ex:}
\gll {Atsa}     ŋókᵃ. \\
    \\
come:\textsc{3sg}   dog:\textsc{nom}
\glt ‘The dog comes.’ 
\z




\ea\label{ex:}
\gll {Cɛa}     \textit{boroka}     ŋókíkᵃ. \\
    \\
kill:\textsc{3sg}   bushpig:\textsc{nom}   dog:\textsc{acc}
\glt ‘The bushpig kills the dog.’ 
\z




\ea\label{ex:}
\gll {Maa}     eméá     ŋókíkᵉ. \\
    \\
give:\textsc{3sg}   meat:\textsc{acc}   dog:\textsc{dat}
\glt ‘He gives meat to the dog.’ 
\z




\ea\label{ex:}
\gll {Mɨta     ima}     ŋókí. \\
    \\
be:\textsc{3sg}   child:\textsc{nom}   dog:\textsc{gen}
\glt ‘It is the child of the dog.’ 
\z




\ea\label{ex:}
\gll {Xɛɓa}     ŋókú. \\
    \\
fear:\textsc{3sg}   dog:\textsc{abl}
\glt ‘He fears the dog.’ 
\z




\ea\label{ex:}
\gll {Ƙaa}     ŋókᵒ. \\
    \\
go:\textsc{3sg}  dog:\textsc{ins}
\glt ‘He goes with the dog.’ 
\z




\ea\label{ex:}
\gll {Bɛna}     ŋókúkᵒ. \\
    \\
not.be:\textsc{3sg}  dog:\textsc{cop}
\glt ‘It is not a dog.’ 
\z




\ea\label{ex:}
\gll {Mɨta}     \textit{ŋók}ⁱ. \\
    \\
be:\textsc{3sg}  dog:\textsc{obl}
\glt ‘It is a dog.’ 
\z


Eight examples are given above because Icétôd has eight cases: nominative, accusative, dative, genitive, ablative, instrumental, copulative, and oblique. \tabref{tab:7}.1 below presents the non-final and final forms of the suffixes that mark all eight of these cases. Keep in mind that the null symbol <Ø> signifies either 1) that the case suffix is inaudible or, for the oblique case, 2) that there is no case suffix. 


\begin{table}
\caption{1: Icétôd case suffixes}
\label{tab:7}


\begin{tabularx}{\textwidth}{XXXX}
\lsptoprule

Case & Abbreviation & Non-final & Final\\
Nominative & \textsc{nom} & {}-a & {}-ᵃ/-\textsuperscript{Ø}\\
Accusative & \textsc{acc} & {}-a & {}-kᵃ\\
Dative & \textsc{dat} & {}-e & {}-kᵉ\\
Genitive & \textsc{gen} & {}-e & {}-e/-\textsuperscript{Ø}\\
Ablative & \textsc{abl} & {}-o & {}-ᵒ/-\textsuperscript{Ø}\\
Instrumental & \textsc{ins} & {}-o & {}-ᵒ/-\textsuperscript{Ø}\\
Copulative & \textsc{cop} & {}-o & {}-kᵒ\\
Oblique & \textsc{obl} & {}-Ø & {}-\textsuperscript{Ø}\\
\lspbottomrule
\end{tabularx}
\end{table}
From \tabref{tab:7}.1, there may appear to be significant ambiguity in the Icétôd case system. For instance, the non-final forms of the nominative and accusative suffixes, the dative and genitive suffixes, and the ablative, instrumental, and copulative suffixes all look the same. In most cases, the key to disambiguating the suffixes is something called ‘subtractive’\textsc{} morphology. Some of the Icétôd case suffixes are subtractive in that they subtract or delete the final vowel of the noun to which they attach. The subtractive cases are the nominative and the instrumental. So, for example, while the non-final forms of the nominative and accusative are identical, their morphological behavior is not: the nominative \{-a\} subtracts the noun’s final vowel, as when \textit{ŋókí-} ‘dog’ becomes \textit{ŋók-á} ‘dog:\textsc{nom}’; by contrast, the accusative suffix is non-subtractive, as in \textit{ŋókí-à} ‘dog:\textsc{acc}’. Other case ambiguities like genitive versus dative and ablative versus copulative in their non-final forms can be resolved in the context of the sentence. Different verbs require different cases.

Since every Icétôd noun ends in a vowel, and since that vowel can be any of the nine (/i, ɨ, e, ɛ, a, ɔ, o, ʉ, u/), the collision of nouns and case suffixes gives rise to all kinds of vowel assimilation (see §2.4.4). The next two tables present declensions of two nouns illustrating vowel assimilation. \tabref{tab:7}.2 shows the noun \textit{fetí-} ‘sun’ declined for all eight cases. In particular, notice how the vowel /o/ in the ablative and copulative suffixes partially assimilate the /i/ in \textit{fetí-} to become /u/. 


\begin{table}
\caption{2: Case declension of \textit{fetí-} ‘sun’}
\label{tab:7}


\begin{tabularx}{\textwidth}{XXX}
\lsptoprule

Case & Non-final & Final\\
\textsc{nom} & feta & fetᵃ\\
\textsc{acc} & fetíá & fetíkᵃ\\
\textsc{dat} & fetíé & fetíkᵉ\\
\textsc{gen} & fetíé & fetí\\
\textsc{abl} & fetúó & fetú\\
\textsc{ins} & feto & fetᵒ\\
\textsc{cop} & fetúó & fetúkᵒ\\
\textsc{obl} & feti & fetⁱ\\
\lspbottomrule
\end{tabularx}
\end{table}
While \tabref{tab:7}.2 shows partial vowel assimilation caused by case suffixation, \tabref{tab:7}.3 shows an instance of total assimilation. In this table, the noun \textit{kíʝá-} ‘land’ is declined for all the eight cases. Note specifically how the final /a/ of \textit{kíʝá-} gets totally assimilated by the non-final dative, genitive, ablative, and copulative suffixes.


\begin{table}
\caption{3: Case declension of \textit{kíʝá-} ‘land’}
\label{tab:7}


\begin{tabularx}{\textwidth}{XXX}
\lsptoprule

Case & Non-final & Final\\
\textsc{nom} & kíʝá & kíʝᵃ\\
\textsc{acc} & kíʝáà & kíʝákᵃ\\
\textsc{dat} & kíʝéè & kíʝákᵉ\\
\textsc{gen} & kíʝéè & kíʝáᵉ\\
\textsc{abl} & kíʝóò & kíʝáᵒ\\
\textsc{ins} & kíʝó & kíʝᵒ\\
\textsc{cop} & kíʝóò & kíʝákᵒ\\
\textsc{obl} & kíʝá & kíʝᵃ\\
\lspbottomrule
\end{tabularx}
\end{table}




\subsection{Nominative (\textsc{nom})}


The \textsc{nominative} case, marked by the suffix \{-a\}, is the ‘naming’ case whose role is to do the following: 1) mark the subject of main clauses, 2) mark the subject of sequential clauses (see §8.10.7), and 3) mark the direct object of clauses with 1\textsuperscript{st} and 2\textsuperscript{nd} person subjects (‘I’, ‘we’, ‘you’). Three example are provided below, each one illustrating one of the three grammatical roles of the nominative case. The third example contains seven sentences to show how Icétôd object-marking is \textsc{split}: objects after 3\textsuperscript{rd}{}-person subjects take the accusative case, while 1\textsuperscript{st} or 2\textsuperscript{nd}{}-person subjects take objects in the nominative case.




Subject of a main clause
\ea\label{ex:}
\gll {Atsáá   lɔŋɔt-}ᵃ. \\
    \\
come:\textsc{prf}   enemies-\textsc{nom}
\glt ‘The enemies have come!’ 
\z




Subject of a sequential clause
\ea\label{ex:}
\gll {Toɓuo   ƙaƙaam-}\textit{a}   kʉláɓákᵃ. \\
    \\
spear:\textsc{seq}   hunter-\textsc{nom}   bushbuck:\textsc{acc}
\glt ‘And the hunter speared the bushbuck.’ 
\z




Object of a clause with a 1/2-person subject




\ea\label{ex:}
\gll {Ŋƙɨá   tɔbɔŋ-}\textit{a} na. \\
    \\
eat:\textsc{1sg}   mush-\textsc{nom}=this
\glt ‘I eat this meal mush.’ 
\z




\ea\label{ex:}
\gll {Ŋƙɨda   tɔbɔŋ-}\textit{a} na. \\
    \\
eat:\textsc{2sg}   mush-\textsc{nom}=this
\glt ‘You eat this meal mush.’ 
\z




\ea\label{ex:}
\gll {Ŋƙa   tɔbɔŋɔ-á na.} \\
    \\
eat:\textsc{3sg}   mush-\textsc{acc}=this
\glt ‘She eats this meal mush.’ 
\z




\ea\label{ex:}
\gll {Ŋƙɨmá     tɔbɔŋ-}\textit{a} na. \\
    \\
eat:\textsc{1pl.exc}   mush-\textsc{nom}=this
\glt ‘We eat this meal mush.’ 
\z




\ea\label{ex:}
\gll {Ŋƙɨsɨna     tɔbɔŋ-}\textit{a} na. \\
    \\
eat:\textsc{1pl.inc}   mush-\textsc{nom}=this
\glt ‘We all this meal mush.’ 
\z




\ea\label{ex:}
\gll {Ŋƙɨtá   tɔbɔŋ-}\textit{a} na. \\
    \\
eat:\textsc{2pl}   mush-\textsc{nom}=this
\glt ‘You all eat this meal mush.’ 
\z




\ea\label{ex:}
\gll {Ŋƙáta   tɔbɔŋɔ-á na.} \\
    \\
eat:3pl   mush-\textsc{acc}=this
\glt ‘They eat this meal mush.’ 
\z






\subsection{Accusative case (\textsc{acc})}


The \textsc{accusative} case, marked by the suffix \{-ka\}, is also split with regard to its basic function. One of its basic functions, that for which it is named, is to mark the direct object of any clause with a 3-person subject. Its other common function is to mark the subject \textit{and} any object of several kinds of subordinate (dependent) clauses (including relative and temporal clauses). Each of these functions is exemplified by one of the following example sentences. In the first example, a sentence with a 1-person subject is also given to show the contrast:




Direct object of a clause with a 3-person subject
\ea\label{ex:}
\gll {Wetésátà   mɛsɛ-}\textit{à}   mùɲ. \\
    \\
drink:\textsc{fut:3pl}   beer-\textsc{acc}   all
\glt ‘They will drink all the beer.’ 
\z




\ea\label{ex:}
\gll {Wetésímà     mɛs-à     mùɲ.} \\
    \\
drink:\textsc{fut:1pl.exc}   beer-\textsc{nom}   all
\glt ‘We will drink all the beer.’ 
\z





Subject and object of a subordinate clause
\ea\label{ex:}
\gll {Mee   kɔrɔɓadi}    \\
    \\
give:\textsc{imp}   thing:\textsc{obl}   
\z

\ea\label{ex:}
\gll {[náa   ɲci-}\textit{a}   detí.] \\
    \\
that\textsc{}  I-\textsc{acc}   bring:\textsc{1sg}
\glt ‘Give me the thing that I brought earlier.’ 
\z




\ea\label{ex:}
\gll {[Noo   ŋgó-}\textit{á}\textit{     bɛɗɨmɛɛ     bi-}\textit{a}],... \\
    \\
when   we-\textsc{acc}   want:\textsc{1pl.exc}   you-\textsc{acc}
\glt ‘When we were looking for you,...’ 
\z






\subsection{Dative (\textsc{dat})}


The \textsc{dative} case, marked by the suffix \{-ke\}, is the ‘to’ or ‘in’ case whose role is to mark indirect objects (also called ‘extended’ or ‘secondary’). These indirect objects may encode semantic ‘roles’ like: destination, location, reception, perception, possession, and purpose. Each of these is illustrated by one of the following example sentences:




Destination
\ea\label{ex:}
\gll {Ƙeesíá   awá-}kᵉ. \\
    \\
go:\textsc{fut:1sg}   home-\textsc{dat}
\glt ‘I’m going home.’ 
\z




Location
\ea\label{ex:}
\gll {Ia     sédà-}kᵉ. \\
    \\
be:\textsc{3sg}   garden-\textsc{dat}
\glt ‘She’s in the garden.’ 
\z


Reception\todo{Recipient?}
\ea\label{ex:}
\gll {Tɔkɔráta   kabasáá   ròɓà-}kᵉ. \\
    \\
divide:\textsc{3pl}   flour:\textsc{acc}   people-\textsc{dat}
\glt ‘They are dividing out flour to people.’ 
\z




Perception\todo{Eexperiencer?}
\ea\label{ex:}
\gll {Ɨɓálá     ɲcì-}\textit{è}   zùkᵘ. \\
    \\
appall:\textsc{3sg}   I-\textsc{dat}   very
\glt ‘It really appalls me.’ (Lit: ‘It is very appalling to me.’) 
\z




Possession
\ea\label{ex:}
\gll {Ia     ɦyɔa     ntsí-}kᵉ. \\
    \\
be:\textsc{3sg}   cattle:\textsc{nom}    he-\textsc{dat}
\glt ‘He has cattle.’ (Lit: ‘There are cattle to him.’) 
\z




Purpose
\ea\label{ex:}
\gll {Ƙaa     ɲera     dakúáƙɔ-}kᵋ. \\
    \\
go:\textsc{3sg}   girls:\textsc{nom}   wood:inside-\textsc{dat}
\glt ‘The girls go for firewood.’ 
\z






\subsection{Genitive (\textsc{gen})}


The \textsc{genitive} case, marked by the suffix \{-e\}, is the ‘of’ case whose role is to encode a possessive or associative relationship a noun has with another noun (or, in rare cases, with a verb). Within the broad notions of possession and association are finer nuances like: ownership, part-whole relationship, kinship, and attribution. These nuances are each illustrated with an example sentence below:




Ownership
\ea\label{ex:}
\gll {Hɔnɨnɨ   ɦyɔa     ńtí-}\textit{e}     ɓórékᵉ. \\
    \\
drive:\textsc{seq}   cattle:\textsc{acc}   they-\textsc{gen}   corral:\textsc{dat}
\glt ‘And they drove their cattle to the corral.’ 
\z




Part-whole relationship
\ea\label{ex:}
\gll {Wasá     dɛɛdɛɛ   kwará-}ᵉ. \\
    \\
stand:\textsc{3sg}   foot:\textsc{dat}   mountain-\textsc{gen}
\glt ‘He’s standing at the foot of the mountain.’ 
\z




Kinship
\ea\label{ex:}
\gll {Mɨná     cekíá     ntsí-}\textit{é}     zùkᵘ. \\
    \\
love:\textsc{3sg}   wife:\textsc{acc}   he-\textsc{gen}   very
\glt ‘He loves his wife very much.’ 
\z




Attribution
\ea\label{ex:}
\gll {Maráŋá   muceá   bì-}\textsuperscript{Ø}. \\
    \\
good:\textsc{3sg}   way:\textsc{nom}   you-\textsc{gen}
\glt ‘Your luck is good.’ (lit: Your way is good.) 
\z


The genitive case has two further roles. One is the \textsc{nominalization} of clauses, that is, when a whole clause is changed into a noun phrase that can be used as a subject or object in another clause. For example, the clause \textit{Cɛɨƙɔta náa eakwa ídèmèkᵃ} ‘The man killed the snake’ can be compressed into the nominalized \textit{cɛɛsʉƙɔta eakwéé ídèmè} ‘the killing of the man of the snake’ or ‘the man’s killing of the snake’. The other secondary role of the genitive has to do with verb \textit{ƙámón} ‘to be like’. For unknown historical reasons, this verb requires genitive case marking on its complement, as in \textit{Ƙámá ròɓèè mùɲ} ‘He’s like all people’, where \textit{ròɓè-è} is analyzed as ‘people-\textsc{gen’}.





\subsection{Ablative (\textsc{abl})}


The \textsc{ablative} case, marked by the suffix \{-o\}, is the ‘from’ case whose function is to mark objects with the following semantic roles: origin/source, cause, stimulus, source of judgment, location of activity (versus static location covered by the dative case). Each of these concepts are illustrated below with one example apiece:




Origin/source
\ea\label{ex:}
\gll {Atsía     awá-}ᵒ. \\
    \\
come:\textsc{1sg}   home-\textsc{abl}
\glt ‘I come from home.’ 
\z




Cause
\ea\label{ex:}
\gll {Baduƙota noo   ɲɛƙɛ{}-}ᵓ. \\
    \\
die:\textsc{3sg}=\textsc{pst}     hunger-\textsc{abl}
\glt ‘He died from hunger.’ 
\z




Stimulus
\ea\label{ex:}
\gll {Xɛɓa     ɲérà-}ᵒ. \\
    \\
fear:\textsc{3sg}   girls-\textsc{abl}
\glt ‘He’s shy of girls.’ 
\z




Source of judgment
\ea\label{ex:}
\gll {Daa     ɲcù-}\textsuperscript{Ø}. \\
    \\
nice:\textsc{3sg}   I-\textsc{abl}
\glt ‘It’s nice to me.’ 
\z




Location of activitiy
\ea\label{ex:}
\gll {Cɛmáta   sédìkà-}ᵒ. \\
    \\
fight:\textsc{3pl}   gardens-\textsc{abl}
\glt ‘They are fighting in the gardens.’ 
\z






\subsection{Instrumental (\textsc{ins})}


The \textsc{instrumental} case, marked by the suffix \{-o\}, is the ‘by’ or ‘with’ case. Unlike the ablative suffix \{-o\}, the instrumental suffix is subtractive, meaning that it first deletes the noun’s final vowel. The function of the instrumental case is to mark secondary objects with such semantic roles as: instrument/means, pathway, accompaniment, manner, time, and occupation. Each of these nuances is illustrated by one sentence each in the following examples:




Instrument/means
\ea\label{ex:}
\gll {Toɓíá   noo     gasoa       ɓɨs-}ᵓ. \\
    \\
spear:\textsc{1sg}=\textsc{pst}   warthog:\textsc{nom}   spear-\textsc{ins}
\glt ‘I speared a warthog with a spear.’ 
\z




Pathway
\ea\label{ex:}
\gll {Ƙaini     fots-}\textit{o}     gígìròkᵉ. \\
    \\
go:\textsc{3pl}   ravine-\textsc{ins}   downside:\textsc{dat}
\glt ‘And they went down by way of the ravine.’ 
\z



Accompaniment
\ea\label{ex:}
\gll {Atsímá naa     kúrúɓád-}\textit{o}   ŋgóᵉ. \\
    \\
come:\textsc{1pl}=\textsc{pst}   things-\textsc{ins}   we:\textsc{gen}
\glt ‘We came with our things.’ 
\z




Manner
\ea\label{ex:}
\gll {Ráʝétuo   ɲcie   gáánàs-}ᵓ. \\
    \\
answer:\textsc{3sg}   I:\textsc{dat}   badness-\textsc{ins}
\glt ‘He answered me with hostility.’ 
\z




Time
\ea\label{ex:}
\gll {Bɨraa     ɲɛƙa     ódoicik-}\textit{ó} ni. \\
    \\
lack:\textsc{3sg}   hunger:\textsc{nom}   days-\textsc{ins}=these
\glt ‘There is no hunger these days.’ 
\z




Occupation
\ea\label{ex:}
\gll {Cɛma     fítés-}\textit{o}   ƙwázìkàᵉ. \\
    \\
fight:\textsc{3sg}   washing-\textsc{ins}   clothes:\textsc{gen}
\glt ‘She’s washing clothes.’ (lit: ‘She is fighting with the washing of clothes.’) 
\z






\subsection{Copulative (\textsc{cop})}


The \textsc{copulative} case, marked by the suffix \{-ko\}, is the ‘is’ or ‘coupling’ case whose function is to link one noun to another in a relationship of exact identity. In this function, the copulative marks three kinds of nouns: 1) a focused (fronted) noun, 2) the complement of a verbless \textsc{copula} (linking verb) clause, and 3) the complement of a negative copula of identity clause. These different uses of the copulative are illustrated in the following sentences.




Fronted noun\todo{something seems to be missing here}




Fronted subject


\ea\label{ex:}
\gll {Ŋ}\textit{gó-}\textit{ó} naa   wetím. \\
    \\
we-\textsc{cop}=\textsc{pst}   drink:\textsc{1pl.exc}
\glt ‘It was we (who) drank (it).’ 
\z




Fronted object
\ea\label{ex:}
\gll {Emó-}\textit{ó}     bɛɗɨ. \\
    \\
meat-\textsc{cop}   want:\textsc{1sg}
\glt ‘It is meat (that) I want.’ 
\z




Fronted secondary object
\ea\label{ex:}
\gll {Ɲɛƙɔ{}-}\textit{ɔ}     ƙaiátèè   ƙàƙààƙɔkᵋ. \\
    \\
hunger-\textsc{cop}   go:\textsc{plur:3pl}   hunt:inside:\textsc{dat}
\glt ‘It is (due to) hunger (that) they keep going hunting.’ 
\z





Verbless copula complement
\ea\label{ex:}
\gll {Ìsù-}\textit{kᵒ}\textit{?   Ámó-}\textit{o}\textit{   keɗe...?   Ámá-}kᵒ. \\
    \\
what-\textsc{cop}   person-\textsc{cop}   or     person-\textsc{cop}
\glt ‘What is it? A person or...? It’s a person.’ 
\z




Negative copula complement
\ea\label{ex:}
\gll {Bɛna náá     ɲcù-}kᵒ. \\
    \\
not.be:\textsc{3sg}=\textsc{pst}   I-\textsc{cop}
\glt ‘It was not me!’ 
\z






\subsection{Oblique (\textsc{obl})}


The \textsc{oblique} case, marked by the absence of any suffix, is the ‘leftover’ case. As such, it is employed to mark nouns in a variety of disparate grammatical roles and functions. Among these are the following: 1) The subject and/or object of an imperative clause, 2) the subject and/or object of an optative clause, 3) the object of a preposition, and 3) a vocative noun. Each of these are demonstrated by at least one sentence in the examples below:




Subject and/or object of an imperative clause
\ea\label{ex:}
\gll {Deté     bi     cue dííǃ} \\
    \\
bring:\textsc{imp}   you:\textsc{obl}   water:\textsc{obl}=those
\glt ‘You bring that water!’ 
\z

Subject and/or object of an optative clause
\ea\label{ex:}
\gll {Ɲ}\'{} ci   nesíbine     emuti     ntsí. \\
    \\
I:\textsc{obl}   listen:\textsc{1sg:opt}   story:\textsc{obl}   he:\textsc{gen}
\glt ‘Let me listen to her story.’ 
\z




Object of a preposition
\ea\label{ex:}
\gll {Túbia     ima     ɲcia   páka   awᵃ.} \\
    \\
follow:\textsc{3sg}   child:\textsc{nom}   I:\textsc{acc}   until   home:\textsc{obl}
\glt ‘The child follows me up to home.’ 
\z




\ea\label{ex:}
\gll {Kirotánía  kóteré   ɦyekesí   bì.} \\
    \\
sweat:\textsc{1sg}   for     life:\textsc{obl}   you:\textsc{gen}
\glt ‘I sweat for your survival.’ 
\z





Vocative
\ea\label{ex:}
\gll {\'{E}é   wice,     atsúǃ} \\
    \\
hey   children:\textsc{obl}  come:\textsc{imp}
\glt ‘Hey children, come!’ 
\z




\section{Verbs}



\subsection{Overview}


Icétôd verbs consist of a verbal root (written in this book with a hyphen, as in \textit{wèt-} ‘drink’) and a variety of available derivational and inflectional suffixes. The language has no prefixes except those borrowed centuries ago that no longer have any active function, for example the /a/ in \textit{ábʉbʉƙ-} ‘bubble’ or the /i/ in \textit{iɓóɓór-} ‘hollow out’. Reduplicating a verb root, partially or totally, has long been a strategy for creating a sense of continuousness or repetitiveness, as when \textit{ɨtsán-} ‘disturb’ becomes \textit{ɨtsanɨtsán-} ‘torment relentlessly’. 

Icétôd employs a large number of suffixes to create longer verb stems. Among these are the \textsc{infinitive} and other deverbalizing suffixes that change a verb into a morphological noun that can take case endings, demonstratives, relative clauses, etc. One very key verb-building strategy of Icétôd is the so-called \textsc{directional} suffixes that signify the direction of the verb’s movement to or away from the speaker. These two directionals have also been extended metaphorically to express the beginning or completion of actions or processes. Another set of verbal suffixes deal with \textsc{voice} and \textsc{valency}, that is, the number of objects the verb requires. Among these are the \textsc{passive}, \textsc{impersonal} passive, \textsc{middle}, \textsc{causative}, and \textsc{reciprocal}.

Once a verb is taken from the mental lexicon and used in speech, it often requires \textsc{subject-agreement} marking, which Icétôd does with pronominal suffixes. Icétôd also has a special verbal suffix, the \textsc{dummy} \textsc{pronoun}, that goes on the verb whenever a peripheral argument, like a place or time designation, has been (re)moved.

The Icétôd verbal system has a variety of verbal paradigms based on \textsc{mood} and \textsc{aspect}. The basic distinction in mood is between \textsc{realis} and \textsc{irrealis}, or things that have happened and things that have not, respectively. Other modal distinctions include the \textsc{optative}, \textsc{subjunctive}, \textsc{imperative}, and \textsc{negative}. As for aspect, the specification of the internal structure of a verb—complete or incomplete—Icétôd has suffixes that mark \textsc{present perfect}, \textsc{intentional}{}-\textsc{imperfective}, \textsc{pluractional}, \textsc{sequential}, and \textsc{simultaneous}. Lastly, Icétôd exhibits a special set of \textsc{adjectival} suffixes to cover the language’s need to express adjectival concepts.




\subsection{Infinitives}
\subsubsection{Intransitive (\textsc{inf})}

\textsc{intransitive} verbs are those that allow only a subject—a direct object does not figure into its semantic schema. The Icétôd intransitive \textsc{infinitive} suffix is \{-ònì-\}. It converts an intransitive verb to a morphological noun that can be used as a noun in a noun phrase. The infinitive is the \textsc{citation} \textsc{form} of a verb, the form one cites in a dictionary or in isolation from other words. \tabref{tab:8}.1 gives a few examples of intransitive infinitives from the lexicon:


\begin{table}
\caption{1: Icétôd intransitive infinitives}
\label{tab:8}


\begin{tabularx}{\textwidth}{XXX}
\lsptoprule

Root & \multicolumn{2}{X}{Intransitive infinitive}\\
áƙáf- & áƙáfòn & ‘to yawn’\\
bòt- & bòtòn & ‘to migrate’\\
cɨ- & cɨɔn & ‘to be satiated’\\
dód- & dódòn & ‘to hurt’\\
ɛf- & ɛfɔn & ‘to be tasty’\\
gwɨr- & gwɨrɔn & ‘to squirm’\\
iƙú- & iƙúón & ‘to howl’\\
\lspbottomrule
\end{tabularx}
\end{table}
Because the infinitive is technically a morphological noun, it can be fully declined for case as all nouns can. \tabref{tab:8}.2 gives the case declension of the verb \textit{wàtònì-} ‘to rain’, which shows some vowel assimilation effects on [+ATR] vowels, as when /io/ becomes /uo/ in the ablative and copulative cases. \tabref{tab:8}.3 does the same for the [-ATR] verb \textit{wɛdɔnɨ-} ‘to detour’. Note /ɨɔ/ becoming /ʉɔ/ there as well:


\begin{table}
\caption{2: Case declension of \textit{wàtònì-} ‘to rain’}
\label{tab:8}


\begin{tabularx}{\textwidth}{XXX} & Non-final & Final\\
\lsptoprule
\textsc{nom} & wàtònà & wàtòn\\
\textsc{acc} & wàtònìà & wàtònìkᵃ\\
\textsc{dat} & wàtònìè & wàtònìkᵉ\\
\textsc{gen} & wàtònìè & wàtònì\\
\textsc{abl} & wàtònùò & wàtònù\\
\textsc{ins} & wàtònò & wàtònᵒ\\
\textsc{cop} & wàtònùò & wàtònùkᵒ\\
\textsc{obl} & wàtònì & wàtòn\\
\lspbottomrule
\end{tabularx}
\end{table}

\begin{table}
\caption{3: Case declension of \textit{wɛdɔnɨ-} ‘to detour’}
\label{tab:8}


\begin{tabularx}{\textwidth}{XXX} & Non-final & Final\\
\lsptoprule
\textsc{nom} & wɛdɔnà & wɛdɔn\\
\textsc{acc} & wɛdɔnɨà & wɛdɔnɨkᵃ\\
\textsc{dat} & wɛdɔnɨɛ & wɛdɔnɨkᵋ\\
\textsc{gen} & wɛdɔnɨɛ & wɛdɔnɨ\\
\textsc{abl} & wɛdɔnʉɔ & wɛdɔnʉ\\
\textsc{ins} & wɛdɔnɔ & wɛdɔnᵓ\\
\textsc{cop} & wɛdɔnʉɔ & wɛdɔnʉkᵓ\\
\textsc{obl} & wɛdɔnɨ & wɛdɔn\\
\lspbottomrule
\end{tabularx}
\end{table}

\subsubsection{Transitive (\textsc{inf})}

\textsc{transitive} verbs are those that admit a subject \textit{and} a direct object into its schematic of an active event. The Icétôd transitive infinitive suffix is \{-ésí-\}. It converts a transitive verb to a morphological noun that can be used as a noun in a noun phrase. \tabref{tab:8}.4 provides a few examples of transitive infinitives from the lexicon:


\begin{table}
\caption{4: Icétôd transitive infinitives}
\label{tab:8}


\begin{tabularx}{\textwidth}{XXX}
\lsptoprule

Root & \multicolumn{2}{X}{Transitive infinitive}\\
ágʉʝ- & ágʉʝɛs & ‘to gulp’\\
ban- & banɛs & ‘to sharpen’\\
cɛb- & cɛbɛs & ‘to roughen’\\
ɗóɗ- & ɗóɗés & ‘to point at’\\
erég- & erégès & ‘to employ’\\
gɨʝ- & gɨʝɛs & ‘to shave’\\
ɨlɔƙ- & ɨlɔƙɛs & ‘to dissolve’\\
\lspbottomrule
\end{tabularx}
\end{table}
%%please move \begin{table} just above \begin{tabular

\tabref{tab:8.5} gives the case declension of the deverbalized noun \textit{wetésí-} ‘to drink’, which shows vowel assimilation effects on [+ATR] vowels. \tabref{tab:8.6} does the same for the [-ATR] verb \textit{wɛtsʼɛsɨ-} ‘to knap’.
 



\begin{table}
\caption{Case declension of \textit{wetésí-} ‘to drink’}
\label{tab:8.5}


\begin{tabularx}{\textwidth}{XXX}
& Non-final & Final\\
\lsptoprule
\textsc{nom} & wetésá & wetés\\
\textsc{acc} & wetésíà & wetésíkᵃ\\
\textsc{dat} & wetésíè & wetésíkᵉ\\
\textsc{gen} & wetésíè & wetésí\\
\textsc{abl} & wetésúò & wetésú\\
\textsc{ins} & wetésó & wetésᵒ\\
\textsc{cop} & wetésúò & wetésúkᵒ\\
\textsc{obl} & wetésí & wetés\\
\lspbottomrule
\end{tabularx}
\end{table}

\begin{table}
\caption{6: Case declension of \textit{wɛtsʼɛsɨ-} ‘to knap’}
\label{tab:8}


\begin{tabularx}{\textwidth}{XXX} & Non-final & Final\\
\lsptoprule
\textsc{nom} & wɛtsʼɛsá & wɛtsʼɛs\\
\textsc{acc} & wɛtsʼɛsɨà & wɛtsʼɛsɨkᵃ\\
\textsc{dat} & wɛtsʼɛsɨɛ & wɛtsʼɛsɨkᵋ\\
\textsc{gen} & wɛtsʼɛsɨɛ & wɛtsʼɛsɨ\\
\textsc{abl} & wɛtsʼɛsʉɔ & wɛtsʼɛsʉ\\
\textsc{ins} & wɛtsʼɛsɔ & wɛtsʼɛsᵓ\\
\textsc{cop} & wɛtsʼɛsʉɔ & wɛtsʼɛsʉkᵓ\\
\textsc{obl} & wɛtsʼɛsɨ & wɛtsʼɛs\\
\lspbottomrule
\end{tabularx}
\end{table}

\subsubsection{Semi-transitive}

So-called \textsc{semi-transitive} verbs fall between transitive and intransitive in that they take an object, but the object is the reflexive pronoun \textit{asɨ-} ‘-self’. This means that semi-transitive verbs are morphologically transitive but almost intransitive semantically. Another name for this is ‘middle’ (although see another Icétôd middle verb in §8.6.3). \tabref{tab:8}.7 provides a sample of semi-transitive verbs from the lexicon. No case declension is given for these because they decline the same way as the transitive infinitives above in §8.2.2.


\begin{table}
\caption{7: Icétôd semi-transitive infinitives}
\label{tab:8}


\begin{tabularx}{\textwidth}{XXXXX}
\lsptoprule

Root & \multicolumn{2}{X}{} & \multicolumn{2}{X}{Semi-transitive infinitive}\\
bal- & \multicolumn{2}{X}{‘ignore’} & balɛsá asɨ & ‘to neglect -self’\\
\multicolumn{2}{X}{hoɗ-} & ‘free’ & hoɗésá asɨ & ‘to get freed’\\
\multicolumn{2}{X}{ɨrɨts-} & ‘keep’ & ɨrɨtsɛsa asɨ & ‘to control -self’\\
\multicolumn{2}{X}{ɨrʊts-} & ‘fling’ & ɨrʉtsɛsa asɨ & ‘to race across’\\
\multicolumn{2}{X}{ɨtɨŋ-} & ‘force’ & ɨtɨŋɛsa asɨ & ‘to force -self’\\
\multicolumn{2}{X}{kɔk-} & ‘close’ & kɔkɛsá asɨ & ‘to cover -self’\\
\multicolumn{2}{X}{toɓ-} & ‘spear’ & toɓésá asɨ & ‘to shoot across’\\
\lspbottomrule
\end{tabularx}
\end{table}



\subsection{Deverbalizers}
\subsubsection{Abstractive (\textsc{abst})}

The \textsc{abstractive} suffix \{-ásɨ-\} can be used to replace the intransitive suffix \{-ònì-\} to convert an intransitive verb to an abstract noun, for example when \textit{hábòn} ‘to be hot’ becomes \textit{hábàs} ‘heat’. \tabref{tab:8}.8 gives examples of abstract nouns derived from intransitive verbs:


\begin{table}
\caption{8: Icétôd abstract nouns derived from verbs}
\label{tab:8}


\begin{tabularx}{\textwidth}{XXXXX}
\lsptoprule

\multicolumn{3}{X}{Intransitive infinitive} & \multicolumn{2}{X}{Abstract noun}\\
ɓàŋɔn & \multicolumn{2}{X}{ ‘to be loose’} & ɓaŋás & ‘looseness’\\
\multicolumn{2}{X}{ɛfɔn} & ‘to be tasty’ & ɛfás & ‘(tasty) fat’\\
\multicolumn{2}{X}{gaanón} & ‘to be bad’ & gaánàs & ‘badness’\\
\multicolumn{2}{X}{ɦyɛtɔn} & ‘to be fierce’ & ɦyɛtás & ‘fierceness’\\
\multicolumn{2}{X}{kòmòn} & ‘to be many’ & komás & ‘manyness’\\
\multicolumn{2}{X}{ŋwàxɔn} & ‘to be disabled’ & ŋwaxás & ‘disability’\\
\multicolumn{2}{X}{xɛɓɔn} & ‘to be shy’ & xɛɓás & ‘shyness’\\
\lspbottomrule
\end{tabularx}
\end{table}
Because verbs deverbalized by the abstractive suffix are morphological nouns, they are fully declined for case. \tabref{tab:8}.9 gives one such case declension of the abstract noun \textit{kuɗásɨ-} ‘shortness’:


\begin{table}
\caption{9: Case declension of \textit{kuɗásɨ-} ‘shortness’}
\label{tab:8}


\begin{tabularx}{\textwidth}{XXX} & Non-final & Final\\
\lsptoprule
\textsc{nom} & kuɗásá & kuɗás\\
\textsc{acc} & kuɗásɨà & kuɗásɨkᵃ\\
\textsc{dat} & kuɗásɨɛ & kuɗásɨkᵋ\\
\textsc{gen} & kuɗásɨɛ & kuɗásɨ\\
\textsc{abl} & kuɗásʉɔ & kuɗásʉ\\
\textsc{ins} & kuɗásɔ & kuɗásᵓ\\
\textsc{cop} & kuɗásʉɔ & kuɗásʉkᵓ\\
\textsc{obl} & kuɗásɨ & kuɗás\\
\lspbottomrule
\end{tabularx}
\end{table}

\subsubsection{Behaviorative (\textsc{bhvr})}

The \textsc{behaviorative} suffix \{-nànèsì-\} first converts a noun to a verb and then the verb back into an abstract noun. (It is probably a complex suffix in which \{-(n)an-\} is the denominalizing element related to the stative suffix from §8.11.4 and \{-esi-\} the deverbalizing element related to the transitive suffix from §8.2.2. or the abstractive suffix from §8.3.1). Regardless of its composition, the suffix as a whole gives a noun-based stative concept the meaning of an abstract noun, as when \textit{ámá-} ‘person’ becomes \textit{ámánànès} ‘personhood’ or ‘personality’. \tabref{tab:8}.10 provides a few examples of behavioratives:


\begin{table}
\caption{10: Icétôd behaviorative abstract nouns}
\label{tab:8}


\begin{tabularx}{\textwidth}{XXXX}
\lsptoprule

\multicolumn{2}{X}{Noun root} & \multicolumn{2}{X}{Behaviorate noun}\\
babatí- & ‘his/her father’ & babatínánès & ‘fatherhood’\\
cekí- & ‘woman’ & cekínánès & ‘womanhood’\\
dzɔɗátɨ- & ‘rectum’ & dzɔɗátɨnànès & ‘grabbiness’\\
dzúú- & ‘theft’ & dzúnánès & ‘thievery’\\
imá- & ‘child’ & imánánès & ‘childhood’\\
lɔŋɔtá- & ‘enemy’ & lɔŋɔtánànès & ‘enmity’\\
ŋókí- & ‘dog’ & ŋókínànès & ‘poverty’\\
\lspbottomrule
\end{tabularx}
\end{table}
Because behavioratives are nouns, they are declined for case. \tabref{tab:8}.11 gives the case declension for the word \textit{eakwánánèsì}{}- ‘manhood’:


\begin{table}
\caption{11: Case declension of \textit{eakwánánèsì-} ‘manhood’}
\label{tab:8}


\begin{tabularx}{\textwidth}{XXX} & Non-final & Final\\
\lsptoprule
\textsc{nom} & eakwánánèsà & eakwánánès\\
\textsc{acc} & eakwánánèsìà & eakwánánèsìkᵃ\\
\textsc{dat} & eakwánánèsìè & eakwánánèsìkᵉ\\
\textsc{gen} & eakwánánèsìè & eakwánánèsì\\
\textsc{abl} & eakwánánèsùò & eakwánánèsù\\
\textsc{ins} & eakwánánèsò & eakwánánèsᵒ\\
\textsc{cop} & eakwánánèsùò & eakwánánèsùkᵒ\\
\textsc{obl} & eakwánánèsì & eakwánánès\\
\lspbottomrule
\end{tabularx}
\end{table}

\subsubsection{Patientive (\textsc{pat})}

The \textsc{patientive} suffix \{-amá-\} converts a verb to a noun that is characterized by the meaning of the verb. It is called ‘patientive’ because the derived noun usually fulfills the role of ‘patient’ or object of the original verb, as when \textit{meetés} ‘to give’ produces \textit{meetam} ‘gift’. \tabref{tab:8}.12 gives some examples of patientive nouns from the lexicon:


\begin{table}
\caption{12: Icétôd patientive nouns}
\label{tab:8}


\begin{tabularx}{\textwidth}{XXXX}
\lsptoprule

\multicolumn{2}{X}{Verb root} & \multicolumn{2}{X}{Patientive noun}\\
áts- & ‘chew’ & atsʼamá- & ‘chewy food’\\
ɓɛk- & ‘provoke’ & ɓɛkamá- & ‘provocation’\\
dʉb- & ‘knead’ & dʉbamá- & ‘dough’\\
dzígw- & ‘buy/sell’ & dzígwamá- & ‘merchandise’\\
gam- & ‘kindle’ & gamamá- & ‘kindling’\\
isúɗ- & ‘distort’ & isuɗamá- & ‘falsehood’\\
ŋƙ- & ‘eat’ & ŋƙamá- & ‘eatable’\\
\lspbottomrule
\end{tabularx}
\end{table}
Because patientives are nouns, they are fully declined for case. \tabref{tab:8}.13 gives the full declension of the noun \textit{wetamá-} ‘drink(able)’:


\begin{table}
\caption{13: Case declension of \textit{wetamá-} ‘drink(able)’}
\label{tab:8}


\begin{tabularx}{\textwidth}{XXX} & Non-final & Final\\
\lsptoprule
\textsc{nom} & wetama & wetam\\
\textsc{acc} & wetamáá & wetamákᵃ\\
\textsc{dat} & wetaméé & wetamákᵉ\\
\textsc{gen} & wetaméé & wetamáᵉ\\
\textsc{abl} & wetamóó & wetamáᵒ\\
\textsc{ins} & wetamo & wetamᵒ\\
\textsc{cop} & wetamóó & wetamákᵒ\\
\textsc{obl} & wetama & wetam\\
\lspbottomrule
\end{tabularx}
\end{table}



\subsection{Directionals}
\subsubsection{Venitive (\textsc{ven})}

The \textsc{venitive} suffix \{-ét-\} denotes a direction \textit{toward} a deictic center, usually (but not always) the speaker. It can be translated variously as ‘here’, ‘this way’, ‘out’, or ‘up’, but it is the Middle English word ‘hither’ that captures its essence nicely. The venitive suffix comes between the verb root and the infinitive suffix, whether intransitive or transitive. It can be used to augment any verb whose meaning includes motion or movement of any kind. \tabref{tab:8}.14 gives a few examples:


\begin{table}
\caption{14: Icétôd venitive verbs}
\label{tab:8}


\begin{tabularx}{\textwidth}{XXXX}
\lsptoprule

Intransitive &  & \multicolumn{2}{X}{Transitive}\\
arétón & ‘to cross this way’ & béberetés & ‘to pull this way’\\
ɦyɔtɔgɛtɔn & ‘to approach here’ & ɗʉrɛtɛs & ‘to pull out’\\
ɨlɛɛtɔn & ‘to come visit’ & futetés & ‘to blow this way’\\
irímétòn & ‘to rotate this way’ & hɔnɛtɛs & ‘to drive out’\\
ŋkéétòn & ‘to get up’ & ɨrɨŋɛtɛs & ‘to turn this way’\\
tɛɛtɔn & ‘to fall down’ & iteletés & ‘to watch here’\\
tʉwɛtɔn & ‘to sprout up’ & seɓetés & ‘to sweep up’\\
\lspbottomrule
\end{tabularx}
\end{table}
Venitive infinitives are morphological nouns and thus are declined for case. But since they end with intransitive or transitive suffixes, the reader is referred to §8.2.1 and §8.2.2 for similar case declensions.


\subsubsection{Andative (\textsc{and})}

The \textsc{andative} suffix \{-uƙot-\} denotes motion \textit{away from} a deictic center, usually the speaker (but not always). It can be translated variously as ‘away’, ‘off’, ‘out’, ‘that way’, or ‘there’, but it is the Middle English word ‘thither’ that captures its essence nicely. Unlike the venitive suffix, the andative comes after both the verbal root and the infinitive suffix (in an infinitival construction). It can be used to augment any verb whose meaning includes motion or movement of any kind. \tabref{tab:8}.15 provides a few examples of andative verbs:


\begin{table}
\caption{15: Icétôd andative verbs}
\label{tab:8}


\begin{tabularx}{\textwidth}{XXXX}
\lsptoprule

Intransitive &  & \multicolumn{2}{X}{Transitive}\\
aronuƙotᵃ & ‘to cross that way’ & hɔnɛsʉƙɔtᵃ & ‘to drive off/away’\\
botonuƙotᵃ & ‘to move away’ & ɨɗɛɛsʉƙɔtᵃ & ‘to hide way’\\
bʉrɔnʉƙɔtᵃ & ‘to fly off/away’ & ídzesuƙotᵃ & ‘to shoot (away)’\\
ɨɓákɔnʉƙɔtᵃ & ‘to go next to’ & ígorésúƙotᵃ & ‘to cross over’\\
isépónuƙotᵃ & ‘to flow away’ & ƙanésúƙotᵃ & ‘to take away’\\
kúbonuƙotᵃ & ‘to go out of sight’ & maƙésúƙotᵃ & ‘to give away’\\
tʉlʉŋɔnʉƙɔtᵃ & ‘to storm off’ & tɔrɛsʉƙɔtᵃ & ‘to toss away’\\
\lspbottomrule
\end{tabularx}
\end{table}
Because the andative suffix comes after infinitive suffixes, whenever an andative infinitive is declined for case, it is the andative suffix that takes case endings. \tabref{tab:8}.16 gives a declension of the [+ATR] andative verb \textit{séɓésuƙotí-} ‘to sweep off’, while \tabref{tab:8}.17 gives a similar declension for the [-ATR] verb \textit{sɛkɛsʉƙɔtɨ-} ‘to scrub off’:


\begin{table}
\caption{16: Case declension of \textit{séɓésuƙotí-} ‘to sweep off’}
\label{tab:8}


\begin{tabularx}{\textwidth}{XXX} & Non-final & Final\\
\lsptoprule
\textsc{nom} & séɓésuƙota & séɓésuƙotᵃ\\
\textsc{acc} & séɓésuƙotíá & séɓésuƙotíkᵃ\\
\textsc{dat} & séɓésuƙotíé & séɓésuƙotíkᵉ\\
\textsc{gen} & séɓésuƙotíé & séɓésuƙotí\\
\textsc{abl} & séɓésuƙotúó & séɓésuƙotú\\
\textsc{ins} & séɓésuƙoto & séɓésuƙotᵒ\\
\textsc{cop} & séɓésuƙotúó & séɓésuƙotúkᵒ\\
\textsc{obl} & séɓésuƙoti & séɓésuƙotⁱ\\
\lspbottomrule
\end{tabularx}
\end{table}

\begin{table}
\caption{17: Case declension of \textit{sɛkɛsʉƙɔtɨ-} ‘to scrub off’}
\label{tab:8}


\begin{tabularx}{\textwidth}{XXX} & Non-final & Final\\
\lsptoprule
\textsc{nom} & sɛkɛsʉƙɔta & sɛkɛsʉƙɔtᵃ\\
\textsc{acc} & sɛkɛsʉƙɔtɨá & sɛkɛsʉƙɔtɨkᵃ\\
\textsc{dat} & sɛkɛsʉƙɔtɨɛ & sɛkɛsʉƙɔtɨkᵋ\\
\textsc{gen} & sɛkɛsʉƙɔtɨɛ & sɛkɛsʉƙɔtɨ\\
\textsc{abl} & sɛkɛsʉƙɔtʉɔ & sɛkɛsʉƙɔtʉ\\
\textsc{ins} & sɛkɛsʉƙɔtɔ & sɛkɛsʉƙɔtᵓ\\
\textsc{cop} & sɛkɛsʉƙɔtʉɔ & sɛkɛsʉƙɔtʉkᵓ\\
\textsc{obl} & sɛkɛsʉƙɔtɨ & sɛkɛsʉƙɔtᶤ\\
\lspbottomrule
\end{tabularx}
\end{table}



\subsection{Aspectuals}
\subsubsection{Inchoative (\textsc{inch})}

The \textsc{inchoative} suffix \{-ét-\} is identical to the venitive suffix described in §8.4.1, and this is because its meaning is a metaphorical extension of the meaning of the venitive. That is, the venitive meaning of ‘hither’ was extended to mean the beginning of a state or activity (for intransitives) or the starting up of some action or process (for transitives). The inchoative behaves morphologically (including case declensions) exactly the same as the venitive. \tabref{tab:8}.18 gives a few examples of intransitive and transitive verbs in the inchoative aspect:


\begin{table}
\caption{18: Icétôd inchoative verbs}
\label{tab:8}


\begin{tabularx}{\textwidth}{XXXX}
\lsptoprule

Intransitive &  & \multicolumn{2}{X}{Transitive}\\
aeétón & ‘to start ripening’ & balɛtɛs & ‘to ignore’\\
dikwétón & ‘to start dancing’ & ewanetés & ‘to take note of’\\
ɛkwɛtɔn & ‘to start early’ & hoɗetés & ‘to liberate’\\
ɨɛɓɛtɔn & ‘to grow cold’ & ináƙúetés & ‘to destroy’\\
lɛʝɛtɔn & ‘to catch fire’ & rɛɛtɛs & ‘to coerce’\\
tsekétón & ‘to grow bushy’ & taʝaletés & ‘to relinquish’\\
wasɛtɔn & ‘to refuse’ & tamɛtɛs & ‘to ponder\\
\lspbottomrule
\end{tabularx}
\end{table}

\subsubsection{Completive (\textsc{comp})}

The \textsc{completive} suffix \{-uƙot-\} is identical to the andative suffix described in §8.4.2, and this is because its meaning is a metaphorical extension of the meaning of the andative. That is, the andative meaning of ‘thither’ was extended to mean the completion of a change of state or activity (for intransitives) or the fulfillment of some action or process (for transitives). The completive behaves morphologically (including case declensions) exactly the same as the andative. \tabref{tab:8}.19 gives a few examples of lexical verbs in the completive aspect:


\begin{table}
\caption{19: Icétôd completive verbs}
\label{tab:8}


\begin{tabularx}{\textwidth}{XXXX}
\lsptoprule

Intransitive &  & \multicolumn{2}{X}{Transitive}\\
aeonuƙotᵃ & ‘to become ripe’ & anɛsʉƙɔtᵃ & ‘to remember’\\
barɔnʉƙɔtᵃ & ‘to become rich’ & dɔxɛsʉƙɔtᵃ & ‘to reprimand’\\
hábonuƙotᵃ & ‘to become hot’ & ɦyeésúƙotᵃ & ‘to learn’\\
hɛɗɔnʉƙɔtᵃ & ‘to shrivel up’ & kurésúƙotᵃ & ‘to defeat’\\
mɨtɔnʉƙɔtᵃ & ‘to become’ & ŋábɛsʉƙɔtᵃ & ‘to finish up’\\
sɛkɔnʉƙɔtᵃ & ‘to fade away’ & ŋƙáƙésuƙotᵃ & ‘to devour’\\
zoonuƙotᵃ & ‘to become big’ & toɓésúƙotᵃ & ‘to plunder’\\
\lspbottomrule
\end{tabularx}
\end{table}

\subsubsection{Pluractional (\textsc{plur})}

The \textsc{pluractional} suffix \{-í-\} denotes an action or state that is construed as inherently \textit{plural}. This notion of plurality can mean any of the following: 1) an intransitive action done more than once or done by more than one subject, 2) a state attributed more than once or of more than one subject, 3) a transitive action done more than once, done by more than one subject, or done to more than one object. In short, the pluractional suffix conveys the idea that the application of the verb is multiple. The pluractional suffix comes just before the infinitive suffix and is a dominant [+ATR] suffix. \tabref{tab:8}.20 gives a few examples of intransitive and transitive pluractional verbs:


\begin{table}
\caption{20: Icétôd pluractional verbs}
\label{tab:8}


\begin{tabularx}{\textwidth}{XXXX}
\lsptoprule

Intransitive &  & \multicolumn{2}{X}{Transitive}\\
kóníón & ‘to be one-by-one’ & abutiés & ‘to sip continually’\\
ŋatíón & ‘to run (of many)’ & esetiés & ‘to interrogate’\\
ŋkáíón & ‘to get up (of many)’ & gafariés & ‘to stab repeatedly’\\
toɓéíón & ‘to be usually right’ & nesíbiés & ‘to obey habitually’\\
tatíón & ‘to drip constantly’ & tirifiés & ‘to investigate’\\
\lspbottomrule
\end{tabularx}
\end{table}



\subsection{Voice and valence}
\subsubsection{Passive (\textsc{pass})}

The Icétôd \textsc{passive} suffix \{-ósí-\} has the unusual distinction of being able to modify both intransitive and transitive verbs. With intransitive verbs, the passive adds the nuance of characteristicness to the meaning of the verb, often with the help of root reduplication. With transitive verbs, it has the usual function of a passive, which is to convert the object of a transitive verb into the subject of an intransitive verb. \tabref{tab:8}.21 gives examples of both intransitive and transitive passives:


\begin{table}
\caption{21: Icétôd passives}
\label{tab:8}


\begin{tabularx}{\textwidth}{XXXX}
\lsptoprule

Intransitive &  & \multicolumn{2}{X}{Transitive}\\
botibotos & ‘to be migratory’ & búdòs & ‘to be hidden’\\
ɓɛkɛsɔs & ‘to be mobile’ & cookós & ‘to be guarded’\\
deƙwideƙos & ‘to be quarrelsome’ & ɗɔtsɔs & ‘to be joined’\\
ɗɛɲɨɗɛɲɔs & ‘to be restless’ & ʝʉɔs & ‘to be roasted’\\
gúránós & ‘to be hot-tempered’ & ŋáɲɔs & ‘to be open’\\
mɔɲɨmɔɲɔs & ‘to be gossipy’ & ógoós & ‘to be left’\\
tsuwoós & ‘to be active’ & tsáŋós & ‘to be anointed’\\
\lspbottomrule
\end{tabularx}
\end{table}
Another quirky feature of the Icétôd passive \{-ósí-\} is that it can function both as a passive infinitive suffix (taking case) and as a regular inflectional suffix followed by subject-agreement pronouns. When it is declined for case, it declines just like the transitive suffix \{-ésí-\} in §8.2.2. Example (1) below illustrates this in a sentence where the passive infinitive \textit{búdòsì-} ‘to be hidden’ gets the accusative case. Then, example (2) shows the same passive acting as a verb proper, taking the 3\textsc{pl} subject-agreement pronominal suffix \{-át-\}:




\ea\label{ex:}
\gll {Bɛɗáta   búd}\textit{òsì}{}-kᵃ. \\
    \\
want:\textsc{3pl}   hidden:\textsc{pass-acc}
\glt ‘They want to be hidden.’ 
\z




\ea\label{ex:}
\gll {Búd}\textit{os}{}-átᵃ. \\
    \\
hidden:\textsc{pass}{}-\textsc{3pl:real}
\glt ‘They are hidden.’ 
\z




\subsubsection{Impersonal passive (\textsc{ips})}

The \textsc{impersonal passive} suffix \{-àn-\} behaves like a typical passive in that it eliminates the agent of a transitive verb and promotes the object to subject. However, unlike the passive \{-ósí-\} described above, the impersonal passive cannot be specified for the person or number of its subject. Instead, it remains marked for 3\textsc{sg} regardless of who or what the subject may be. Another strange property of \{-an-\} is that it can be used with intransitive verbs as well (just like the passive). When used with intransitive verbs, it has the function of downplaying the identity of the subject. For this reason, it can often be translated as ‘People...’ or ‘One...’, as in \textit{Tódian} ‘People say (it)’. The impersonal passive is a sentence-level morpheme that does not exist in the lexicon, and so it must be illustrated in examples like (3)-(4):




\ea\label{ex:}
\gll {Ɨnɔmɛs}\textit{án}\textit{à}   bì. \\
    \\
beat:\textsc{fut:ips}  you:\textsc{nom}
\glt ‘You will be beaten.’ 
\z




\ea\label{ex:}
\gll {Ƙaí}\textit{án}\textit{à}   ƙàƙààƙɔkᵋ. \\
    \\
go:\textsc{plur:ips} hunt:inside:\textsc{dat}
\glt ‘People go hunting.’ (Lit. ‘It is gone for hunting.’) 
\z




\subsubsection{Middle (\textsc{mid})}

Icétôd has two \textsc{middle} suffixes: \{-m-\} and \{-ím-\}. Like the semi-transitive construction discussed in §8.2.3, the middle suffixes convert simple transitive verbs into something in the ‘middle’ of transitive and intransitive. That is, the Icétôd middle verbs convey that idea that if an action is done to an entity, it is the entity itself—if anything—doing it to itself alone, apart from any other explicit agent. The middles eliminate a clear agent and promote the patient to the subject position. 

The middle suffix \{-m-\} always has a vowel between it and the preceding verb root. This vowel is usually a copy of the root vowel, as when \textit{ɗusés} ‘cut’ becomes \textit{ɗusúmón} ‘to cut (alone)’, but it can also have a non-copy vowel as in \textit{bokímón} ‘to get caught’. For its part, the middle suffix \{-ím-\}—a dominant [+ATR] suffix—is always paired with the inchoative suffix \{-ét-\}, thereby forming the complex morpheme \{-ímét-\}. \tabref{tab:8}.22 below gives some examples of these two suffixes converting transitive verbs to middle verbs:


\begin{table}
\caption{22: Icétôd middle verbs}
\label{tab:8}


\begin{tabularx}{\textwidth}{XXXX}
\lsptoprule

Transitive &  & \multicolumn{2}{X}{Middle with \{-m-\}}\\
ŋáɲɛs & ‘to open’ & ŋáɲámòn & ‘to open (alone)’\\
pakés & ‘to split’ & pakámón & ‘to split (alone)’\\
pulés & ‘to pierce’ & pulúmón & ‘to go out’\\
raʝés & ‘to return’ & raʝámón & ‘to return (alone)’\\
terés & ‘to divide’ & terémón & ‘to divide (alone)’\\
Transitive &  & \multicolumn{2}{X}{Middle with \{-ímét-\}}\\
átsʼɛs & ‘to chew’ & atsʼímétòn & ‘to wear out (alone)’\\
iɓéléés & ‘to overturn’ & iɓéléìmètòn & ‘to overturn (alone)’\\
kɔkɛs & ‘to close’ & kokíméton & ‘to close (alone)’\\
rébès & ‘to deprive’ & rébìmètòn & ‘to be deprived (alone)’\\
tɔrɛɛs & \multicolumn{1}{X}{‘to coerce’} & toreimétòn & ‘to be coerced (alone)’\\
\lspbottomrule
\end{tabularx}
\end{table}

\subsubsection{Reciprocal (\textsc{recip})}

The \textsc{reciprocal} suffix \{-ínósí-\} denotes a reciprocal relationship that a verb’s subject has with itself. That is, the reciprocal collapses the subject and direct object of a transitive verb, or the subject and a secondary object of an intransitive verb, into just the subject of a reciprocal verb. In this regard, it is similar to the semi-transitive verbs from §8.2.3 that make use of the reflexive pronoun \textit{asɨ-} ‘-self’. \tabref{tab:8}.23 provides a few examples of reciprocals derived from other verbs:


\begin{table}
\caption{23: Icétôd reciprocal verbs}
\label{tab:8}


\begin{tabularx}{\textwidth}{XXXX}
\lsptoprule

\multicolumn{2}{X}{Intransitive} & \multicolumn{2}{X}{Reciprocal}\\
ɓɛƙɛs & ‘to walk’ & ɓɛƙɛsɨnɔs & ‘to walk together’\\
ɨɓákɔn & ‘to be next to & ɨɓákɨnɔs & ‘to be next to each other’\\
tódòn & ‘to speak’ & tódinós & ‘to speak to each other’\\
\multicolumn{2}{X}{Transitive} & Reciprocal & \\
ɦyeés & ‘to know’ & ɦyeínós & ‘to be related’\\
ɨŋaarɛs & ‘to help’ & ɨŋáárɨnɔs & ‘to help each other’\\
mɨnɛs & ‘to love’ & mɨnɨnɔs & ‘to love each other’\\
\lspbottomrule
\end{tabularx}
\end{table}
Like the passive \{-ósí-\} discussed in §8.6.1, the reciprocal suffix can take either case endings (as a morphological noun) or subject-agreement endings (as a morphological verb). A case declension of \textit{ínínósí-} ‘to cohabitate’ is shown in \tabref{tab:8}.24, and in example (5) below, the reciprocal verb \textit{ɨɓákɨnɔsɨ-} ‘to be next to each other’ gets the accusative case. Then, example (6) shows the same verb acting as a verb proper, with the 3\textsc{pl} subject-agreement marker \{-át-\}:


\begin{table}
\caption{24: Case declension of \textit{ínínósí-} ‘to cohabitate’}
\label{tab:8}


\begin{tabularx}{\textwidth}{XXX} & Non-final & Final\\
\lsptoprule
\textsc{nom} & ínínósá & ínínós\\
\textsc{acc} & ínínósíà & ínínósíkᵃ\\
\textsc{dat} & ínínósíè & ínínósíkᵉ\\
\textsc{gen} & ínínósíè & ínínósí\\
\textsc{abl} & ínínósúò & ínínósú\\
\textsc{ins} & ínínósó & ínínósᵒ\\
\textsc{cop} & ínínósúò & ínínósúkᵒ\\
\textsc{obl} & ínínósí & ínínós\\
\lspbottomrule
\end{tabularx}
\end{table}



\ea\label{ex:}
\gll {Bɛɗáta     ɨɓák}\textit{ɨnɔsɨ}{}-kᵃ. \\
    \\
want:\textsc{3pl:real}   next.to:\textsc{recip-acc}
\glt ‘They want to be next to each other.’ 
\z




\ea\label{ex:}
\gll {Ɨɓák}\textit{ɨnɔs}{}-átᵃ. \\
    \\
next.to:\textsc{recip-3pl:real}
\glt ‘They are next to each other.’ 
\z




\subsubsection{Causative (\textsc{caus})}

Icétôd expresses causativity with a morphological causative, the \textsc{causative} suffix \{-ìt-\}. When this suffix is added to a verb with meaning X, it changes the meaning of the verb to ‘cause/make (to) X’. This suffix can be used to causativize intransitive and transitive verbs and comes right after the verb root, before the infinitive marker (if present) and any other suffixes like an inchoative or pluractional. If the last vowel of the verb root is /u/, the causative may be assimilated to \{-ùt-\}. \tabref{tab:8}.25 gives several examples of causativized verbs:


\begin{table}
\caption{25: Icétôd causative verbs}
\label{tab:8}


\begin{tabularx}{\textwidth}{XXXX}
\lsptoprule

\multicolumn{2}{X}{Intransitive} & \multicolumn{2}{X}{Causative}\\
bùkòn & ‘to be prostrate’ & bukites & ‘to lay prostrate’\\
itúrón & ‘to be proud’ & itúrútés & ‘to praise’\\
xɛɓɔn & ‘to be timid’ & xɛɓɨtɛs & ‘to intimidate’\\
\multicolumn{2}{X}{Transitive} &  & \\
dimés & ‘to refuse’ & dimités & ‘to prohibit’\\
naƙwɛs & ‘to suckle’ & naƙwɨtɛs & ‘to give suckle’\\
zízòn & ‘to be fat’ & zízités & ‘to fatten’\\
\lspbottomrule
\end{tabularx}
\end{table}



\subsection{Subject-agreement}


Whenever Icétôd grammar calls for verbs to agree with subjects, one of the seven pronominal suffixes in \tabref{tab:8}.26 are used. Just like the free pronouns described back in §5.2, these bound pronominal suffixes are organized along three axes: 1) person (1/2/3), 2) number (singular/plural), and 3) clusivity (exclusive/inclusive). The form these pronominals ultimately take depends on the grammatical mood of the verb to which they attach. If the verb is in the irrealis mood (see §8.9.1 below), the suffixes appear with their underlying forms. Whereas if they are in the realis mood (see §8.9.2 below), the realis suffix \{-a\} first subtracts or deletes their final vowel. The difference in the two mood-based paradigms is illustrated below in \tabref{tab:8}.26:


\begin{table}
\caption{26: Icétôd subject-agreement suffixes}
\label{tab:8}


\begin{tabularx}{\textwidth}{XXXXX} & \multicolumn{2}{X}{Irrealis (underlying)} & \multicolumn{2}{X}{Realis (modified)}\\
\lsptoprule
& Non-final & Final & Non-final & Final\\
\textsc{1sg} & {}-íí & {}-í & {}-íá & {}-í\\
\textsc{2sg} & {}-ídì & {}-îdⁱ & {}-ídà & {}-îdᵃ\\
\textsc{3sg} & {}-ì & {}-\textsuperscript{i} & {}-a & {}-\textsuperscript{a}\\
\textsc{1pl.exc} & {}-ímí & {}-ím & {}-ímá & {}-ím\\
\textsc{1pl.inc} & {}-ísínì & {}-ísín & {}-ísínà & {}-ísín\\
\textsc{2pl} & {}-ítí & {}-ítⁱ & {}-ítá & {}-ítᵃ\\
\textsc{3pl} & {}-átì & {}-átⁱ & {}-átà & {}-átᵃ\\
\lspbottomrule
\end{tabularx}
\end{table}

To see instances of the Icétôd subject-agreement suffixes in actual language use, you may refer back to example (11) in §7.2.




\subsection{Dummy pronoun (\textsc{dp})}


Icétôd has a special verbal affix called the \textsc{dummy pronoun} because it represents a secondary (indirect) object that has been (re)moved. That is, the dummy pronoun is a form of object-marking on the verb, but not of direct object marking. For example, if an indirect object expressing location or time or means is moved to the front of a clause for emphasis, it leaves a trace on the verb in the form of the dummy pronoun. Seen from another perspective, the dummy pronoun is always a clue that there is a missing syntactic constituent in the clause.

The dummy pronoun has the form \{-\'{} dè\} and is very volatile in terms of allomorphy, changing its form in different environments. Once the /d/ is lost in non-final forms, vowel assimilation and vowel harmony so distort the dummy pronoun as to make it almost unrecognizable at times. \tabref{tab:8}.27 below is given to illustrate its diverse allomorphy:


\begin{table}
\caption{27: Allomorphs of the dummy pronoun \{-\'{} dè\}}
\label{tab:8}


\begin{tabularx}{\textwidth}{XXX} & Non-final & Final\\
\lsptoprule
\{-\'{} dè\} & {}-\'{} è & {}-\'{} dᵉ\\
& {}-\'{} ɛ & {}-\'{} dᵋ\\
& {}-\'{} ì & \\
& {}-\'{} ɨ & \\
& {}-\'{} ò & \\
& {}-\'{} ɔ & \\
\lspbottomrule
\end{tabularx}
\end{table}
Examples (7)-(8) illustrate the dummy pronoun in two different morphological forms. Note that the tones associated with the pronoun in these examples do not match what is shown in \tabref{tab:8}.27; this is because of local tonal interference. In terms of function, the dummy pronoun in (7) indicates that an indirect object—the destination of the verb \textit{ƙáátà} ‘they go (went)’—has been displaced from its usual spot after the verb to a place of focus at the beginning of the sentence (\textit{Ntsúó}). Then in (8), the dummy pronoun marks an indirect object—the location of staying—that is missing from the clause entirely. Since this sentence was taken out of context from a story, most likely the missing object had been already mentioned earlier in the discourse:




\ea\label{ex:}
\gll {Ntsúó noo     Icéá     ƙáátà-}dᵉ. \\
    \\
it:\textsc{cop}=\textsc{past}    Ik:\textsc{acc}   go:\textsc{3pl:real-dp}
\glt ‘It’s where the Ik went (to).’ 
\z




\ea\label{ex:}
\gll {Jʼɛʝʉkɔ-}\textit{ɔ}     sàà     ròɓàᵉ. \\
    \\
stay:\textsc{3sg:seq-dp}   other:\textsc{nom}   people:\textsc{gen}
\glt ‘And other people stayed (there).’ 
\z






\subsection{Mood}
\subsubsection{Irrealis}

A basic distinction in grammatical \textsc{mood} cleaves Icétôd verbal aspects and modalities right down the center, and this distinction is between \textsc{irrealis} and \textsc{realis}. As it applies to Icétôd, the irrealis mood includes states and events whose \textit{actuality} or \textit{reality} are not expressly encoded in the grammar. Another way of saying this is that irrealis verbs in Icétôd can say anything \textit{but} whether a state or event has happened, is happening, or will happen. The morphological manifestation of the irrealis is that the final suffix of an irrealis verb—a subject-agreement pronoun—surfaces with its underlying form. 

The verbal aspects and modalities that fall under the irrealis mood include the \textsc{optative}, \textsc{subjunctive}, \textsc{imperative}, \textsc{negative}, \textsc{sequential}, and \textsc{simultaneous}, which are discussed in §8.10 below. 


\subsubsection{Realis (\textsc{real})}

In contrast to irrealis, the \textsc{realis} mood includes states and events whose actuality or reality \textit{are} encoded in the grammar. That is to say, realis verbs in Icétôd include in their meaning the fact that something has taken place, is taking place, or will take place in the real world. The morphological manifestation of the realis mood is seen in the realis suffix \{-a\} that subtracts or deletes the final vowel of the subject-agreement suffix to which it attaches (see \tabref{tab:8}.26). In terms of verb types, the realis mood includes declarative statements in the past or non-past, questions about the past or non-past, and, rather paradoxically, negative imperatives (which one would see as irrealis).




\subsection{Verb paradigms}
\subsubsection{Intentional-imperfective (\textsc{int/ipfv})}

The \textsc{intentional-imperfective} aspect suffix \{-és-\} has two basic functions, hence its hyphenated title. One function is to denote either an intention on the part of animate subjects or an imminence on the part of inanimate subjects. And it is in this role that it finds use as the usual translation for the English future tense. It is also the answer to the question, “How do you express future tense in Icétôd?” A second function is to denote grammatical imperfectivity, that is, a sense that a state or event is ongoing, incomplete. The two concepts collapse into one when intention/imminence is viewed as the incomplete coming-to-be of a future state or event.  And even though intention or imperfectivity may seem to fall under an irrealis mood, \{-és-\} can actually be used with verbs in either the realis or irrealis mood. 

In \tabref{tab:8}.28 below, \{-és-\} is illustrated with the verb \textit{àts-} ‘come’ in its imperfective sense with a recent past tense marker (\textit{nákᵃ}) and then in its intentional sense, translated with English as future tense ‘will’:


\begin{table}
\caption{28: Icétôd intentional-imperfective aspect}
\label{tab:8}


\begin{tabularx}{\textwidth}{XXX}
\lsptoprule

\multicolumn{2}{X}{Imperfective} & \\
\textsc{1sg} & Atsésíà nàkᵃ. & ‘I was coming.’\\
\textsc{2sg} & Atsésídà nàkᵃ. & ‘You were coming.’\\
\textsc{3sg} & Atsesa nákᵃ. & ‘S/he/it was coming.’\\
\textsc{1pl.exc} & Atsésímà nàkᵃ & ‘We were coming.’\\
\textsc{1pl.inc} & Atsésísìnà nàkᵃ. & ‘We all were coming.’\\
\textsc{2pl} & Atsésítà nàkᵃ. & ‘You all were coming.’\\
\textsc{3pl} & Atsésátà nàkᵃ. & ‘They were all coming.’\\
\multicolumn{2}{X}{Intentional} & \\
\textsc{1sg} & Atsésí. & ‘I will come.’\\
\textsc{2sg} & Atsésîdᵃ. & ‘You will come.’\\
\textsc{3sg} & Atsés. & ‘S/he/it will come.’\\
\textsc{1pl.exc} & Atsésím. & ‘We will come.’\\
\textsc{1pl.inc} & Atsésísìn. & ‘We all will come.’\\
\textsc{2pl} & Atsésítᵃ. & ‘You all will come.’\\
\textsc{3pl} & Atsésátᵃ. & ‘They all will come.’\\
\lspbottomrule
\end{tabularx}
\end{table}

\subsubsection{Present perfect (\textsc{prf})}

The Icétôd \textsc{present perfect} aspect suffix \{-\'{} ka\} denotes a state or event recently completed (‘perfected’) but still relevant in the present. The suffix has a ‘floating’ high tone that shows up on the preceding syllable of 3\textsc{sg} verbs, for example in \textit{Nabʉƙɔtákᵃ} ‘It is finished’. And the /k/ in \{-\'{} ka\} disappears in non-final environments, making \{-\'{} a\} an allomorph. \tabref{tab:8}.29 presents the paradigm of the present perfect with the verb \textit{àts-} ‘come’ in both non-final and final environments:


\begin{table}
\caption{29: Icétôd present perfect aspect}
\label{tab:8}


\begin{tabularx}{\textwidth}{XXXXX}
\lsptoprule

\multicolumn{2}{X}{} & Non-final & Final & \\
\textsc{1sg} & \multicolumn{2}{X}{Atsíaà...} & Atsíàkᵃ. & ‘I have come’\\
\textsc{2sg} & \multicolumn{2}{X}{Atsídàà...} & Atsídàkᵃ. & ‘You have come’\\
\textsc{3sg} & \multicolumn{2}{X}{Atsáá...} & Atsákᵃ. & ‘She has come’\\
\textsc{1pl.exc} & \multicolumn{2}{X}{Atsímáà...} & Atsímákᵃ. & ‘We have come’\\
\textsc{1pl.inc} & \multicolumn{2}{X}{Atsísínàà...} & Atsísínàkᵃ. & ‘We all have come’\\
\textsc{2pl} & \multicolumn{2}{X}{Atsítáà...} & Atsítákᵃ. & ‘You all have come’\\
\textsc{3pl} & \multicolumn{2}{X}{Atsátàà...} & Atsátàkᵃ. & ‘They have come’\\
\lspbottomrule
\end{tabularx}
\end{table}

\subsubsection{Optative (\textsc{opt})}

The Icétôd \textsc{optative} mood is used to express wishes, even sarcastic ones like ‘Let the enemies comeǃ’. Optative verbs are often introduced with supporting imperative verbs like \textit{\'{O}goe} or \textit{Taláké}, both of which mean ‘Let...’. And all Icétôd optative verbs are translated into English with a sentence beginning with ‘Let...’ or ‘May...’. 

Morphologically, the optative is marked by a combination of tone and special irregular suffixes. All optative verbs except 3\textsc{pl} show a kind of high-tone ‘leveling’ in the subject-agreement suffixes. The leveled high tone is pushed out to the end, creating a floating high tone. This high tone is not seen except in the fact that the last syllable of the subject-agreement suffixes remains at mid-tone level (instead of low). In addition to tone, suffixes mark the optative. Special irregular ones are used for \textsc{1sg}, 1\textsc{pl.exc}, and 1\textsc{pl.inc}, while standard irrealis ones are used for the other members of the paradigm. Note that the 1\textsc{pl.inc} may also be called the ‘hortative’. Another peculiarity of the Icétôd optative is that there is no difference between its non-final and final forms. \tabref{tab:8}.30 presents the optative on the verb \textit{àts-} ‘come’:


\begin{table}
\caption{30: Icétôd optative mood}
\label{tab:8}


\begin{tabularx}{\textwidth}{XXX}
\lsptoprule

\textsc{1sg} & Atsine. & ‘Let me come.’\\
\textsc{2sg} & Atsidi. & ‘May you come.’\\
\textsc{3sg} & Atsi. & ‘Let her come.’\\
\textsc{1pl.exc} & Atsima. & ‘Let us come.’\\
\textsc{1pl.inc} & Atsano. & ‘Let us all come.’\\
\textsc{2pl} & Atsiti. & ‘May you all come.’\\
\textsc{3pl} & Atsáti. & ‘Let them come.’\\
\lspbottomrule
\end{tabularx}
\end{table}

\subsubsection{Subjunctive (\textsc{subj})}

The Icétôd \textsc{subjunctive} mood is used to encode statements that are contingent or temporally unrealized. In that regard, it is an essentially irrealis verb form because it captures states or events that have not yet happened. It is also essentially irrealis in that it is marked simply by the absence of any marking. In other words, the subject-agreement suffixes surface with their underlying forms in the subjunctive mood, just as they appear in \tabref{tab:8}.26 above. The subjunctive is usually introduced either by \textit{ɗɛmʉsʉ} ‘unless, until’ or \textit{damu (koʝa)} ‘may’. \tabref{tab:8}.31 gives the full subjunctive paradigm with \textit{àts-} ‘come’:


\begin{table}
\caption{31: Icétôd subjunctive mood}
\label{tab:8}


\begin{tabularx}{\textwidth}{XXXX} & Non-final & Final & \\
\lsptoprule
\textsc{1sg} & ɗɛmʉsʉ atsíí... & ɗɛmʉsʉ atsí. & ‘unless I come’\\
\textsc{2sg} & ɗɛmʉsʉ atsídì... & ɗɛmʉsʉ atsîdⁱ. & ‘unless you come’\\
\textsc{3sg} & ɗɛmʉsʉ atsi... & ɗɛmʉsʉ atsⁱ. & ‘unless she comes’\\
\textsc{1pl.exc} & ɗɛmʉsʉ atsímí... & ɗɛmʉsʉ atsím. & ‘unless we come’\\
\textsc{1pl.inc} & ɗɛmʉsʉ atsísínì... & ɗɛmʉsʉ atsísín. & ‘unless we all come’\\
\textsc{2pl} & ɗɛmʉsʉ atsítí... & ɗɛmʉsʉ atsítⁱ. & ‘unless you all come’\\
\textsc{3pl} & ɗɛmʉsʉ atsátì... & ɗɛmʉsʉ atsátⁱ. & ‘unless they come’\\
\lspbottomrule
\end{tabularx}
\end{table}

\subsubsection{Imperative (\textsc{imp})}

The \textsc{imperative} mood is used to issue commands or instructions. If the recipient of the command is singular, then the suffix used is \{-e\'{} \}, and if the recipient is plural, the suffix is \{-úó\}. The singular \{-e\'{} \} has a floating high tone that raises any preceding low tones to mid. Both imperative suffixes are appended to the end of the verb stem, and no subject-agreement markers are needed. Both imperative suffixes are subject to vowel devoicing before a pause, as shown in \tabref{tab:8}.32:


\begin{table}
\caption{32: Icétôd imperative mood}
\label{tab:8}


\begin{tabularx}{\textwidth}{XXXXXX}
\lsptoprule

Singular &  &  & Plural &  & \\
\textsc{nf} & \textsc{ff} &  & \textsc{nf} & \textsc{ff} & \\
Atse..ǃ & Atsᵉǃ & ‘Comeǃ’ & Atsúó..ǃ & Atsúǃ & ‘Comeǃ’\\
Ƙae..ǃ & Ƙaᵉǃ & ‘Goǃ’ & Ƙoyúó..ǃ & Ƙoyúǃ & ‘Goǃ’\\
Ŋƙɛ..ǃ & Ŋƙᵋǃ & ‘Eatǃ’ & Ŋƙʉɔ..ǃ & Ŋƙʉǃ & ‘Eatǃ’\\
Zɛƙwɛ..ǃ & Zɛƙwᵋǃ & ‘Sitǃ’ & Zɛƙʉɔ..ǃ & Zɛƙʉǃ & ‘Sitǃ’\\
\lspbottomrule
\end{tabularx}
\end{table}

\subsubsection{Negative}

Icétôd negates clauses by means of verblike particles that come first in the negative clause. If the negated clause has a realis verb, then the negator particle used is \textit{ńtá} ‘not’. If the negated clause has an irrealis verb like the sequential (see §8.10.7), then the negator particle is \textit{mòò} or \textit{nòò}. Lastly, if the negated clause is past tense realis or present perfect realis, then the negator particle used is \textit{máà} or \textit{náà}. In the negated clause, the negator particle comes first, followed by the subject, followed by the verb. Any negated verb takes the irrealis mood with the appropriate form of subject-agreement suffixes (see \tabref{tab:8}.26). To make all this more concrete, \tabref{tab:8}.33 gives example of the different negator particles used with different types of clauses:


\begin{table}
\caption{33: Icétôd negative mood}
\label{tab:8}


\begin{tabularx}{\textwidth}{XXX}
\lsptoprule

Realis &  & \\
\textsc{1sg} & \'{N}tá ɦyeí. & ‘I don’t know.’\\
\textsc{2sg} & \'{N}tá ɦyeîdⁱ. & ‘You don’t know.’\\
\textsc{3sg} & \'{N}tá ɦyèⁱ. & ‘She doesn’t know.’\\
Sequential &  & \\
\textsc{1sg} & ...moo ɦyeí. & ‘...and I don’t know.’\\
\textsc{2sg} & ...moo ɦyeîdⁱ. & ‘...and you don’t know.’\\
\textsc{3sg} & ...mòò ɦyèⁱ. & ‘...and she doesn’t know.’\\
Past realis &  & \\
\textsc{1sg} & Máa naa ɦyeí. & ‘I didn’t know.’\\
\textsc{2sg} & Máa naa ɦyeîdⁱ. & ‘You didn’t know.’\\
\textsc{3sg} & Máà nàà ɦyèⁱ. & ‘She didn’t know.’\\
\lspbottomrule
\end{tabularx}
\end{table}

\subsubsection{Sequential (\textsc{seq})}

The Icétôd \textsc{sequential} aspect expresses states or events that happen in sequence. Usually a sequence of verbs starts with an anchoring non-sequential verb, and then a \textsc{clause} \textsc{chain} begins in the sequential aspect. For example, when someone tells a story, they may start with one or two past tense realis verbs to set the stage and then continue the narrative with sequential verbs. Or if someone is giving a set of instructions, they may start with one or two imperative verbs followed by a chain of sequential verbs. Because of its versatility, the Icétôd sequential aspect is the language’s most frequently used verb form.

Morphologically, Icétôd sequential verbs are recognized by a combination of tone, irregular subject-agreement suffixes, and the sequential aspect suffix \{-ko\}. Specifically, all 1\textsuperscript{st} and 2\textsuperscript{nd}{}-person sequential verbs exhibit high-tone leveling in their subject-agreement suffixes, which pushes a high tone out to the right of the verb. This floating high raises the preceding low tones to mid. These tone effects, plus the irregular suffixes, and the sequential marker \{-ko\} are shown in \tabref{tab:8}.34. Note that the sequential paradigm also has an impersonal passive marked with the suffix \{-ese\}. Its function is identical to that of the impersonal passive described back in §8.6.2.


\begin{table}
\caption{34: Icétôd sequential aspect}
\label{tab:8}


\begin{tabularx}{\textwidth}{XXXX} & Non-final & Final & \\
\lsptoprule
\textsc{1sg} & ...atsiaa... & ...atsiakᵒ. & ‘and I come’\\
\textsc{2sg} & ...atsiduo... & ...atsidukᵒ. & ‘and you come’\\
\textsc{3sg} & ...àtsùò... & ...àtsùkᵒ. & ‘and she comes’\\
\textsc{1pl.exc} & ...atsimaa... & ...atsimakᵒ. & ‘and we come’\\
\textsc{1pl.inc} & ...atsisinuo... & ...atsisinukᵒ. & ‘and we all come’\\
\textsc{2pl} & ...atsituo... & ...atsitukᵒ. & ‘and you all come’\\
\textsc{3pl} & ...àtsìnì... & ...àtsìn. & ‘and they come’\\
\textsc{pass} & ...atsese... & ...atses. & ‘and it was come’\\
\lspbottomrule
\end{tabularx}
\end{table}
For more on how the sequential aspect works in actual language contexts, skip ahead to the discussion of clause-chaining in §10.8.2.


\subsubsection{Simultaneous (\textsc{sim})}

The Icétôd \textsc{simultaneous} aspect is used to express states or events that are happening simultaneously to another state or event. In contrast to the sequential, the simultaneous aspect can only be used in subordinate clauses. That is to say, simultaneous clauses usually cannot stand alone without a main clause (with some exceptions). Because of its role of supporting sequential clauses, the simultaneous aspect is also commonly found in narratives and other longer discourses. It can be given a perfective interpretation as in ‘when I came’ or an imperfective one as in ‘while I was coming’.

Morphologically, the simultaneous aspect is marked by the suffix \{-ke\}, which is affixed to the subject-agreement suffixes in their irrealis forms. \tabref{tab:8}.35 presents the simultaneous paradigm of \textit{àts-} ‘come’:


\begin{table}
\caption{35: Icétôd simultaneous aspect}
\label{tab:8}


\begin{tabularx}{\textwidth}{XXXX} & Non-final & Final & \\
\lsptoprule
\textsc{1sg} & ...atsííkè... & ...atsííkᵉ. & ‘while I was coming’\\
\textsc{2sg} & ...atsídìè... & ...atsídìkᵉ. & ‘while you were coming’\\
\textsc{3sg} & ...àtsìè... & ...àtsìkᵉ. & ‘while she was coming’\\
\textsc{1pl.exc} & ...atsímíè... & ...atsímíkᵉ. & ‘while we were coming’\\
\textsc{1pl.inc} & ...atsísínìè... & ...atsísínìkᵉ. & ‘while we all were coming’\\
\textsc{2pl} & ...atsítíè... & ...atsítíkᵉ. & ‘while you all were coming’\\
\textsc{3pl} & ...atsátìè... & ...atsátìkᵉ. & ‘while they were coming’\\
\lspbottomrule
\end{tabularx}
\end{table}



\subsection{Adjectival verbs}
\subsubsection{Overview}

Since Icétôd does not have a separate word class of adjectives, it conveys adjectival concepts with \textsc{adjectival verbs}. These verbs have adjectival meanings but otherwise mostly behave like intransitive verbs. One way they do differ from normal intransitive verbs, though, is in the specific adjectival suffixes they can take. The next four subsections offer brief descriptions of these special adjectival suffixes.


\subsubsection{Physical property I (\textsc{phys1})}

The \textsc{physical property i} adjectival suffix \{- d-\} is found on adjectival verbs that express physical properties like appearance, size, shape, consistency, texture, and other tangible attributes. As a result, Physical Property I verbs are some of the language’s most colorful adjectivals. Physical Property I verbs all contain two syllable with LH tone pattern, and in the infinitive, they take the intransitive suffix \{-ònì-\}. \tabref{tab:8}.36 gives a sample of these colorful descriptive terms:


\begin{table}
\caption{36: Icétôd Physical Property I adjectival verbs}
\label{tab:8}


\begin{tabularx}{\textwidth}{XX}
\lsptoprule

bufúdòn & ‘to be spongy’\\
ɗɔmɔdɔn & ‘to be gluey’\\
dirídòn & ‘to be compacted’\\
ʝamúdòn & ‘to be velvety’\\
lɛtsʼɛdɔn & ‘to be bendy’\\
pɨɗɨdɔn & ‘to be sleek’\\
tsakádòn & ‘to be watery’\\
\lspbottomrule
\end{tabularx}
\end{table}

\subsubsection{Physical property II (\textsc{phys2})}

The \textsc{physical property ii} adjectival suffix \{-m-\} is found in adjectival verbs that also express physical properties like appearance, color, consistency, posture, shape, and texture. It can also express less physical attributes like strength, weakness, or personality traits. Physical Property II verbs usually contain two syllables with a LH tone pattern or three syllables with a LHH tone pattern, and in the infinitive, they take the intransitive suffix \{-ònì\}. \tabref{tab:8}.37 gives a sample of these descriptive adjectival verbs:


\begin{table}
\caption{37: Icétôd Physical Property II adjectival verbs}
\label{tab:8}


\begin{tabularx}{\textwidth}{XX}
\lsptoprule

Bisyllabic & \\
buɗámón & ‘to be black’\\
dʉgʉmɔn & ‘to be hunched’\\
firímón & ‘to be clogged’\\
kikímón & ‘to be stocky’\\
kwɛtsʼɛmɔn & ‘to be damaged’\\
Trisyllabic & \\
bulúƙúmòn & ‘to be bulbous’\\
ʝʉrʉtʉmɔn & ‘to be slippery\\
pelérémòn & ‘to be squinty’\\
ságwàràmòn & ‘to be shadeless’\\
tɛƙɛzɛmɔn & ‘to be shallow\\
\lspbottomrule
\end{tabularx}
\end{table}

\subsubsection{Stative (\textsc{stat})}

The \textsc{stative} adjectival suffix \{-án-\} forms adjectival verbs that express an ongoing state characterized by the meaning of a noun or a transitive verb. Because \{-án-\} contains the vowel /a/, it prevents vowel harmony from spreading between the verbal root and any suffixes that follow the stative suffix (for example, infinitive or subject-agreement suffixes). \tabref{tab:8}.38 presents a few examples of stative adjectival verbs derived from noun roots:


\begin{table}
\caption{38: Icétôd stative verbs derived from nouns}
\label{tab:8}


\begin{tabularx}{\textwidth}{XXXX}
\lsptoprule

Noun &  & Stative verb & \\
cué- & ‘water’ & cuanón & ‘to be liquid’\\
ɛsá- & ‘drunkenness’ & ɛsánón & ‘to be drunk’\\
kirotí- & ‘sweat’ & kirotánón & ‘to be sweaty’\\
ɲɛƙɛ{}- & ‘hunger’ & ɲɛƙánón & ‘to be hungry’\\
ɲèrà- & ‘girls’ & iɲéráánón & ‘to be girl-crazy’\\
\lspbottomrule
\end{tabularx}
\end{table}

\tabref{tab:8.39} gives a few stative verbs derived from transitive verbs:
 


\begin{table}
\caption{Icétôd stative verbs derived from transitive verbs}
\label{tab:8.39}


\begin{tabularx}{\textwidth}{XXXX}
\lsptoprule

Transitive &  & Stative & \\
ɓɛkɛs & ‘to provoke’ & ɓɛkánón & ‘to be provocative’\\
dzɛrɛs & ‘to tear’ & dzɛrɛdzɛránón & ‘to be torn in shreds’\\
itáléés & ‘to forbid’ & itáléánón & ‘to be forbidden’\\
itukes & ‘to heap’ & itukánón & ‘to be congregated’\\
ɨraŋɛs & ‘to spoil’ & ɨráŋʉnánón & ‘to be spoiled’\\
\lspbottomrule
\end{tabularx}
\end{table}

\subsubsection{Distributive (\textsc{distr})}

Icétôd has two \textsc{distributive} adjectival suffixes: \{-aák-\} and \{-ìk-\}. These suffixes have the function of distributing the meaning of the adjectival verb to more than one subject. The suffix \{-aák-\} can be used with all kinds of adjectival verbs, including the physical property and stative varieties, while the suffix \{-ìk-\} has been found only with the two verbs of size, \textit{kwáts-} ‘small’ and \textit{zè-} ‘large’. Moreover, it commonly occurs together with \{-aák-\}, as in \textit{kwátsíkaakón} ‘to be small (of many)’ and \textit{zeikaakón} ‘to be large (of many)’. \tabref{tab:8}.40 gives a sampling of adjectival verbs with the distributive suffix:


\begin{table}
\caption{40: Icétôd distributive adjectival verbs}
\label{tab:8}


\begin{tabularx}{\textwidth}{XX}
\lsptoprule

buɗúdaakón & ‘to be soft (of many)’\\
ɓetsʼaakón & ‘to be white (of many)’\\
gaanaakón & ‘to be bad (of many)’\\
kúɗaakón & ‘to be short (of many)’\\
maráŋaakón & ‘to be good (of many)’\\
nɔtsɔdaakón & ‘to be adhesive (of many)’\\
semélémaakón & ‘to be elliptical (of many)’\\
\lspbottomrule
\end{tabularx}
\end{table}

\section{Adverbs}



\subsection{Overview}


The word class called \textsc{adverbs} is a catch-all category that includes words and clitics of various sorts that say something about a whole clause, for example, ‘how’ or ‘when’ it takes place, or how the speaker feels about the certainty or contingency of the clause. Accordingly, Icétôd adverbs can be divide up into \textsc{manner} adverbs, \textsc{temporal} adverbs, and \textsc{epistemic} adverbs. The following subsections take up each of these adverbial categories in a brief discussion.




\subsection{Manner adverbs}


\textsc{manner} adverbs modify whole clauses by commenting on the manner in which a state comes across or in which an action is done. Manner adverbs come near or at the end of the clause they modify, as shown in (1)-(2) below. \tabref{tab:9}.1 presents a sampling of these adverbs:


\begin{table}
\caption{1: Icétôd manner adverbs}
\label{tab:9}


\begin{tabularx}{\textwidth}{XX}
\lsptoprule

ɗɛmʉsʉ & ‘fast, quickly’\\
hɨɨʝɔ & ‘carefully, slowly’\\
ʝíìkì & ‘always’\\
ʝɨkɨ & ‘really, totally’\\
kɔntɨákᵉ & ‘straightaway’\\
mʉkà & ‘completely, forever’\\
pákà & ‘indefinitely’\\
zùkù & ‘very’\\
\lspbottomrule
\end{tabularx}
\end{table}



\ea\label{ex:}
\gll {Gaana   mɛna díí}     \textit{zuku}\textit{}   ʝɨkᶤ. \\
    \\
bad:\textsc{3sg}   issues:\textsc{nom}=those   \textsc{adv   adv}
\glt ‘Those issues are really bad!’ 
\z




\ea\label{ex:}
\gll {Zízaaƙótùò       ròɓà}     mʉkà. \\
    \\
fat:\textsc{dist:comp:3sg:seq}   people:\textsc{nom}   \textsc{adv}
\glt ‘And the people fattened up completely!’ 
\z






\subsection{Temporal adverbs}
\subsubsection{Overview}

The Icétôd \textsc{temporal} adverbs situate their clause somewhere in the course of time. Icétôd has sets of temporal adverbs that deal with past tense, past perfect tense, and non-past (including future) tense. The past and past perfect tense adverbs are enclitics that come directly after the verb they modify. The future tense adverbs are free adverbs that come near the end or at the end of the clause they modify. 


\subsubsection{Past tense adverbs (\textsc{pst})}

Icétôd divides \textsc{past tense} into four time periods and marks them with special adverbial enclitics. They are: 1) \textsc{recent past} that covers the current day and is marked with =\textit{nákà}, 2) \textsc{removed past} that covers yesterday (or any last or ‘yester-’ time period) and is marked with =\textit{bàtsè}, 3) \textsc{remote past} that covers a few days or weeks before yesterday and is marked with =\textit{nótsò}, and finally, 4) \textsc{remotest past} that covers everything before the remote past and is marked with =\textit{nòkò}. Each of these tense enclitics has a non-final and final form, and as enclitics, they always come directly after the verb in a clause. \tabref{tab:9}.2 illustrates the Icétôd tense markers in all their forms, and examples (3)-(4) illustrate their position in a sentence:


\begin{table}
\caption{2: Icétôd past tense markers}
\label{tab:9}


\begin{tabularx}{\textwidth}{XXXX} & \textsc{nf} & \textsc{ff} & \\
\lsptoprule
Recent & =náà & =nákᵃ & ‘earlier today’\\
Removed & =bèè & =bàtsᵉ & ‘yester-’\\
Remote & =nótsò & =nótsò & ‘a while ago’\\
Remotest & =nòò & =nòkᵒ & ‘long ago’\\
\lspbottomrule
\end{tabularx}
\end{table}



\ea\label{ex:}
\gll {Ƙ}\textit{aá} \textit{bee}   abáŋa     sáásɔsɨǹ. \\
    \\
go:\textsc{3sg}=\textsc{pst}   my.father:\textsc{nom}   yesterday
\glt ‘My father went yesterday.’ 
\z




\ea\label{ex:}
\gll {Maráŋa} \textit{noo}   ɦyekesa   Icé. \\
    \\
good:\textsc{3sg}=\textsc{pst}   life:\textsc{nom}   Ik:\textsc{gen}
\glt ‘The life of the Ik was good (back then).’ 
\z




\subsubsection{Past perfect tense adverbs (\textsc{pst.prf})}

The past tense can be combined with a perfect aspect to yield the \textsc{past perfect} tense. Unlike the simple past tense adverbs, Icétôd past perfect tense adverbs operate along only three periods of time: \textsc{recent} (earlier today), \textsc{removed} (yester-), and \textsc{remote} (before yester-). \tabref{tab:9}.3 presents the Icétôd past perfect tense adverbs, and example sentences (5)-(6) illustrate their use in natural language situations:


\begin{table}
\caption{3: Icétôd past perfect tense markers}
\label{tab:9}


\begin{tabularx}{\textwidth}{XXXX} & \textsc{nf} & \textsc{ff} & \\
\lsptoprule
Recent & =nanáà & =nanákᵃ & ‘had...earlier today’\\
Removed & =nàtsàmʉ & =nàtsàm & ‘had...yester-’\\
Remote & =nànòò & =nànòkᵒ & ‘had...a while ago’\\
\lspbottomrule
\end{tabularx}
\end{table}



\ea\label{ex:}
\gll {Náa   atsíâdᵉ,     ƙaa}     nanákᵃ. \\
    \\
when  come:\textsc{1sg:dp}   go:\textsc{3sg}   \textsc{pst.prf}
\glt ‘When I came earlier, she had (already) gone.’ 
\z




\ea\label{ex:}
\gll {Tsʼɛdɔɔ   nɛ,   tsʼéíƙotátà}     nànòkᵒ. \\
    \\
then:\textsc{ins}   that   die.\textsc{3pl:comp}   \textsc{pst.prf}
\glt ‘By that (time), they had died out a while ago.’ 
\z




\subsubsection{Non-past tense adverbs}

Icétôd divides the \textsc{non-past} tense into three rather vaguely defined time periods suggested by three adverbs. They are: 1) the \textsc{distended} \textsc{present} that includes just before and just after the present and is expressed by the adverb \textit{tsʼɔɔ}, 2) the \textsc{removed future} that includes the \textit{next} future time period (next hour, next day, next year) and is expressed by the adverb \textit{táà}, and 3) the \textsc{remote future} expressed by the adverb \textit{fàrà} (occasionally \textit{fàrò}). \tabref{tab:9}.4 arranges these adverbs in a paradigm, while (7)-(8) below illustrates them in real sentences:


\begin{table}
\caption{4: Icétôd non-past tense markers}
\label{tab:9}


\begin{tabularx}{\textwidth}{XXXX} & \textsc{nf} & \textsc{ff} & \\
\lsptoprule
Distended present & tsʼɔɔ & tsʼɔɔ & ‘just/recently/soon’\\
Removed & táà & táà & ‘next\_\_\_\_’\\
Remote & fàrà & fàr & ‘in the future’\\
\lspbottomrule
\end{tabularx}
\end{table}



\ea\label{ex:}
\gll {Atsíá nàà}     \textit{tsʼɔɔ}\textit{.    Atsésíà}     tsʼɔɔ. \\
    \\
come:\textsc{1sg=past}   just    come:\textsc{int:1sg}   soon
\glt ‘I just came.’      ‘I will come soon.’ 
\z




\ea\label{ex:}
\gll {Atsésíma}     \textit{táa}   baratsᵒ. \\
    \\
come:\textsc{1pl.exc}   next   morning:\textsc{ins}
\glt ‘We will come tomorrow (i.e., next morning).’ 
\z






\subsection{Epistemic adverbs}
\subsubsection{Overview}

The Icétôd \textsc{epistemic} adverbs express how the speaker feels or thinks about the certainty or contingency of the clause. Accordingly, this set of adverbs can be divided into the categories of \textsc{inferential}, \textsc{confirmational}, and \textsc{conditional-hypothetical}. All of the epistemic adverbs are enclitics that follow the verb in normal main clauses, but some of them can also be moved in front of the verb.


\subsubsection{Inferential adverbs (\textsc{infr})}

Icétôd can communicate a degree of uncertainty about a situation by means of a set of \textsc{inferential} tense-based adverbs. This sense of making a tentative inference based on an observation can be translated into English with such turns of phrase as ‘Apparently...’, ‘Maybe...’, ‘It seems that...’, ‘must have’, etc. Two of these inferential particles consist of the proclitic \textit{ná} plus a past-tense particle, while the third combines \textit{ná} with the adverb \textit{tsamʉ}. \tabref{tab:9}.5 presents the three inferential adverbial particles in their final and non-final forms. Note that compared to the past-tense markers above in \tabref{tab:9}.2, the inferential time-scale is moved up one notch more recent. 


\begin{table}
\caption{5: Icétôd inferential adverbs}
\label{tab:9}


\begin{tabularx}{\textwidth}{XXXX} & \textsc{nf} & \textsc{ff} & \\
\lsptoprule
Recent & nábèè & nábàtsᵉ & ‘apparently earlier today’\\
Removed & nátsàmʉ & nátsàm & ‘apparently yester-’\\
Remote & nánòò & nánòkᵒ & ‘apparently long ago’\\
\lspbottomrule
\end{tabularx}
\end{table}
Examples (9)-(10) show the Icétôd inferential adverbs in context. Note that they can be placed before or after the main verb, as in (9):




\ea\label{ex:}
\gll {Baduƙota}     \textit{nábàtsᵉ}.  \textit{Nábee}    baduƙotᵃ. \\
    \\
die:\textsc{comp:3sg   infr}    \textsc{infr}     die:\textsc{comp:3sg}
\glt ‘It died, apparently.’    ‘Apparently, it died.’ 
\z




\ea\label{ex:}
\gll {Nanoo}   teremátᵃ. \\
    \\
\textsc{infr}     separate:\textsc{3pl}
\glt ‘It looks like they separated.’ 
\z




\subsubsection{Confirmational adverbs (\textsc{conf})}

Icétôd can also issue a confirmation of a state or event by means of a set of \textsc{confirmational} adverbs that are derived from the tensed relative pronouns described back in §5.7. When these particles are used, they are placed before the verb, and the verb surfaces in its non-final form, almost like a question rendered in English ‘Why yes, did X \textit{not} happen?’—meaning that, of course, it \textit{did} happen. The confirmational suffixes first presented in \tabref{tab:9}.6 and then demonstrated below in example sentences (11)-(12):


\begin{table}
\caption{6: Icétôd confirmational markers}
\label{tab:9}


\begin{tabularx}{\textwidth}{XXX}
\lsptoprule

Recent & náa & ‘Of course\_\_\_\_earlier today.’\\
Removed & sɨna & ‘Of course\_\_\_\_yester-.’\\
Remote & noo & ‘Of course\_\_\_\_long ago.’\\
\lspbottomrule
\end{tabularx}
\end{table}



\ea\label{ex:}
\gll {Ŋ}\textit{ƙáƙóídà bèè?}    \textit{Sɨna}   ŋƙáƙótíà. \\
    \\
eat:\textsc{comp:2sg=pst}    \textsc{conf}   eat\textsc{:comp:1sg}
\glt ‘Did you eat (it) up?  ‘Yes, of course I did.’ 
\z




\ea\label{ex:}
\gll {Dètà nòò?}      \textit{Nòò}   dètà. \\
    \\
bring:\textsc{3sg=pst}    \textsc{conf}   bring:\textsc{3sg}
\glt ‘Did she bring (it)?’    ‘Yes, of course she did.’ 
\z




\subsubsection{Conditional-hypothetical adverbs (\textsc{cond}/\textsc{hypo})}

If a state or event has not taken place but \textit{could} or \textit{would} take place, Icétôd can express that contingency with its \textsc{conditional-hypothetical} adverbs. There are three of these adverbs, but they are used to cover four periods of time. The first adverb covers non-past and recent past, the second removed past, and third remote past. These conditional-hypothetical adverbs are presented below in \tabref{tab:9}.7:


\begin{table}
\caption{7: Icétôd conditional-hypothetical adverbs}
\label{tab:9}


\begin{tabularx}{\textwidth}{XXXX} & \textsc{nf} & \textsc{ff} & \\
\lsptoprule
Non-past & ƙánàà & ƙánàkᵃ & ‘would’\\
Recent & ƙánàà & ƙánàkᵃ & ‘would have...earlier today’\\
Removed & ƙásàmʉ & ƙásàm & ‘would have...yester-’\\
Remote & ƙánòò & ƙánòkᵒ & ‘would have...a while ago’\\
\lspbottomrule
\end{tabularx}
\end{table}
The conditional-hypothetical adverbs come after the main verb:




\ea\label{ex:}
\gll {Tóída}     \textit{ƙánaa}   ɲcìè? \\
    \\
tell:\textsc{2sg}   \textsc{hypo}     I:\textsc{dat}
\glt ‘You would tell me?’ 
\z




\ea\label{ex:}
\gll {Cɛmɨsɨna}   ƙánòkᵒ. \\
    \\
fight:\textsc{1pl.inc cond}
\glt ‘We all would have fought.’ 
\z




\section{Basic syntax}



\subsection{Noun phrases}


The Icétôd \textsc{noun phrase} consists first and foremost of a noun ‘head’, either a lexical noun or a nominalized lexical verb. As a head-initial language, Icétôd places its noun phrase head first in the phrase. Any subordinate, supporting elements follow the head. These optional elements may include anaphoric demonstratives, possessive markers, relative pronouns/temporal demonstratives, number markers, and spatial demonstratives. The Icétôd noun phrase structure can be formalized as follows, where elements in parentheses are optional:




Icétôd NP structure: 

\textsc{head (anph)(poss)(num)(rel/temp) (dem)}


The syntactical structure formalized in (1) is fleshed out among the real Icétôd noun phrases presented below in examples (2)-(10):



\textsc{head}
\ea\label{ex:}
\gll {wikᵃ} \\
    \\
\glt ‘children’ 
\z




\textsc{head anph}
\ea\label{ex:}
\gll {wika díí} \\
    \\
\glt ‘those (specific) children’ 
\z




\textsc{head poss}
\ea\label{ex:}
\gll {wika ɲcì} \\
    \\
\glt ‘my children’ 
\z




\textsc{head anph poss}
\ea\label{ex:}
\gll {wika díí ɲcì} \\
    \\
\glt ‘those (specific) children of mine’ 
\z




\textsc{head anph poss num}
\ea\label{ex:}
\gll {wika díí ɲcìè lèɓètsè} \\
    \\
\glt ‘those two (specific) children of mine’ 
\z




\textsc{head anph poss rel}
\ea\label{ex:}
\gll {wika díí ɲcie [ni leɓetse]}\textsc{\textsubscript{rel}}  \\
    \\
\glt ‘those (specific) children of mine, two in number’ 
\z




\textsc{head anph poss num rel}
\ea\label{ex:}
\gll {wika díí ɲcie leɓetse [ní dà]}\textsc{\textsubscript{ rel}} \\
    \\
\glt ‘those two nice (specific) children of mine’ 
\z




\textsc{head anph poss num rel dem}
\ea\label{ex:}
\gll {wika díí ɲcie leɓetse [ní daa]}\textsc{\textsubscript{ rel}} ni \\
    \\
\glt ‘those two nice (specific) children of mine, these’ 
\z




\textsc{head anph poss num temp dem}
\ea\label{ex:}
\gll {wika díí ɲcie leɓetse níi ni} \\
    \\
\glt ‘those two (specific) children of mine from earlier, these’ 
\z






\subsection{Clause structure}
\subsubsection{Intransitive}

Icétôd \textsc{intransitive} clauses consist minimally of a verb (\textsc{v}) and a subject (\textsc{s}) in a \textsc{vs} constituent order. The subject may be explicit, in which case it follows the verb, or it may be merely marked on the verb. Basic intransitive clause structure is illustrated in example (11):




\ea\label{ex:}
\gll {Epa}\textsc{\textsubscript{v}}\textsc{}    \textit{ŋókᵃ}\textsc{\textsubscript{s}}. \\
    \\
sleep:\textsc{3sg}   dog:\textsc{nom}
\glt ‘The dog sleeps.’ 
\z


When a tense adverb is needed, it comes directly after the verb and before any explicit subject. And any other adverbial elements like extended objects (\textsc{e}) or adverbs, in that order, come after the subject. This elaborated intransitive clause structure is illustrated in (12):



\ea\label{ex:}
\gll {Epá}\textsc{\textsubscript{v}} \textit{bee}\textsc{\textsubscript{tense}}\textit{   ŋóká}\textsc{\textsubscript{s}}\textit{     kurú}\textsc{\textsubscript{e}}. \\
    \\
sleep:\textsc{3sg}=yester-   dog:\textsc{nom}   shade:\textsc{abl}
\glt ‘The dog slept in the shade yesterday.’ 
\z




\subsubsection{Transitive}

Icétôd \textsc{transitive} clauses consist minimally of a transitive verb (\textsc{v}), an agent (\textsc{a}), and an object (\textsc{o}) in a \textsc{vao} constituent order. The subject may be explicit, in which case it comes between the verb and object, or it may merely be marked on the verb with a suffix. The object may also be dropped, in which case it is inferred from the context. Example (13) below illustrates basic transitive clause structure:




\ea\label{ex:}
\gll {Átsʼá}\textsc{\textsubscript{v}}    \textit{ŋóká}\textsc{\textsubscript{a}}\textit{    ɔkákᵃ}\textsc{\textsubscript{o}}\textsc{.} \\
    \\
gnaw:\textsc{3sg}  dog:\textsc{nom}  bone:\textsc{acc}
\glt ‘The dog gnaws the bone.’ 
\z


When a tense adverb is needed, it comes directly after the verb and before any explicit subject. And any other adverbial elements like extended objects (\textsc{e}) or adverbs, in that order, come after the subject. This elaborated transitive clause structure is illustrated in (14):



\ea\label{ex:}
\gll {Átsʼá}\textsc{\textsubscript{v}}\textit{ bee}\textsc{\textsubscript{tense}}\textit{   ŋóká}\textsc{\textsubscript{a}}\textit{     ɔkáá}\textsc{\textsubscript{o}}\textit{         ódàtù}\textsc{\textsubscript{e}} \\
    \\
gnaw:\textsc{3sg}=yester-  dog:\textsc{nom}  bone:\textsc{acc}  day:\textsc{ins}
\glt ‘The dog gnawed the bone all day yesterday.’ 
\z




\subsubsection{Ditransitive}

Icétôd \textsc{ditransitive} clauses consist minimally of a ditransitive verb (\textsc{v}), an agent (\textsc{a}), an object (\textsc{o}), and an extended object (\textsc{e}) in a \textsc{vaoe} constituent order. If the agent is not mentioned explicitly, then it will still be marked with a suffix on the verb. The object and even extended object may be left implicit but will be understood from context. The basic ditransitive clause structure is illustrated in (15):




\ea\label{ex:}
\gll {Maa}\textsc{\textsubscript{v}}\textit{}     \textit{ƙaƙaama}\textsc{\textsubscript{a}}\textit{   ɔkáá}\textsc{\textsubscript{o}}\textit{     ŋókíkᵉ}\textsc{\textsubscript{e}}. \\
    \\
give:\textsc{3sg}   hunter:\textsc{nom}   bone:\textsc{acc}   dog:\textsc{dat}
\glt ‘The hunter gives a bone to the dog.’ 
\z




\subsubsection{Causative}

By adding an extra element in the form of a causing agent, Icétôd \textsc{causative} verbs change the structure of a clause. If the original clause was a \textsc{vs} intransitive one, then the causative changes it to a transitive \textsc{vao}. If the original clause was a transitive \textsc{vao}, then the causative changes it to a ditransitive \textsc{vaoe}. The following two examples show causative verbs making these structural changes:




Intransitive \textsc{vs} → Causative \textsc{vao}
\ea\label{ex:}
\gll {Fekíà}\textsc{\textsubscript{v}}     \textit{ŋkᵃ}\textsc{\textsubscript{v}}. \\
    \\
laugh:\textsc{1sg}   I:\textsc{nom}
\glt ‘I laugh’. 
\z




\ea\label{ex:}
\gll {Fekitéídà}\textsc{\textsubscript{va}}   \textit{\`{ŋ}kᵃ}\textsc{\textsubscript{o}}. \\
    \\
laugh:\textsc{caus:2sg} I:\textsc{nom}
\glt ‘You make me laugh.’ 
\z





Transitive \textsc{vao} → Causative \textsc{vaoe}
\ea\label{ex:}
\gll {Wetía}\textsc{\textsubscript{ v}}     \textit{ŋka}\textsc{\textsubscript{a}}\textit{     cue}\textsc{\textsubscript{o}}. \\
    \\
drink:\textsc{1sg}   I:\textsc{nom}   water:\textsc{nom}
\glt ‘I drink water.’ 
\z




\ea\label{ex:}
\gll {Wetitéída}\textsc{\textsubscript{va}}\textit{}   \textit{ŋka}\textsc{\textsubscript{o}}\textit{     cuékᵉ}\textsc{\textsubscript{E}}. \\
    \\
drink:\textsc{caus:2sg} I:\textsc{nom}    water:\textsc{dat}
\glt ‘You make me drink water.’ 
\z




\subsubsection{Auxiliary} 

Icétôd has both true \textsc{auxiliary} verbs and \textsc{pseudo-auxiliary} verbs. Both types create modified syntactic structures. The true auxiliaries, shown in \tabref{tab:10}.1, function as the syntactic main verb in a clause, while the \textit{semantic} main verb follows the subject (\textsc{s/a}) in a morphologically defective form that consists of the bare verb stem plus a suffix \{-a\} (which may be the realis marker from §8.9.2). This means the constituent order of clauses with true auxiliary verbs is \textsc{auxSV} for intransitives, \textsc{auxAVO} for transitives, and \textsc{auxAVOE} with extended objects. Again, in all these constructions, the \textsc{aux} acts as the main verb from a syntactic perspective, while the defective verb carries the main meaning of the verbal schema. Another way to analyze this construction would be to say that the auxiliary verb and the defective verb \textit{together} fill the verb slot of the clausal syntax.

The true auxiliaries have both lexical and aspectual meanings, which are nevertheless practically identical in their semantics. However, in their lexical function, the verbs in \tabref{tab:10}.1 do not require a second, morphologically defective verb to augment them; they stand alone.


\begin{table}
\caption{1: Icétôd true auxiliary verbs}
\label{tab:10}


\begin{tabularx}{\textwidth}{XXX}
\lsptoprule

Root & Lexical & Aspectual\\
erúts- & ‘be fresh, new’ & \textsc{recentive}\\
ŋɔr- & ‘do already/early’ & \textsc{anticipative}\\
sár- & ‘be still/not yet’ & \textsc{durative}\\
\lspbottomrule
\end{tabularx}
\end{table}
Example (18) illustrates the use of the recentive aspectual auxiliary verb \textit{erúts-} in an intransitive clause with the structure \textsc{auxSVE:}




\ea\label{ex:}
\gll {Erúts}\textit{íma}\textsc{\textsubscript{AuxS}}   \textit{atsa}\textsc{\textsubscript{v}}\textit{     sédàᵒ}\textsc{\textsubscript{e}}. \\
    \\
\textsc{recent:1pl.exc}   come     garden:\textsc{abl}
\glt ‘We just came from the garden.’ 
\z


Example (19), on the other hand, shows the use of the anticipative verb \textit{ŋɔr-} in a transitive clause with the structure \textsc{auxAVOE}:



\ea\label{ex:}
\gll {Ŋ}\textit{ɔrá}\textsc{\textsubscript{AuxA}}\textit{ naa   cɛa}\textsc{\textsubscript{v}}\textit{   riá}\textsc{\textsubscript{o}}\textit{        baratso}\textsc{\textsubscript{e}}\textit{ nákᵃ.} \\
    \\
\textsc{anticip:3sg=pst}   kill   goat:\textsc{acc} morn:\textsc{ins=dem.pst}
\glt ‘He already killed the goat earlier this morning.’ 
\z


Lastly, sentence (20) exemplifies the durative aspectual verb \textit{sár-} in a simple transitive clause working with the defective verb \textit{tsʼágwa-}:



\ea\label{ex:}
\gll {Sárá}\textsc{\textsubscript{Aux}}\textit{  séda}\textsc{\textsubscript{s}}\textit{     tsʼágwà}\textsc{\textsubscript{v}}. \\
    \\
dur:\textsc{3sg}   garden:\textsc{nom}   unripe
\glt ‘The garden is still unripe.’ 
\z


In contrast, the pseudo-auxiliary verbs only mimic true auxiliaries in that they are fully lexical verbs yet ones with potentially aspectual meanings, including the completive, inchoative, and occupative. However, because they are not \textit{syntactically} auxiliary, they take complements as any lexical verb would (direct objects for the transitive ones and extended objects for the intransitive one). The pseudo-auxiliaries are presented in \tabref{tab:10}.2 with their lexical and aspectual meanings and the cases required in their complements:


\begin{table}
\caption{2: Icétôd pseudo-auxiliary verbs}
\label{tab:10}


\begin{tabularx}{\textwidth}{XXXX}
\lsptoprule

Stem & Lexical & Aspectual & Case required\\
náb-ʉƙɔt- & ‘end, finish’ & \textsc{completive} & \textsc{nom/acc}\\
itsyák-ét- & ‘begin, start’ & \textsc{inchoative} & \textsc{nom/acc}\\
toɗó- & ‘alight, land’ & \textsc{inchoative} & \textsc{nom/acc}\\
isé-ét- & ‘begin, start’ & \textsc{inchoative} & \textsc{nom/acc}\\
cɛm- & ‘fight, struggle’ & \textsc{occupative} & \textsc{ins}\\
\lspbottomrule
\end{tabularx}
\end{table}
Each of the aspectual meanings listed in \tabref{tab:10}.2 are given one example in the following sentences. The brackets in (21) signify that the bracketed noun phrase as a whole is the object of the verb.



Completive
\ea\label{ex:}
\gll {Nábʉƙɔtɨáa}\textsc{\textsubscript{va}}\textit{    [isóméésá   ɲáɓúkwi]}\textsc{\textsubscript{o}}. \\
    \\
finish:\textsc{1sg:prf}   to.read:\textsc{nom}   book:\textsc{gen}
\glt ‘I have finished reading the book.’ 
\z




Inchoative
\ea\label{ex:}
\gll {Itsyaketátaa}\textsc{\textsubscript{va}}\textit{  wáánàkᵃ}\textsc{\textsubscript{o}}. \\
    \\
begin:\textsc{3pl:prf}   praying:\textsc{acc}
\glt ‘They have begun praying.’ 
\z




Occupative
\ea\label{ex:}
\gll {Cɛma}\textsc{\textsubscript{v}}  \textit{  wika}\textsc{\textsubscript{s}}\textit{       wáákᵒ}\textsc{\textsubscript{e}}. \\
    \\
fight:3   children:\textsc{nom}   playing:\textsc{ins}
\glt ‘The children are busy playing.’ 
\z




\subsubsection{Copular}

Icétôd \textsc{copular} clauses have relational rather than referential meanings. They link a \textsc{copular subject} (\textsc{cs}) to a \textsc{copular} \textsc{complement} (\textsc{cc}) which represents an entity or attribute, depending on the specific copular verb involved. The constituent order of copular clauses is therefore \textsc{v-cs-cc}. Icétôd has three distinct copular or ‘be’ verbs that can express five copular relationships between them. These copular verbs are presented in \tabref{tab:10}.3 below, along with the case markings their subjects and complements are obligated to have:


\begin{table}
\caption{3: Icétôd copular verbs}
\label{tab:10}


\begin{tabularx}{\textwidth}{XXXX}
\lsptoprule

Verb & Meaning & \textsc{cs} case & \textsc{cc} case\\
ì- & Existence & \textsc{nom} & \textsc{–}\\
& Location & \textsc{nom} & \textsc{dat}\\
ìr- & Attribution & \textsc{nom} & (adverb)\\
mɨt- & Identity & \textsc{nom} & \textsc{obl}\\
& Possession & \textsc{nom} & \textsc{gen}\\
\lspbottomrule
\end{tabularx}
\end{table}
The three copular verbs in \tabref{tab:10}.3 and their five potential meaning are each exemplified briefly in the sentences below:




Existence
\ea\label{ex:}
\gll {Ia}\textsc{\textsubscript{v}}\textit{     didigwarí}\textsc{\textsubscript{cs}}. \\
    \\
be:\textsc{3sg}   rain.top:\textsc{nom}
\glt ‘Heaven [i.e. God] is (there).’ 
\z




Location
\ea\label{ex:}
\gll {Ia}\textsc{\textsubscript{v}}\textit{   lɔŋɔtá}\textsc{\textsubscript{cs}}\textit{     muceékᵉ}\textsc{\textsubscript{cc}}. \\
    \\
be:3   enemies:\textsc{nom}   way:\textsc{dat}
\glt ‘Enemies are on the way.’ 
\z




Attribution
\ea\label{ex:}
\gll {Ira}\textsc{\textsubscript{vcs}}\textit{     tíyé}\textsc{\textsubscript{adv}}. \\
    \\
be:\textsc{3sg}   like.this
\glt ‘It is like this.’ 
\z




Identity
\ea\label{ex:}
\gll Mɨtɨá\textsc{\textsubscript{v}}   ŋka\textsc{\textsubscript{cs}}    bábò\textsc{\textsubscript{cc}}.\\
be:\textsc{1sg}   I:\textsc{nom}    father.your:\textsc{obl}\\
\glt ‘I am your father.’ 
\z




Possession
\ea\label{ex:}
\gll Mɨta\textsc{\textsubscript{v}}     [awa     na]\textsc{\textsubscript{cs}}   ŋgóᵉ\textsc{\textsubscript{cc}}.\\
be:\textsc{3sg}   home:\textsc{nom}   this   we:\textsc{gen}\\
\glt ‘This house is ours.’ 
\z




\subsubsection{Fronted}

Icétôd can put special emphasis on any core nominal element by moving it to the front of the clause, before the verb, subject, and other constituents. Doing so obviously disrupts the usual syntactic structure of main clauses. Two kinds of fronting are observed in the language: 1) a \textsc{cleft} construction and 2) \textsc{left-dislocation}. In a cleft construction, the emphasized noun is moved to the front and given the copulative case. This puts it in an identifying relationship with the original clause out of which it just came. As a result, the newly arranged clause can be viewed as a kind of copular clause where the fronted element is the copular subject and the original clause the copular complement. This can in turn be formulized as: [NP:\textsc{cop}]\textsc{\textsubscript{cs}}\textsc{ [clause]}\textsc{\textsubscript{cc}}. To make this more concrete, the next examples show the cleft construction with a simple transitive clause (29a) whose object (\textit{mɛs}) gets fronted and marked with the copulative case (29b). 




Cleft construction
\ea\label{ex:}
\gll {Bɛɗɨmà}\textsc{\textsubscript{v}}    \textit{\`{ŋ}gwà}\textsc{\textsubscript{a}}\textit{    mɛs}\textsc{\textsubscript{o}}. \\
    \\
want:\textsc{1pl.exc}   we:\textsc{nom}  beer:\textsc{nom}
\glt ‘We want beer.’ 
\z




\ea\label{ex:}
\gll {Mɛsɔɔ}\textsc{\textsubscript{cc}}     \textit{[ŋgóá    bɛɗɨm.]}\textsc{\textsubscript{cs}} \\
    \\
beer:cop    we:\textsc{acc}   want:\textsc{1pl.exc}   
\glt ‘It is beer (that) we want.’ 
\z


Whereas the cleft construction involves removing a clausal element from a clause and building a new clause, left-dislocation simply relocates the element to the front of the clause, but still within the same clause. In this fronted position it is given the nominative case. This type of fronting can be formulized as: [NP:\textsc{nom} \textsc{‖}\textsc{ clause]}\textsc{\textsubscript{clause}}, where the double vertical line symbolized a short pause. Icétôd left-dislocation is illustrated in the following two example sentences:




Left-dislocation
\ea\label{ex:}
\gll {Mée}   eníí     kaúdza díí. \\
    \\
not:\textsc{prf}   see:\textsc{1sg}   money:\textsc{nom}=\textsc{anph}
\glt ‘I haven’t seen that money.’ 
\z




\ea\label{ex:}
\gll {Kaúdza} díí,     mée     ení. \\
    \\
money:\textsc{nom}=\textsc{anph}   not:\textsc{prf}   see:\textsc{1sg}
\glt ‘That money, I haven’t seen (it).’ 
\z






\subsection{Subordinate clauses}
\subsubsection{Overview}

The constituent order of Icétôd \textsc{subordinate} clauses differs from that of \textsc{main} clauses. Specifically, Icétôd subordinate clauses exhibit an \textsc{sv} order with intransitive verbs, an \textsc{av} order with transitives, and an \textsc{ave} order with ditransitives—in short ‘\textsc{sv}’ instead of the usual ‘\textsc{vs}’. Case marking in subordinate clauses is also different: The fronted subject/agent and \textit{every} direct object take the accusative case. 

The next two subsection deal with two key kinds of subordinate clause, the relative (§10.3.2) and the adverbial (§10.3.3).


\subsubsection{Relative clauses}

\textsc{relative clauses} are subordinate clauses that modify a noun. Icétôd relative clauses are restrictive, meaning they can only narrow the reference of their head noun rather than adding extra details about it. Relative clauses are introduced by the tensed relative pronouns discussed back in (§5.7), which, within the relative clause, stand in for a noun in the main clause called the \textsc{common argument} (\textsc{ca}). As such, the common argument is a full verbal argument in the main clause, while in the relative clause, the relative pronoun fills its slot.

As a subordinate clause, an Icétôd relative clause exhibits a different constituent order than typical main clauses. That is, an intransitive relative clause has the order \textsc{sv} (instead of \textsc{vs}), and a transitive relative clause has the order \textsc{oav} (instead of \textsc{vao}). In the former (intransitive), the subject slot (\textsc{s}) is filled by the relative pronoun, and in the latter (transitive), it is the object (\textsc{o}) that is represented by the relative pronoun. Furthermore, apart for the relative pronouns themselves, all subjects and direct objects in relative clauses are marked with the accusative case—another sign of subordination.

These various attributes of Icétôd relative clauses are illustrated in examples (31)-(32) below. In (31), the common argument in the main clause is \textit{emuta} ‘story’, which is modified by the relative clause \textit{nɛ ɛf} ‘that is funny’. Note how the subject slot of the relative clause is filled by the relation pronoun \textit{nɛ} (actually \textit{na} with its vowel assimilated). Then, in (32), the common argument of the main clause is \textit{ima} ‘child’, modified by the relative clause \textit{náa ɲcia takí} ‘that I mentioned’. Observe that since the verb of the relative clause is transitive (\textit{takés} ‘to mean, mention’), it requires a direct object, which in this instance is fulfilled by the relative pronoun \textit{náa} representing the noun \textit{ima}:




[Warning: Draw object ignored]Intransitive (\textsc{sv})

\ea\label{ex:}
\gll {Nesíbimaa     emuta}\textsc{\textsubscript{ca}}\textit{   [nɛ}\textsc{\textsubscript{s}}\textit{   ɛf}\textsc{\textsubscript{v}}\textit{]}\textsc{\textsubscript{rel}}. \\
    \\
hear:\textsc{1pl.exc:prf}  story:\textsc{nom}   =\textsc{rel}   sweet:\textsc{3sg}
\glt ‘We’ve heard a story that is funny.’ 
\z


[Warning: Draw object ignored]Transitive (\textsc{oav})
\ea\label{ex:}
\gll {Atsáá       ima}\textsc{\textsubscript{ca}}\textit{   [náa}\textsc{\textsubscript{o}}\textit{   ɲcia}\textsc{\textsubscript{a}}\textit{   takí}\textsc{\textsubscript{v}}\textit{]}\textsc{\textsubscript{rel}}. \\
    \\
come:\textsc{3sg}:\textsc{prf}   child   =\textsc{rel}   I:\textsc{acc}   mention:\textsc{1sg}
\glt ‘The child I mentioned earlier has come.’ 
\z




\subsubsection{Adverbial clauses}

The category of \textsc{adverbial clauses} is rather broad as it includes any subordinate clause that modifies a main clause adverbially. Adverbial clause are subordinate or ‘dependent’ precisely because they cannot stand alone but must be linked to an independent main clause. As subordinate clauses, adverbial clauses exhibit a constituent order that differs from both main clauses and relative clauses. Specifically, intransitive adverbial clauses have the order \textsc{sv}, while transitive adverbial clauses have the order \textsc{avo}. Another correlate of subordination seen in most adverbial clauses—with the exception of the conditional and hypothetical ones— is accusative case-marking on all core constituents (\textsc{s/a/o}) if they are explicitly mentioned. 

Among the main kinds of adverbial clause in Icétôd are the following: \textsc{temporal}, \textsc{simultaneous}, \textsc{conditional}, \textsc{hypothetical}, \textsc{manner}, \textsc{reason}/\textsc{cause}, and \textsc{concessive}. Most types of adverbial clause—except for the manner type—have their own dedicated connective (or ‘conjunction’) or set of connectives, many of which are listed back in \tabref{tab:3}.8 under §3.14. Without exception, the subordinating connectives come first in the adverbial clause. Lastly, Icétôd adverbial clauses may come before or after the main clause they modify. 

Each type of adverbial clause is given one example apiece below:




Temporal
\ea\label{ex:}
\gll [Noo   ntsíá     baduƙotâdᵉ]\textsc{\textsubscript{temp}} ,   ƙɔɗɨakᵒ. \\
    \\
when   he:\textsc{3sg}   die:\textsc{3sg:dp}     cry:\textsc{1sg:seq}
\glt ‘When he died, I cried.’ 
\z




Simultaneous
\ea\label{ex:}
\gll [Náa   ntsíá     badúƙótìkᵉ]\textsc{\textsubscript{simul}}\textit{,   ƙɔɗɛsɨakᵒ.} \\
    \\
as   he:\textsc{3sg}   die:\textsc{3sg:sim}    cry:\textsc{ipfv}:\textsc{1sg:seq}
\glt ‘As he was dying, I was crying.’ 
\z




Conditional
\ea\label{ex:}
\gll [Na   ntsa     badúƙótùkᵒ]\textsc{\textsubscript{cond}}\textit{,   ƙɔɗɨakᵒ.} \\
    \\
if   he:\textsc{nom}   die:\textsc{3sg:seq}     cry:\textsc{1sg:seq}
\glt ‘If he dies, I’ll cry.’ 
\z




Hypothetical
\ea\label{ex:}
\gll [Na   ƙánoo   ntsa    badúƙótùkᵒ]\textsc{\textsubscript{hypo}}\textit{,}  \\
    \\
if   would’ve   he:\textsc{3sg}  die:\textsc{3sg:seq}   
\z

\ea\label{ex:}
\gll ƙɔɗɨaa   ƙánòkᵒ. \\
    \\
cry:\textsc{1sg:seq}  would’ve
\glt ‘If he would’ve died, I would’ve cried.’ 
\z




Manner
\ea\label{ex:}
\gll Badúƙótuo   [(ntsíá)   tisílíkᵉ]\textsc{\textsubscript{manner}}. \\
    \\
die:\textsc{3sg:seq}   (he:\textsc{acc})  peaceful:3\textsc{sg:sim}
\glt ‘And he died peacefully (lit. ‘he being peaceful’).’ 
\z




Reason/cause
\ea\label{ex:}
\gll Baduƙotáá   [ɗúó     ídzanâdᵉ]\textsc{\textsubscript{reason}}. \\
    \\
die:\textsc{3sg:prf}   because   shoot:\textsc{ips:3sg:dp}
\glt ‘He has died because he was shot.’ 
\z




Concessive
\ea\label{ex:}
\gll [Áta   ntsíá     badúƙótìkᵉ]\textsc{\textsubscript{concess}}\textit{,   ńtá   ƙɔɗɨ.} \\
    \\
even   he:\textsc{acc}   die:\textsc{3sg:sim}    not   cry:\textsc{1sg}
\glt ‘Even if he dies, I will not cry.’ 
\z






\subsection{Questions}
\subsubsection{Overview}

Questions in Icétôd can be formed in two mutually exclusive ways: 1) by leaving the final word in the question in its non-final form (along with special a special questioning intonation) or 2) by rearranging the syntax of the sentence. The first method is employed with what is called \textsc{polar} or yes/no questions: those whose answer is either ‘yes’ or ‘no’. The second method is used for \textsc{content} or wh-questions: those whose answer is a substantive response to such interrogative pronouns as \textit{who?}, \textit{what?}, \textit{when?}, \textit{where?}, etc. These two types of question are briefly described in the following two subsections.


\subsubsection{Polar questions}

Polar questions are those that elicit a ‘yes’ or ‘no’ in response. In Icétôd, they are formed by leaving the last word or particle in the question in its non-final form (revisit §2.3 and §2.4.3 for a review). This open-endedness of form is a fascinating way the grammar reflects the open-endedness of a question—open to a response. Besides the non-final form of the last word, polar questions are often identified by a change in intonation. This interrogative intonation is enacted by what is called a \textsc{boundary} low tone: a low tone that attaches to the final syllable. If the final syllable already has a low tone, then the boundary tone is not audible. But if the final syllable has a high tone, the boundary tone manifests as a high-low glide. 

The following two examples illustrate these features of polar questions. Note in the first part of (40) how the present perfect suffix \{-\'{} ka\} shows up in its non-final form (\textit{{}-\'{} à}), while in the second part, the final form is used (\textit{{}-\'{} kᵃ}). Then, (41) shows the interrogative boundary low tone attaching to the high tone on the final syllable of \textit{cekúó} ‘is a woman’, creating a high-low down-glide (\textit{cekúô}):




\ea\label{ex:}
\gll Nábʉƙɔtá\textit{à}\textit{?      Ee, nábʉƙɔtá}kᵃ. \\
    \\
finish:\textsc{comp:3sg:prf}[\textsc{nf}]  yes finish:\textsc{comp:3sg:prf[ff]}
\glt ‘Is it finished?’    ‘Yes, it is finished.’ 
\z




\ea\label{ex:}
\gll Cekú\textit{ô}?      Ee, cekúó     ntsaᵃ. \\
    \\
woman:\textsc{cop[nf]}    yes woman:\textsc{cop}   she:\textsc{nom}
\glt ‘Is it a woman?’    ‘Yes, she’s a woman.’ 
\z




\subsubsection{Content questions}

In contrast to polar questions, content questions cannot logically take ‘yes’ or ‘no’ for an answer. Rather, answers to content questions—as their name implies—must contain content relevant to the specific interrogative pronoun used to make the inquiry (Icétôd interrogative pronouns are listed in §5.5). So if the question contains the pronoun \textit{ǹdò-} ‘who?’, the answer must include a person. Or if the question contains the pronoun \textit{ndáí-} ‘where?’, the response must refer to a specific location, and so on. Icétôd forms content questions by placing an interrogative pronoun in the syntactic slot of the unknown entity being queried (i.e. a person, place, time, manner, etc.). For example, in (42) below, the interrogative pronoun \textit{ndaí-} ‘where?’ is filling the normal place where an object encoding the destination of \textit{ƙà-} ‘go’ would go. The same is true in (43), where the pronoun \textit{ìsì-} ‘what?’ fills the direct object slot required by the verb \textit{bɛɗ-} ‘want’:




\ea\label{ex:}
\gll Ƙeesída     ndaíkᵉ? \\
    \\
go:\textsc{int:2sg:real}   where:\textsc{dat}
\glt ‘You are going where? 
\z




\ea\label{ex:}
\gll Bɛɗá       ìsìkᵃ? \\
    \\
want:\textsc{3sg:real}   what:\textsc{nom}
\glt ‘He wants what?’ 
\z


However, what is more common is for the interrogative pronoun to be fronted for emphasis (see §10.2.7). As in other instances of fronting in Icétôd, the fronted element is given the copulative case marker \{-ko\}. In the sentences below, examples (42)-(43) are repeated in their fronted (focused) forms, and two other pronouns are illustrated:



\ea\label{ex:}
\gll Ndaíó   ƙeesídàdᵉ? \\
    \\
where:\textsc{cop}   go:\textsc{2sg:real:dp}
\glt ‘Where are you going?’ 
\z




\ea\label{ex:}
\gll Isio     bɛɗᵃ? \\
    \\
what:\textsc{cop}   want:\textsc{3sg:real}
\glt ‘What does he want?’ 
\z




\ea\label{ex:}
\gll Ndoo     óá       ɲcìkᵃ? \\
    \\
who:\textsc{cop}   call:\textsc{3sg:real}   I:\textsc{acc}
\glt ‘Who calls me?’ 
\z




\ea\label{ex:}
\gll {\'{N}tɛɛnɔɔ   tákîdᵃ?} \\
    \\
which:\textsc{cop}   mean:\textsc{2sg:real}
\glt ‘Which (one) do you mean?’ 
\z






\subsection{Quotations}


Quotations involve reporting someone’s speech (or thought)—the speaker’s own or someone else’s—directly or indirectly. Icétôd fulfills this communicative need through the use of the verb \textit{kʉt-} ‘say’ followed by the actual quotation treated as an add-on clause. That is, unlike complements described below in §10.6, a quoted sentence in Icétôd is \textit{not} an object of the verb \textit{kʉt-}. Instead, it is tacked on ‘extra-syntactically’ and given the oblique case (the ‘leftover’ case). This is proven by the fact that when the pronoun \textit{ìsì-} ‘what?’ appears to be the object of \textit{kʉt-} with a \textsc{3sg} or \textsc{3pl} subject, \textit{ìsì-} takes the oblique case instead of the accusative case as one would expect otherwise (§7.3):




\ea\label{ex:}
\gll Kʉtà     ìs?    NOT  *Kʉta   ísìk? \\
    \\
say:\textsc{3sg}   what[\textsc{obl}]    say:\textsc{3sg}   what:\textsc{acc}
\glt ‘He says what?’      ‘He says what?’ 
\z


Many languages, English included, distinguish between direct and indirect quotative formulas, for example the direct “I said, ‘I will come’” or the indirect “I said I will come”. By contrast, Icétôd does not distinguish the two grammatically. Instead, the proper sense has to be discerned from the context (and possibly from intonation). So the statement \textit{Kʉtɨá naa atsésí} could mean either “I said ‘I will come’” or “I said I will come”, depending on factors other than syntax. 

In Icétôd quotative sentences, if there is an addressee of the quotation, they will appear in the dative case. And the quotative particle \textit{tàà} ‘that’ is often inserted just before the quotation, though it is optional. Sentences (49)-(50) below provide some examples of Ik quotations:



\ea\label{ex:}
\gll Kʉtɨá     bie   [Pakóícéo noo   dzígwì]\textsc{\textsubscript{quotation}} \\
    \\
say:\textsc{1sg}   you:\textsc{dat} Turkana:\textsc{cop=pst} buy:\textsc{plur}
\glt ‘I’m telling you it was the Turkana who used to buy.’ 
\z




\ea\label{ex:}
\gll Kʉtana   taa   [atsúó   ɗɛmʉs]\textsc{\textsubscript{quotation}}  \\
    \\
say:\textsc{ips}  that   come:\textsc{imp}   quickly
\glt ‘They are saying, ‘Come quickly!’.” 
\z






\subsection{Complements}


\textsc{Complements} are individual clauses that function as an ‘argument’\textsc{} of the verb—as either subject or object. In other words, they are clauses within clauses: unlike subordinate clauses which are added \textit{onto} main clauses, complement clauses are added \textit{into} other clauses. Icétôd complement clauses are introduced by the \textsc{complementizer} \textit{tòìmɛnà-} ‘that’, which is combination of a form of the verb \textit{tód-} ‘speak’ and the noun \textit{mɛná-} ‘issues, words’. This compound word gives some evidence that Icétôd complement clauses (of this particular type) evolved from quotative clauses like those described above in §10.5.

Because a complement clause fits within the grammar of a clause, it must somehow be declined for case (because all arguments of a verb in Icétôd take case, without exception). To meet this requirement, the complementizer \textit{tòìmɛnà-} bears the burden of case on behalf of the whole complement clause it is introducing. So technically, it is the complementizer—not the complement clause alone—that is the verbal argument. But because \textit{tòìmɛnà-} plus the complement is a frozen quotative formula, the whole construction can be seen as an argument.

To illustrate this, (51) presents a simple complement clause governed by the cognitive verb \textit{èn-} ‘see’. The \{curly brackets\} indicate the boundaries of the main clause from the point of view of the syntax, in which the verb \textit{èn-} ‘see’ selects its object \textit{tòìmɛnà-} ‘that’ for the accusative case. The [square brackets] mark the boundary of the complement clause seen from the point of view of semantics, for the actual content of ‘seeing’ is the clause \textit{that we have become very rich}:




\ea\label{ex:}
\gll {\{Enáta  [toimɛnaa\}}\textsc{\textsubscript{obj}}\textit{ barʉƙɔtɨmáà   zùkᵘ]}\textsc{\textsubscript{compl}} \\
    \\
see:\textsc{3pl}   that:\textsc{acc}    rich:\textsc{comp:1pl:prf}   very
\glt ‘They see that we have become very rich.’ 
\z


In addition to a direct object, an Ik complement clause can also function as an indirect object or even the ‘complement’ of a copular clause. For instance, in (52) below, \textit{tòìmɛnà-} and by extension the whole complement clause is acting as the indirect object of the verb \textit{xɛɓ-} ‘be afraid of, fear’, which requires the ablative case. Then, in (53), the verb is the copular verb \textit{mɨt-} ‘be’, which requires its nominal compliment to be in the oblique case, as is seen with \textit{tòìmɛnà-}:



\ea\label{ex:}
\gll Xɛɓɨá     [toimɛnɔɔ   maíá     sílím]\textsc{\textsubscript{compl}} \\
    \\
fear:\textsc{1sg}   that:\textsc{abl}   ill:\textsc{1sg}   AIDS:nom
\glt ‘I am afraid that I’m ill with AIDS.’ 
\z




\ea\label{ex:}
\gll Mɨta ʝa   [toimɛna   ńtá   nesíbi       mɛnákᵃ]\textsc{\textsubscript{compl}} \\
    \\
be:\textsc{3sg}=just   that[\textsc{obl}]   not   hear:\textsc{3sg} words:\textsc{acc}
\glt ‘It is just that she doesn’t understand instructions.’ 
\z






\subsection{Comparatives}


\textsc{Comparatives} are grammatical constructions that allow the comparison of two entities on the basis of some characteristic. Icétôd has two strategies for doing this: 1) the mono-clausal, which involves one simple clause, and 2) the bi-clausal, which involves a complex clause. Mono-clausal comparatives place the \textsc{comparee} (entity being compared) in the nominative case and the \textsc{standard} (entity the comparee is being compared to) in the ablative case. Since most comparable attributes are expressed as intransitive verbs in Icétôd, the \textsc{parameter} (attribute) of the comparison is also an adjectival verb in such constructions. For example, in (54)-(55) below, the intransitive verbs \textit{zè-} ‘big’ and \textit{dà-} ‘nice’ are acting as the parameters, while their subjects are the comparees in the nominative case and their extended objects the standards in the ablative case:




\ea\label{ex:}
\gll {Zeíá     \'{ŋ}kà     bù.} \\
    \\
big:\textsc{1sg}   I:\textsc{nom}   you:\textsc{abl}
\glt ‘I am bigger than you.’ 
\z




\ea\label{ex:}
\gll Daa     ɗa na       kɨɗɔɔ \\
    \\
nice:\textsc{3sg}   this.one:\textsc{nom}=this   that.one:\textsc{abl}
\glt ‘This one is nicer than that one.’ 
\z


Bi-clausal comparatives, on the other hand, combine a main clause with a subordinate or ‘co-subordinate’ clause (§10.8.2). Both types are introduced by the verb \textit{ɨlɔ-} ‘exceed, surpass’, which acts as the \textsc{index} of the comparison (the gauge of the degree of difference between compared entities). If the indexical verb introduces a subordinate clause, it takes the simultaneous aspect, while if it introduces a co-subordinate clause, it takes the sequential aspect. In such bi-clausal comparatives, the comparee is still the subject of the main clause, while the standard is the object of the dependent clause. And the parameter remains with the main clause verb (as in mono-clausal comparatives). But unlike mono-clausals, bi-clausal comparatives can have intransitive or transitive parametric verbs. In other words, actions as well as attributes can be compared in this type of construction.

In (56) below, the parameter lies with the verb \textit{tɔkɔb-} ‘cultivate’, and ‘he’ (marked as 3\textsc{sg} on the verb) is being compared with ‘us’ (\textit{ŋgó-}). The index of the comparison is the verb \textit{ɨlɔɨɛ} ‘he surpassing’, which reveals the inequality of the compared actions of the two entities. Example (57) follows the exact same logic, only that the indexical verb \textit{ɨlɔɨnɨ} is in the sequential aspect instead of the simultaneous: 



\ea\label{ex:}
\gll Tokobia     eɗi̊á        [ɪlɔɨɛ     ŋgókᵃ]\textsc{\textsubscript{sim}} \\
    \\
cultivate:\textsc{plur:3sg} grain:\textsc{acc} surpass:\textsc{3sg:sim} we:\textsc{acc}
\glt ‘He cultivates grain more than us.’ 
\z




\ea\label{ex:}
\gll Sáɓúmósáta     [ɨlɔɨnɨ          toni  ɲeryaŋ]\textsc{\textsubscript{seq}} \\
    \\
kill:\textsc{recip:3pl} exceed:\textsc{3pl:seq} even government[\textsc{obl}]
\glt ‘They’re killing each other even more than the feds.’ 
\z






\subsection{Clause combining}
\subsubsection{Clause coordination}

Two or more clauses can be linked in Icétôd through clause \textsc{coordination}. This can result in clause \textsc{addition} (‘and’), which joins two independent clauses of equal status. It can result in \textsc{contrast} (‘but’), which joins clauses of equal syntactic status, the second of which is a counterexpectation to the first. And thirdly, clause coordination can result in \textsc{disjunction} (‘or’), in which two clauses of equal status are presented as different possible options.

Clause addition is achieved in two primary ways: 1) simply adjoining the clauses with a pause in between (represented by a period in writing) or 2) linking the clauses with a coordinating connective like \textit{koto} ‘and, but, then’ or \textit{ńdà} ‘and’. A third way to add a clause that alters its syntactic profile is to nominalize it—change all its main parts to nouns, put them in a noun phrase, and link it up to the first with \textit{ńdà}. Note from (42) that with this third method, because the word \textit{ńdà} ‘and’ is acting as a sort preposition, it requires its head noun(s) to be in the oblique case. Its head nouns in example (42) are the nominalized subject (\textit{ŋgo}) and verb (\textit{ŋƙɛsɨ})—both in the oblique case.

Each of above three ways of adding clauses are illustrated below:




\ea\label{ex:}
\gll Mɨnɨa     ɲécáyᵃ.   Mɨná       ntsa   mɛsɛkᵃ. \\
    \\
love:\textsc{1sg}   tea:\textsc{nom}   love:\textsc{3sg} she:\textsc{nom}   beer:\textsc{acc}
\glt ‘I love tea. She loves beer.’ 
\z




\ea\label{ex:}
\gll Ƙaƙiésána noo       ńtí, \\
    \\
hunt:\textsc{plur:ipfv:ips:real=pst}   how
\glt ‘How did people used to go hunting, 
\z



\ea\label{ex:}
\gll ńda   ƙaíána noo         waa   waicíkée     ńti? \\
    \\
and   go:\textsc{plur:ips:real=pst} pick:\textsc{nom} greens:\textsc{gen} how
\glt and how did they used to go picking greens?’
\z  



\ea\label{ex:}
\gll Itétimaa awákᵉ, \\
    \\
return:\textsc{1sg:seq} home:\textsc{dat} 
\glt ‘We returned home, 
\z



\ea\label{ex:}
\gll ńda\textit{}  \textit{ŋgo}\textit{}     \textit{ŋƙɛsɨ}\textit{}     tɔbɔŋɔᵉ. \\
    \\
and   we:\textsc{obl}   to.eat:\textsc{obl}   mush:\textsc{gen}
\glt and we ate mealmush.’
\z  

Clause contrast in Icétôd can be expressed in two primary ways: 1) by simply adjoining the two clauses with a pause in between (marked with by a comma or period in writing) or 2) by linking the two clauses with the contrastive connective \textit{kòtò}, which can mean ‘but’ as well as ‘and, then, therefore, etc.’. Both types are exemplified below:



\ea\label{ex:}
\gll Bɛna     ɲcùkᵒ.     Bùkᵒ. \\
    \\
not:\textsc{3sg}   I:\textsc{cop}     you:\textsc{cop}
\glt ‘It’s not me. It’s you. 
\z




\ea\label{ex:}
\gll Bɛɗʉƙɔtɨa naa     ɲɛmɛlɛkʉ, \\
    \\
search:\textit{comp:1sg=pst}   hoe:\textit{nom}
\glt ‘I went and looked for the hoe, 
\z



\ea\label{ex:}
\gll koto   máa naa   ŋunetí. \\
    \\
but   not=\textit{pst}   find:\textit{1sg}
\glt but I did not find (it).’
\z  

Lastly, the idea of disjunction is expressed in Icétôd through the use of the connectives \textit{kèɗè} ‘or’ or \textit{kòrì} ‘or’, as illustrated in (43)-(44) below:



\ea\label{ex:}
\gll Tɔkɔbɛsɨda       eɗa,  \\
    \\
farm:\textsc{ipfv:2sg:real}   grain:\textsc{nom}
\glt ‘Are you farming grain, 
\z



\ea\label{ex:}
\gll keɗe   ńtá   tɔkɔbɛsɨdᶤ? \\
    \\
or   not   farm:\textsc{ipfv:2sg:real}
\glt or are you not farming (it)?’
\z  



\ea\label{ex:}
\gll Enída       mɛna     gaanaakátìkᵉ, \\
    \\
see:\textsc{2sg:real}   things:\textsc{nom}   bad:\textsc{distr:3pl:sim}
\glt ‘Do you see things being bad all around, 
\z



\ea\label{ex:}
\gll kori   maráŋaakátìkᵉ? \\
    \\
or   good:\textsc{distr:3pl:sim}
\glt or as being good all around?
\z  



\subsubsection{Clause chaining}

But in fact, the most common way Icétôd links independent clauses is through clause ‘co-subordination’ or \textsc{clause chaining}. To create a chain of clauses, the grammar starts with an anchoring phrase or clause to set the stage modally or temporally, and then it puts all the following mainline verbs in the sequential aspect (see §8.10.7), creating a chain of two or more clauses. When clause chaining is used in a story, the temporal ‘anchor’ can be a simple time expression like \textit{kaɨnɨkò nùkᵘ} ‘in those years’ or a tensed statement like \textit{Atsa noo ámá ntanée taa Apáálɔrɛŋ} ‘There came a man named Apaaloreng’. In (47) below, the clause chain is anchored by the initial adverbial phrase \textit{Na kónít}\textit{ó ódoue baratsoó} ‘One day, in the morning’, which puts the whole sentence in a temporal frame. Thenceforth, the clause chain proceeds clause by clause, each marked as \textsc{seq1}, \textsc{seq2}, etc.:




\ea\label{ex:}
\gll [Na     kónító      ódoue   baratsoó]\textsc{\textsubscript{adv}} \\
    \\
when    one    day:\textsc{gen}   morning:\textsc{ins}   
\glt ‘One day, in the morning,  
\z

\ea\label{ex:}
\gll [ipu\textbf{\textit{o}}\textit{            taƙáɨkakᵃ]}\textsc{\textsubscript{seq1}} \\
    \\
cast:\textsc{3sg:seq}   shoes:\textsc{acc}
\glt he cast (his) shoes (in divination),
\z  
\ea\label{ex:}
\gll [eɡu\textbf{\textit{o}}\textit{           taƙáɨka         ɛbakᵃ]}\textsc{\textsubscript{seq2}} \\
    \\
put:\textsc{3sg:seq}   shoes\textsc{:nom}     gun:\textsc{acc}
\glt and the shoes made (the shape of) a gun,
\z  

\ea\label{ex:}
\gll [ipu\textbf{\textit{o}} \textit{           naɓó]}\textsc{\textsubscript{seq3}} \\
    \\
cast:\textsc{3sg:seq}   again 
\glt and he cast (them) again,
\z  

\ea\label{ex:}
\gll [eɡ\textbf{\textit{ini}}\textit{      ɛbakᵃ]}\textsc{\textsubscript{seq4}} \\
    \\
put:\textsc{3pl}:\textsc{seq}     gun:\textsc{acc}
\glt and they made a gun.’
\z  

Although the sequential aspect and clause chains are very common in narrative discourse, they are also used extensively for other types of discourse, for example, exposition and instruction. The following expository clause chain in (48) details some of the steps taken in the cultural activity of grinding tobacco leaves. Note that there are two anchoring adverbial clauses, one at the beginning and one in the third line. After each one, there is a string of one or more verbs (and clauses) set in the sequential aspect. 



\ea\label{ex:}
\gll [Náa   iryámétanɨɛ   gwasákᵉ]\textsc{\textsubscript{adv1}} \\
    \\
when   get:\textsc{ips:sim}   stone:\textsc{dat}
\glt ‘When a stone is acquired, 
\z



\ea\label{ex:}
\gll [ŋɔɛ\textit{ɛsɛ}\textit{     ɲaɓáláŋɨtᵃ]}\textsc{\textsubscript{seq1}} \\
    \\
grind:\textsc{inch:sps}   soda.ash:\textsc{nom}
\glt soda ash is ground up.
\z  


\ea\label{ex:}
\gll [náa   ɲaɓáláŋɨtɨá     iwíɗímètìkᵉ]\textsc{\textsubscript{adv2}} \\
    \\
when   soda.ash:\textsc{acc}   pulverize:\textsc{mid:sim}
\glt When the soda ash is ground to powder,
\z  


\ea\label{ex:}
\gll [eg\textit{esé}\textit{e   lɔtɔɓᵃ]}\textsc{\textsubscript{seq2}} \\
    \\
put:\textsc{sps:dp}   tobacco:\textsc{nom}
\glt tobacco is put into it,
\z  


\ea\label{ex:}
\gll [ŋɔ\textit{ɛsɛ}\textit{]}\textsc{\textsubscript{seq3}} \\
    \\
grind:\textsc{sps}
\glt and it is ground
\z  


\ea\label{ex:}
\gll [páka ɲapúɗúmùƙòtù\textit{kᵒ}\textit{]}\textsc{\textsubscript{seq4}} \\
    \\
until powdery:\textsc{comp:seq}
\glt until it becomes fine powder....’
\z  

Finally, the sequential aspect and clause chaining is often found operating in a set of consecutive instructions. As instructions, the clause chain may begin with one or more imperative verbs, followed by the sequential verbs in a chain of further commands or instructions.



\ea\label{ex:}
\gll [Na   bɛɗɨdɔɔ     bɛrɛsá   hoe]\textsc{\textsubscript{adv}} \\
    \\
if   want:\textsc{2sg:seq}   to.build:\textsc{nom}  house:\textsc{gen}
\glt ‘If you want to build a house, 
\z



\ea\label{ex:}
\gll [bɛrɛ     tí]\textsc{\textsubscript{imp1}} \\
    \\
build:\textsc{imp}   like.this
\glt build (it) like this:
\z  


\ea\label{ex:}
\gll [Kawete   titíríkᵃ,   kɛɗɨtɨn,   ńda   sim]\textsc{\textsubscript{imp2}} \\
    \\
cut:\textsc{imp}   pole:\textsc{pl}   reed:\textsc{p}    and   fiber
\glt Cut poles, reeds, and fiber,
\z  


\ea\label{ex:}
\gll [iréɲuƙoidu\textit{o}\textit{     bácɨkᵃ]}\textsc{\textsubscript{seq1}} \\
    \\
clear:\textsc{comp:2sg:seq}   area:\textsc{nom}
\glt clear away the area,
\z  


\ea\label{ex:}
\gll [úgidu\textit{o}\textit{   ripitín]}\textsc{\textsubscript{seq2}} \\
    \\
dig:\textsc{2sg:seq}   hole:\textsc{pl:nom}
\glt dig holes,
\z  


\ea\label{ex:}
\gll [otíduk\textit{w}\textit{éé     titíríkᵃ]}\textsc{\textsubscript{seq3}} \\
    \\
pour:\textsc{2sg:seq:dp}   pole:\textsc{pl:nom}
\glt and put the poles into them...’
\z  

\subsection{Appendix A: Icétôd affixes}

All of the affixes discussed in the preceding sketch are listed in the table below for easy reference. When looking for an affix in the list, keep in mind that if it has two forms (for example the \{-e\} and \{-ɛ\} of the genitive case), both forms are given their own separate entry.

\begin{table}
\caption{ Full list of Icétôd affixes}

\begin{tabularx}{\textwidth}{XXXX}
\lsptoprule

Non-final & Final & Name & Section\\
{}-Ø & {}-Ø & Irrealis modality & §8.9.1\\
{}-Ø & {}-Ø & Oblique case & §7.9\\
{}-a & {}-ᵃ & Nominative case & §7.2\\
{}-a & {}-ᵃ & Realis modality & §8.9.2\\
{}-a & {}-kᵃ & Accusative case & §7.3\\
{}-a & {}-kᵃ & Present perfect aspect & §8.10.2\\
{}-aák- & – & Distributive adjectival & §8.11.5\\
{}-am(a)- & {}-am & Singulative & §4.2.3\\
{}-am(á)- & {}-am & Patientive & §8.3.3\\
{}-án- & – & Stative adjectival & §8.11.4\\
{}-an(ɨ)- & – & Impersonal passive mood & §8.6.2\\
{}-ano\'{}  & {}-ano\'{}  & First plural incl. optative & §8.10.3\\
{}-ás(ɨ)- & {}-ás & Abstractive & §8.3.1\\
{}-át(ì)- & {}-át(ì) & Third person plural & §8.7\\
{}\'{-}d- & – & Physical property I & §8.11.2\\
{}-e & {}-ᵉ & Genitive case & §7.5\\
{}-e & {}-kᵉ & Dative case & §7.4\\
{}-e & {}-kᵉ & Simultaneous aspect & §8.10.8\\
{}\'{-}è & {}\'{-}dᵉ & Dummy pronoun & §8.8\\
{}-e\'{}  & {}-ᵉ\'{}  & Imperative singular & §8.10.5\\
{}-ed(e)- & {}-edᵉ & Possessive singular & §4.2.4\\
{}-és- & – & Imperfective aspect & §8.10.1\\
{}-és- & – & Intentional modality & §8.10.1\\
\lspbottomrule
\end{tabularx} 

\end{table}


\begin{tabularx}{\textwidth}{XXXX}
\lsptoprule

Non-final & Final & Name & Section\\
{}-és(í)- & {}-és & Transitive infinitive & §8.2.2\\
{}-ese\'{}  & {}-esᵉ\'{}  & Sequential imp. passive & §8.10.7\\
{}-èt- & – & Venitive directional & §8.4.1\\
{}-èt- & – & Inchoative aspect & §8.5.1\\
{}-ɛ & {}-ᵋ & Genitive case & §7.5\\
{}-ɛ & {}-kᵋ & Dative case & §7.3\\
{}-ɛ & {}-kᵋ & Simultaneous aspect & §8.10.8\\
{}\'{-}ɛ & {}\'{-}dᵉ & Dummy pronoun & §8.8\\
{}-ɛ\'{}  & {}-ᵋ\'{}  & Imperative singular & §8.10.5\\
{}-ɛd(ɛ)- & {}-ɛdᵋ & Possessive singular & §4.2.4\\
{}-ɛs- & – & Imperfective aspect & §8.10.1\\
{}-ɛs- & – & Intentional modality & §8.10.1\\
{}-ɛs(ɨ)- & {}-ɛs & Transitive infinitive & §8.2.2\\
{}-ɛsɛ\'{}  & {}-ɛsᵋ\'{}  & Sequential imp. passive & §8.10.7\\
{}-ɛt- & – & Venitive directional & §8.4.1\\
{}-ɛt- & – & Inchoative aspect & §8.5.1\\
{}-ì & {}-ⁱ & Third person singular & §8.7\\
{}\'{-}ì & {}\'{-}dᵉ & Dummy pronoun & §8.8\\
{}-í- & – & Pluractional aspect & §8.5.3\\
{}-í(í)- & {}-í(í) & First person singular & §8.8\\
{}-ia\'{} - & – & First singular sequential & §8.10.7\\
{}-íd(ì)- & {}-íd(ì) & Second person singular & §8.8\\
{}-ìk- & – & Distributive adjectival & §8.11.5\\
{}-ìk(à)- & {}-ìkᵃ & Plurative & §4.2.1\\
{}-íkó/-íkw- & {}-íkᵒ & Plurative & §4.2.1\\
{}-ím(í)- & {}-ím(í) & First plural exclusive & §8.7\\
{}-ima\'{} - & {}-ima\'{}  & First pl. exc. optative & §8.10.3\\
{}-ima\'{} - & – & First pl. exc. sequential & §8.10.7\\
{}-ímét- & – & Middle II mood & §8.6.3\\
\lspbottomrule
\end{tabularx}

\begin{tabularx}{\textwidth}{XXXX}
\lsptoprule

Non-final & Final & Name & Section\\
{}-ìn(ì)- & {}-ìn & Possessive plural & §4.2.4\\
{}-ine\'{}  & {}-ine\'{}  & First singular optative & §8.10.3\\
{}-ìnì & {}-ìn & Third plural sequential & §8.10.7\\
{}-ínós(í)- & {}-ínós & Reciprocal & §8.6.4\\
{}-ísín(ì)- & {}-ísín(ì) & First plural inclusive & §8.7\\
{}-ìt- & – & Causative mood & §8.6.5\\
{}-ít(í)- & {}-ít(í) & Second person plural & §8.7\\
{}-ítín(í)- & {}-ítín & Plurative & §4.2.1\\
{}-ɨ & {}-ᶤ & Third person singular & §8.7\\
{}\'{-}ɨ & {}\'{-}dᵉ & Dummy pronoun & §8.8\\
{}-ɨ(ɨ)- & {}-ɨ(ɨ) & First person singular & §8.7\\
{}-ɨa\'{} - & – & First singular sequential & §8.10.7\\
{}-ɨd(ɨ)- & {}-ɨd(ɨ) & Second person singular & §8.7\\
{}-ɨk- & – & Distributive adjectival & §8.11.5\\
{}-ɨk(à)- & {}-ɨkᵃ & Plurative & §4.2.1\\
{}-ɨm(ɨ)- & {}-ɨm(ɨ) & First plural exclusive & §8.7\\
{}-ɨma\'{} - & {}-ɨma\'{}  & First pl. exc. optative & §8.10.3\\
{}-ɨma\'{} - & – & First pl. exc. sequential & §8.10.7\\
{}-ɨn(ɨ)- & {}-ɨn & Possessive plural & §4.2.4\\
{}-ɨnɔs(ɨ)- & {}-ɨnɔs & Reciprocal & §8.6.4\\
{}-ɨnɛ\'{}  & {}-ɨnɛ\'{}  & First singular optative & §8.10.3\\
{}-ɨnɨ- & {}-ɨn & Third plural sequential & §8.10.7\\
{}-ɨsɨn(ɨ)- & {}-ɨsɨn(ɨ) & First plural inclusive & §8.7\\
{}-ɨt- & – & Causative & §8.6.5\\
{}-ɨt(ɨ)- & {}-ɨt(ɨ) & Second person plural & §8.7\\
{}-ɨtɨn(ɨ)- & {}-ɨtɨn & Plurative & §4.2.1\\
{}\'{-}m- & – & Middle I mood & §8.6.3\\
{}\'{-}m- & – & Physical property II & §8.11.3\\
{}-nànès(ì)- & {}-nànès & Behaviorative & §8.3.2\\
\lspbottomrule
\end{tabularx}

\begin{tabularx}{\textwidth}{XXXX}
\lsptoprule

Non-final & Final & Name & Section\\
{}-o & {}-ᵒ & Ablative case & §7.6\\
{}-o & {}-ᵒ & Instrumental case & §7.7\\
{}-o & {}-kᵒ & Copulative case & §7.8\\
{}-o & {}-kᵒ & Sequential aspect & §8.10.7\\
{}\'{-}ò & {}\'{-}dᵉ & Dummy pronoun & §8.8\\
{}-òn(ì)- & {}-òn & Intransitive infinitive & §8.2.1\\
{}-ós(í)- & {}-ós & Passive mood & §8.6.1\\
{}-ɔ & {}-ᵓ & Ablative case & §7.6\\
{}-ɔ & {}-ᵓ & Instrumental case & §7.7\\
{}-ɔ & {}-kᵓ & Copulative case & §7.8\\
{}-ɔ & {}-kᵓ & Sequential aspect & §8.10.7\\
{}\'{-}ɔ & {}\'{-}dᵉ & Dummy pronoun & §8.8\\
{}-ɔm(a)- & {}-ɔm & Singulative & §4.2.3\\
{}-ɔn(ɨ)- & {}-ɔn & Intransitive infinitive & §8.2.1\\
{}-ɔs(ɨ)- & {}-ɔs & Passive mood & §8.6.1\\
{}-uƙot(í)- & {}-uƙotⁱ & Andative directional & §8.4.2\\
{}-uƙot(í)- & {}-uƙotⁱ & Completive aspect & §8.5.2\\
{}-úó & {}-ú & Imperative plural & §8.10.5\\
{}-ʉƙɔt(ɨ)- & {}-ʉƙɔtᶤ & Andative directional & §8.4.2\\
{}-ʉƙɔt(ɨ)- & {}-ʉƙɔtᶤ & Completive aspect & §8.5.2\\
{}-ʉɔ & {}-ʉ & Imperative plural & §8.10.5\\
\lspbottomrule
\end{tabularx}

\section{Bibliography}

\begin{verbatim} 
Schrock, Terrill. 2014. \textit{A grammar of Ik (Icé-tód): Northeast Uganda’s last thriving Kuliak language.} Utrecht: LOT.

Schrock, Terrill. 2015. \textit{A guide to the developing orthography of Icetod}. SIL Uganda and the Ik Agenda Development Initiative. 


\end{verbatim}