\section{Adverbs}\label{sec:9}



\subsection{Overview}\label{sec:9.1}


The word class called \textsc{adverbs} is a catch-all category that includes words and clitics of various sorts that say something descriptive about a whole clause, for example, ‘how’ or ‘when’ it takes place, or how the speaker feels about the \isi{certainty} or contingency of the clause. Accordingly, Ik adverbs can be divide up into \textsc{manner} adverbs, \textsc{temporal} adverbs, and \textsc{epistemic} adverbs. The following subsections take up each of these adverbial categories in a brief discussion.




\subsection{Manner adverbs}\label{sec:9.2}


\textsc{manner} adverbs modify whole clauses by commenting on, for example, the manner in which a state comes across or in which an action is done. Manner adverbs usually come near or at the end of the clause they modify, as shown in example sentences \REF{ex:adv:1}-\REF{ex:adv:2} below. \tabref{tab:adv:manner} presents a sampling of these adverbs:


\begin{table}
\caption{Ik manner adverbs}
\label{tab:adv:manner}
\begin{tabularx}{.66\textwidth}{XX}
\lsptoprule
ɗ\`{ɛ}m\`{ʉ}s\`{ʉ} & ‘fast, quickly’\\
hɨ{\Í}ʝ\'{ɔ} & ‘carefully, slowly’\\
ʝíìkì & ‘always’\\
ʝ{\Í}k{\Ì} & ‘really, totally’\\
kɔntɨákᵉ & ‘straightaway’\\
m\`{ʉ}kà & ‘completely, forever’\\
pákà & ‘indefinitely’\\
zùkù & ‘very’\\
\lspbottomrule
\end{tabularx}
\end{table}



\ea\label{ex:adv:1}
\gll Gaana   mɛna=díí     \textbf{zuku}   \textbf{ʝ{\Í}k\ᶤ}. \\
bad:\textsc{3sg}   issues:\textsc{nom}=those   very   completely    \\
\glt ‘Those issues are really very bad!’ 
\z




\ea\label{ex:adv:2}
\gll Zízaaƙótùò       ròɓà     \textbf{m\`{ʉ}kà}. \\
fat:\textsc{dist:comp:3sg:seq}   people:\textsc{nom}   completely    \\
\glt ‘And the people fattened up completely!’ 
\z






\subsection{Temporal adverbs}\label{sec:9.3}
\subsubsection{Overview}\label{sec:9.3.1}

The Ik \textsc{temporal} adverbs situate their clause somewhere in the course of time. Ik has sets of temporal adverbs that deal with past \isi{tense}, past perfect \isi{tense}, and non-past (including future) \isi{tense}. The past and past perfect \isi{tense} adverbs are enclitics that come directly after the verb they modify. The future \isi{tense} adverbs are free adverbs that come near the end or at the end of the clause. 


\subsubsection{Past \isi{tense} adverbs (\textsc{pst})}\label{sec:9.3.2}

Ik divides \textsc{past tense} into four time periods and marks them with adverbial enclitics. They are: 1) \textsc{recent past} that covers the current day and is marked with =\textit{nákà}, 2) \textsc{removed past} that covers yesterday (or any last or ‘yester-’ time period like 'yesterday' or 'yesteryear') and is marked with =\textit{bàtsè}, 3) \textsc{remote past} that covers a few days or weeks before yesterday and is marked with =\textit{nótsò}, and finally, 4) \textsc{remotest past} that covers everything before the remote past and is marked with =\textit{nòkò}. Each of these \isi{tense} enclitics comes directly after ther verb and has a non-final and final form. \tabref{tab:adv:past} illustrates the Ik \isi{tense} markers in all their forms, and examples \REF{ex:adv:3}-\REF{ex:adv:4} illustrate their typical post-verbal position in a sentence:


\begin{table}
\caption{Ik past \isi{tense} markers}
\label{tab:adv:past}


\begin{tabularx}{\textwidth}{XXXX}
\lsptoprule

& \textsc{nf} & \textsc{ff} & \\
\midrule
Recent & =náà & =nákᵃ & ‘earlier today’\\
Removed & =bèè & =bàtsᵉ & ‘last/yester-’\\
Remote & =nótsò & =nótsò & ‘a while ago’\\
Remotest & =nòò & =nòkᵒ & ‘long ago’\\
\lspbottomrule
\end{tabularx}
\end{table}



\ea\label{ex:adv:3}
\gll Ƙ{\'{a}á}=\textbf{bee}   abáŋa     sáás\`{ɔ}s{\Ì}n. \\
go:\textsc{3sg}=\textsc{pst2}   my.father:\textsc{nom}   yesterday    \\
\glt ‘My father went yesterday.’ 
\z




\ea\label{ex:adv:4}
\gll Maráŋa=\textbf{noo}   ɦyekesa   Icé. \\
good:\textsc{3sg}=\textsc{pst4}   life:\textsc{nom}   Ik:\textsc{gen}    \\
\glt ‘The life of the Ik was good (back then).’ 
\z




\subsubsection{Past perfect \isi{tense} adverbs (\textsc{pst.prf})}\label{sec:9.3.3}

The past \isi{tense} can be combined with a perfect \isi{aspect} to yield the \textsc{past perfect} \isi{tense}. Unlike the simple past \isi{tense} adverbs, Ik past perfect \isi{tense} adverbs operate along only three periods of time: \textsc{recent} (earlier today), \textsc{removed} (yester-), and \textsc{remote} (before yester-). \tabref{tab:adv:prf} presents the Ik past perfect \isi{tense} adverbs, and example sentences \REF{ex:adv:5}-\REF{ex:adv:6} illustrate their use in natural contexts:


\begin{table}
\caption{Ik past perfect \isi{tense} markers}
\label{tab:adv:prf}


\begin{tabularx}{\textwidth}{XXXl}
\lsptoprule

& \textsc{nf} & \textsc{ff} & \\
\midrule
Recent & =nanáà & =nanákᵃ & ‘had {\dots} earlier today’\\
Removed & =nàtsàm\`{ʉ} & =nàtsàm & ‘had {\dots} yester-’\\
Remote & =nànòò & =nànòkᵒ & ‘had {\dots} a while ago’\\
\lspbottomrule
\end{tabularx}
\end{table}



\ea\label{ex:adv:5}
\gll Náa   atsíâdᵉ,     ƙaa=\textbf{nanák\ᵃ}. \\
when  come:\textsc{1sg:dp}   go:\textsc{3sg}=\textsc{pst.prf}    \\
\glt ‘When I came earlier, she had (already) gone.’ 
\z




\ea\label{ex:adv:6}
\gll Tsʼ\'{ɛ}d\'{ɔ}\'{ɔ}=nɛ,   tsʼéíƙotátà=\textbf{nànòk\ᵒ}. \\
then:\textsc{ins}=\textsc{dem}   die.\textsc{3pl:comp}=\textsc{pst.prf}    \\
\glt ‘By that (time), they had died out a while ago.’ 
\z




\subsubsection{Non-past \isi{tense} adverbs}\label{sec:9.3.4}

Ik divides the \textsc{non-past} \isi{tense} into three rather vaguely defined time periods suggested by three adverbs. They are: 1) the \textsc{distended} \textsc{present} that includes just before and just after the present and is expressed by the \isi{adverb} \textit{tsʼ\`{ɔ}\`{ɔ}}, 2) the \textsc{removed future} that includes the \textit{next} future time period (next hour, next day, next year) and is expressed by the \isi{adverb} \textit{táá}, and 3) the \textsc{remote future} expressed by the \isi{adverb} \textit{fàrà} (occasionally \textit{fàrò}). \tabref{tab:adv:npst} arranges these adverbs in a paradigm, while \REF{ex:adv:7}-\REF{ex:adv:8} below illustrates them in natural sentences:


\begin{table}
\caption{Ik non-past \isi{tense} markers}
\label{tab:adv:npst}


\begin{tabularx}{\textwidth}{lXXX}
\lsptoprule

& \textsc{nf} & \textsc{ff} & \\
\midrule
Distended present & tsʼ\`{ɔ}\`{ɔ} & tsʼ\`{ɔ}\`{ɔ} & ‘just/recently/soon’\\
Removed & táá & táá & ‘next\_\_\_\_’\\
Remote & fàrà & fàr & ‘in the future’\\
\lspbottomrule
\end{tabularx}
\end{table}



\ea\label{ex:adv:7}
  \ea
  \gll Atsíá=nàà     \textbf{tsʼ\`{ɔ}\`{ɔ}}.    \\
come:\textsc{1sg=pst}   just    \\
  \glt ‘I just came.’      
  \ex
  \gll {Atsésíà}     \textbf{tsʼ\`{ɔ}\`{ɔ}}.\\
 come:\textsc{int:1sg}   soon    \\
  \glt ‘I will come soon.’
  \z
\z




\ea\label{ex:adv:8}
\gll Atsésíma     \textbf{táá}   baratsᵒ. \\
come:\textsc{1pl.exc}   next   morning:\textsc{ins}    \\
\glt ‘We will come tomorrow (i.e., next morning).’ 
\z






\subsection{Epistemic adverbs}\label{sec:9.4}
\subsubsection{Overview}\label{sec:9.4.1}

The Ik \textsc{epistemic} adverbs express how the speaker feels or thinks about the \isi{certainty} or contingency of the clause. Accordingly, this set of adverbs can be divided into the categories of \textsc{inferential}, \textsc{confirmational}, and \textsc{conditional-hypothetical}. All of the epistemic adverbs are enclitics that follow the verb in normal main clauses, but some of them can also be moved in front of the verb.


\subsubsection{Inferential adverbs (\textsc{infr})}\label{sec:9.4.2}

Ik can communicate a degree of un\isi{certainty} about a situation by means of a set of \textsc{inferential} tense-based adverbs. This sense of making a tentative inference based on an observation can be translated into English with such turns of phrase as ‘Apparently {\dots}’, ‘Maybe {\dots}’, ‘It seems that {\dots}’, ‘must have’, etc. Two of these inferential particles consist of the \isi{proclitic} \textit{ná} plus a past-\isi{tense} \isi{particle}, while the third combines \textit{ná} with the \isi{adverb} \textit{tsamʉ}. \tabref{tab:adv:inf} presents the three inferential adverbial particles in their final and non-final forms. Note that compared to the past-\isi{tense} markers above in \tabref{tab:adv:past}, the inferential time-scale is moved up one notch more recent. Examples \REF{ex:adv:9}-\REF{ex:adv:10} show the Ik inferential adverbs in context. Note that they can be placed before or after the main verb.


\begin{table}
\caption{Ik inferential adverbs}
\label{tab:adv:inf}


\begin{tabularx}{\textwidth}{XXXl}
\lsptoprule

& \textsc{nf} & \textsc{ff} & \\
\midrule
Recent & =nábèè & =nábàtsᵉ & ‘apparently earlier today’\\
Removed & =nátsàm\`{ʉ} & =nátsàm & ‘apparently yester-’\\
Remote & =nánòò & =nánòkᵒ & ‘apparently long ago’\\
\lspbottomrule
\end{tabularx}
\end{table}





\ea\label{ex:adv:9}
  \ea
  \gll Baduƙota=\textbf{nábàts\ᵉ}.  \\
die:\textsc{comp:3sg=infr} \\
  \glt ‘It died, apparently.’    
  \ex
  \gll \textbf{Nábee}    baduƙotᵃ. \\
\textsc{infr}       die:\textsc{comp:3sg}    \\
  \glt ‘Apparently, it died.’ 
  \z
\z





\ea\label{ex:adv:10}
\gll \textbf{Nánoo}   teremátᵃ. \\
\textsc{infr}     separate:\textsc{3pl}    \\
\glt ‘It looks like they separated.’ 
\z




\subsubsection{Confirmational adverbs (\textsc{conf})}\label{sec:9.4.3}

Ik can also issue a confirmation of a state or event by means of a set of \textsc{confirmational} adverbs that are derived from the tensed relative pronouns described back in \sectref{sec:5.7}. When used, these adverbs are placed before the verb, and the verb surfaces in its non-final form, almost like a question rendered in English ‘Why yes, did X \textit{not} happen?’ – meaning that, of course, it \textit{did} happen. These suffixes are first presented in \tabref{tab:adv:conf} and then demonstrated in \REF{ex:adv:11}-\REF{ex:adv:12}:


\begin{table}
\caption{Ik confirmational markers}
\label{tab:adv:conf}


\begin{tabularx}{.66\textwidth}{XXl}
\lsptoprule

Recent & náa & ‘Of course\_\_\_\_earlier today.’\\
Removed & sɨna & ‘Of course\_\_\_\_yester-.’\\
Remote & noo & ‘Of course\_\_\_\_long ago.’\\
\lspbottomrule
\end{tabularx}
\end{table}



\ea\label{ex:adv:11}
  \ea
  \gll Ŋƙáƙóídà=bèè?  \\
eat:\textsc{comp:2sg=pst2}     \\
  \glt ‘Did you eat (it) up?’
  \ex
  \gll \textbf{Sɨna}   ŋƙáƙótíà.\\
\textsc{conf}   eat\textsc{:comp:1sg}    \\
  \glt ‘Yes, of course I did.’
  \z
\z





\ea\label{ex:adv:12}
  \ea
  \gll Dètà=nòò?      \\
bring:\textsc{3sg=pst4}    \\
  \glt ‘Did she bring (it)?’    
  \ex
  \gll \textbf{Nòò}   dètà.  \\
     \textsc{conf}   bring:\textsc{3sg}    \\
  \glt ‘Yes, of course she did.’
  \z
\z





\subsubsection{Conditional-hypothetical adverbs (\textsc{cond}/\textsc{hypo})}\label{sec:9.4.4}

If a state or event has not taken place but \textit{could} or \textit{would} take place, Ik can express that contingency with its \textsc{conditional-hypothetical} adverbs. There are three of these adverbs, but they are used to cover four periods of time. The first \isi{adverb} covers non-past and recent past, the second removed past, and third remote past. These conditional-hypothetical adverbs are presented in \tabref{tab:adv:cond}:


\begin{table}
\caption{Ik conditional-hypothetical adverbs}
\label{tab:adv:cond}


\begin{tabularx}{\textwidth}{XXXl}
\lsptoprule

& \textsc{nf} & \textsc{ff} & \\
\midrule
Non-past & =ƙánàà & =ƙánàkᵃ & ‘would’\\
Recent & =ƙánàà & =ƙánàkᵃ & ‘would have {\dots} earlier today’\\
Removed & =ƙásàm\`{ʉ} & =ƙásàm & ‘would have {\dots} yester-’\\
Remote & =ƙánòò & =ƙánòkᵒ & ‘would have {\dots} a while ago’\\
\lspbottomrule
\end{tabularx}
\end{table}
The conditional-hypothetical adverbs come after the main verb:


\exewidth{(123)}

\ea\label{ex:adv:13}
\gll Tóída=\textbf{ƙánaa}   \'{ɲ}cìè? \\
tell:\textsc{2sg=hypo}     I:\textsc{dat}    \\
\glt ‘You would tell me?’ 
\z




\ea\label{ex:adv:14}
\gll Cɛm{\Í}s{\Í}na=\textbf{ƙánòk\ᵒ}. \\
fight:\textsc{1pl.inc=cond}    \\
\glt ‘We all would have fought.’ 
\z




